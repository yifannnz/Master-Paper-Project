%% 中文摘要页
\begin{ChineseAbstract}
论文中文摘要字数约为300\textasciitilde{}600字,如遇特殊需要字数可以略多,限一页。

论文摘要是论文内容不加注释和评论的简短陈述,一般以第三人称语气写成。摘要的编写应遵循下列原则:
\begin{enumerate}[label={\arabic*)},itemindent=3.5em,leftmargin=0em] 
\item 摘要应具有独立性和自含性,即不阅读论文的全文,就能获得必要的信息。摘要是学位论文的缩影,是学位论文的主要内容、见解、结论简短明了的缩写。
\item 摘要应是一篇完整的短文,可以独立使用,可以引用。
\item 摘要的内容应包含与论文等同量的主要信息,供读者确定有无必要阅读全文,也可供文摘汇编等二次文献采用。
\item 摘要一般应说明研究工作的目的意义、研究方法、研究结果、主要结论及意义、创造性成果和新见解,而重点是结论和创新点。
\item 要用文字表达,不要附图和照片,除了实在无变通办法可用以外,摘要中不用图、表、化学结构式、非公知公用的符号和术语,不要使用表格、公式、上下标以及其他特殊符号,要突出重点,阐述清楚,少用数据表。论文摘要用语力求简洁、准确。原则上300\textasciitilde{}600字。
\end{enumerate}

\ChineseKeywords{摘要;论文;要求;字数;格式(关键词个数为3\textasciitilde{}5个,正式写作请删除此括号)}
\end{ChineseAbstract}

%% 英文摘要页
\begin{EnglishAbstract}

   In environmental economics, environmental resources including environmental quality are categorized as amenity resources. Due to its importance to human welfare, the amenity resources theoretical study and valuation is an ongoing issue at the academic frontier in the environmental economics area.  
  
\EnglishKeywords{Key word 1; Key word 2; Key word 3; $\cdots$}
注:论文的英文摘要应有英文题目和关键词,内容与中文摘要相同,用另页置于中文摘要之后;外文摘要实词在300个左右。


\end{EnglishAbstract}
