\appendix

\chapter{单位}

以下内容可放在附录之内:
\begin{enumerate}[label={(\arabic*)},itemindent=2em,align=left,labelsep=0em]
%\begin{enumerate}[label={(\arabic*)}, align=left, leftmargin=2.5em, labelwidth=2em, labelsep=0.5em]
%		\begin{enumerate}[label={(\arabic*)}, align=left, itemindent=3.5em, leftmargin=0em]
\item 正文内过于冗长的公式推导;
\item 方便他人阅读所需的辅助性数学工具或表格;
\item 重复性数据和图表;
\item 论文使用的主要符号的意义和单位;
\item 程序说明和程序全文;
\item 企业应用证明;
\item 项目鉴定报告;
\item 获奖成果证书;
\item 设计图纸;
\item 程序源代码;
\item 论文发表;
\item 作者简介。
\end{enumerate}

这部分内容可省略。如果省略,删掉此页。

书写格式说明:

标题“附录A 附录内容名称”样式为字体:黑体,英文用Times New Roman字体,居中,加粗,字号:小三,2.41倍行距,段前17磅,段后为16.5磅。

附录正文样式为字体宋体小四,英文用Times New Roman字体小四,两端对齐书写,段落首行左缩进2个字符。1.3倍行距(段落中有数学表达式时,可根据表达需要设置该段的行距),段前0.1行,段后0.1行,1.3倍行距。


示例


\begin{table}
\centering
\caption{表 A.1 国际单位制的基本单位}
\noindent\renewcommand{\arraystretch}{0.9}
\begin{tabular}{C{0.33\textwidth}C{0.34\textwidth}C{0.33\textwidth}}
\toprule
量的名称 & 单位名称 & 单位符号 \\
\midrule
长度 & 米 & m \\
质量 & 千克(公斤) & kg \\
时间 & 秒 & s \\
电流 & 安〔培〕 & A \\
热力学温度 & 开〔尔文〕 & K \\
发光强度 & 坎〔德拉〕 & cd \\
\bottomrule
\end{tabular}
\end{table}
\vspace{-1em}  % 向上收紧一行
\begin{table}
\centering
\caption{表 A.2 国家规定的非国际单位制单位}
\noindent\renewcommand{\arraystretch}{0.9}
\begin{tabular}{C{0.14\textwidth}C{0.2\textwidth}C{0.15\textwidth}C{0.51\textwidth}}
\toprule
量的名称 & 单位名称 & 单位符号 & 换算关系和说明 \\
\midrule
\multirow{3}{*}{时间}
 & 分 & min & 1\,min=60\,s \\
 & {\small [小]时} & h & 1\,h=60\,min=3600\,s \\
 & 天(日) & d & 1\,d=24\,h=86400\,s \\
\multirow{3}{*}{平面角}
& {\small [角]秒} & $''$ & $1'' = (\pi/648000)\,\mathrm{rad}$ \\
 & {\small [角]分} & $'$ & $1' = 1^\circ/60 = (\pi/10800)\,\mathrm{rad}$ \\
 & 度 & $^\circ$ & $1^\circ = (\pi/180)\,\mathrm{rad}$ \\
 & 度 & $^\circ$ & 1$^\circ=(\pi/180)\,rad $\\
旋转速度 & 转每分 & r/min & 1\,r/min=(1/60)\,r/s \\
长度 & 海里 & n\,mile & 1\,n\,mile=1852\,m \\ 
速度 & 节 & kn & 1\,kn=1\,n\,mile/h=(1852/3600)\,m/s\\
 &  &  & (只用于航行) \\
\bottomrule
\end{tabular}
\end{table}