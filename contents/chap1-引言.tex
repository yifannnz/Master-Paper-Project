\chapter{引言}

\section{研究背景及意义}

\subsection{研究背景}



% % \begin{figure} [htb]
% %     \centering
% %     \includegraphics[width=0.6\textwidth]{images/intro/nonnewton_curve.png}
% %     \bicaption{不同类型非牛顿流体剪切率与黏度关系曲线}{Relationship curves between shear rate and viscosity of different types of non-Newtonian fluids}
% %     \label{fig:intro_nonnewton_curve}
% % \end{figure


% % \begin{figure} [htb]
% %     \centering
% %     \includegraphics[width=0.6\textwidth]{images/intro/nonnewton_sample.png}
% %     \bicaption{常见的非牛顿流体}{Common non-Newtonian fluids}
% %     \label{fig:intro_nonnewton_samples}
% % \end{figure}
 
\subsection{研究意义}


\section{研究内容}


% 本文的总体方法由三个逐步递进的部分构成,研究路线如图\figref{fig:intro_overview}所示:

% % \begin{figure}[htb]
% %     \centering
% %     \includegraphics[width=0.6\textwidth]{images/intro/overview.png}
% %     \bicaption{本文研究路线与总体框架示意图}{Overview of the research roadmap and overall framework}
% %     \label{fig:intro_overview}
% % \end{figure}


\section{研究创新点}



\section{论文组织结构}

% \begin{equation}
% \frac{D\rho}{Dt}=-\rho\,\nabla\cdot\mathbf{v},
% \label{eq:ns_continuity}
% \end{equation}
% \begin{equation}
% \rho\frac{D\mathbf{v}}{Dt}=-\nabla p+\nabla\cdot\boldsymbol{\tau}+\mathbf{f}_{\mathrm{st}}+\rho\mathbf{g}+\mathbf{f}_{\mathrm{ext}},
% \label{eq:ns_momentum}
% \end{equation}
