\chapter{引言}

\section{研究背景及意义}
当今时代科技发展日新月异,材料科学作为支撑众多领域发展的核心基础,其战略地位愈发凸显。从航天工程所依赖的耐极端环境、超高强韧的特种金属材料,到微电子行业不可或缺的高迁移率、微型化集成的新型半导体材料,再到生命健康领域所使用的生物兼容、功能适配的植入与诊疗材料,材料本身的性能优劣与制备质量,直接决定了下游产品的整体水平、可靠性以及可应用的场景边界。因此,系统探究材料的本征特性与服役行为,对推动技术创新突破、增强国家核心竞争力,具有至关重要的作用。

材料的微观结构与宏观性能之间关系挖掘是材料科学中的重要话题\cite{ramakrishna2019materials}。研究人员利用多种表征手段从不同尺度开展材料构效关系的研究。比如,在介观尺度下揭示多层纳米结构的界面效应对裂纹传播的影响\cite{butler2018machine};在纳观尺度下表征原子催化活性变化\cite{liu2024site}等。在多种尺度研究中,原子层面的结构表征能精确揭示原子的排列、缺陷等微观信息\cite{banhart2011structural,ziatdinov2017deep,ziletti2018insightful},直接关联材料的结构与性能。这不仅能帮助理解相变、界面效应以及缺陷对材料力学、电学和化学性能的影响,还能进而优化材料设计与应用。因此,要更加深入全面地探索材料的构效关系,提升材料性能,设计符合需求的新材料,原子尺度的观测手段起到了至关重要的作用\cite{tilley2020crystals,muller2009structure}。

\begin{figure} [htb]
    \centering
    \includegraphics[width=0.6\textwidth]{images/chap1/1.png}
    \bicaption{材料微观结构与宏观性能之间关系}{Relationship between microstructure and macroscopic properties of materials}
    \label{fig:1}
\end{figure}

作为揭示材料微观结构和宏观性能之间内在联系的核心手段,现代显微影像表征技术呈现出多样化和高精度发展的趋势。电子显微镜根据成像模式、电子束与样品交互方式和样品厚度以及分辨率的差异分为扫描电子显微镜(SEM)、透射电子显微镜(TEM)和扫描透射电子显微镜(STEM)。这些技术能够识别从毫米到数十皮米的细节,并提供材料在形貌、相态、晶体学以及分子和原子结构等方面的独特信息。随着TEM成像技术及原位表征技术的快速发展,结合像差校正的原位TEM已能够在原子乃至亚埃尺度上直接观测材料的原子排列与结构演化,在揭示材料微观结构、动态行为、物理化学性质及相变机制等方面发挥着不可替代的作用。凭借其超高的空间与能量分辨率,该技术已广泛应用于材料、化学、能源及生命科学等领域,成为研究纳米至原子尺度微观行为及其演化规律的重要手段。

随着计算能力的大幅提升,机器学习和深度学习技术不断突破,AI在不同领域得到了广泛应用,并正在成为科学研究中的核心分析工具。计算机–电子显微术–材料科学的结合应运而生,许多方法也被引入到TEM数据自动化分析任务中。原子定位是晶体或分子结构解析的前置任务,在原子尺度的数据分析中,原子定位的准确性直接决定了后续材料分析的有效性,并为深入研究材料微观结构与性能之间的关系提供技术支撑。

随着图像采集设备的进步,TEM已经能够获得原子级分辨率的图像,提供原子位置、晶格参数和缺陷等信息。然而,在实际成像中,材料的原子结构及缺陷形貌往往呈现出高度复杂的空间分布特征,其边界在噪声、像差及成像模糊等因素的影响下难以清晰界定,给传统的人工分析和规则化处理方法带来了巨大挑战。在材料表征与性能分析过程中,研究人员通常需要对原子列、晶界及缺陷区域进行精确定位与标注,而这一过程高度依赖人工经验与主观判断,容易导致结果不一致,并限制分析精度与重复性。此外,由于高分辨率 TEM 图像及原位实验数据通常具有高维度和大规模特性,人工分析不仅工作量巨大,而且难以满足高通量材料研究的需求。

综上所述,开发一种准确、高效且具有良好泛化能力的原子尺度结构质量提升与分割方法,已成为提升材料显微表征与数据驱动材料研究潜力的重要研究方向。原子分辨率TEM图像中蕴含着丰富的结构信息,通过对原子排列、局域畸变及缺陷形态等特征的定量化分析,能够为理解材料的相变行为、界面效应及性能演化机制提供关键依据。这类基于显微图像的结构特征提取方法,为材料设计、性能优化及新材料发现提供了重要的数据支撑。

目前,得益于卷积神经网络和Transformer架构的突破性进展,深度学习已经成为图像分割的主流方法,但在材料围观结构分析领域,仍存在一些制约性的瓶颈问题亟待解决,具体总结如下:

(1)原子图像质量提升方面。由于设备性能限制、环境干扰及操作条件等因素,TEM原子图像在采集过程中不可避免地会出现退化现象,表现为原子排列模糊、结构信息不完整等问题。传统的视觉增强方法通常因噪声、对比度低和灰度分布不均等影响,对复杂退化图像的适应性较差,难以有效恢复高质量的原子结构信息。深度学习方法虽然在图像增强领域展现出优势,但对大规模标注数据和专业知识高度依赖,在实际应用中面临数据获取困难的挑战。有限数据训练的模型极易出现过拟合现象,在未见过的材料图像上泛化能力较差。此外,现有方法在处理细微对象多、分布广的数据时,长距离依赖关系和全局特征提取能力不足。因此,亟待开发一种精度高、数据依赖性低、泛化性强的图像质量提升方法。

(2)实验TEM图像的精准分割方面。原子定位与分割的精准性直接影响材料微观结构的定量表征和性能分析。当前缺乏公开且完备的实验数据集,现有研究多依赖基于理论模型生成的模拟数据,虽具备高可控性,但难以全面反映真实实验环境下的复杂性。现有分割方法在噪声强烈或对比度低的情况下往往表现不佳,并且通常需要依赖人工设定参数,处理效率低且存在人为误差。CNN模型在细微对象的长距离依赖关系和全局特征捕捉方面能力有限,导致分割精度受限。因此,构建涵盖多种材料体系、多样结构特征的高质量实验数据集,并开发鲁棒的精准分割算法,对于实现低信噪比、高伪影图像的高效处理具有重要意义。

(3)自动化表征分析系统缺失方面。随着高分辨显微影像数据生成速度的提升,传统的TEM图像分析方法在应对复杂样品结构、表征精度及处理效率等方面的局限性日益凸显。目前缺乏集成化的自动化表征分析系统,难以实现从图像获取、质量提升、精准分割到多维度特征提取的全流程智能化处理。开发基于人工智能和机器学习的自动化TEM图像表征分析方法,能够显著提升数据处理效率,增强对复杂样品结构和成分特征的定量分析能力,降低人工成本,提高数据处理的准确性与一致性。

综上所述,针对上述瓶颈问题,本课题旨在构建原子级纳观影像鲁棒分割算法及自动化表征分析系统,围绕图像质量提升、精准分割、自动化表征分析三方面开展深入研究,推动透射电子显微影像分析方法从传统的人工依赖向高通量、智能化、自动化方向转型。这不仅对材料科学研究具有重要的科学价值,还将为深入理解材料的本质机理与性能调控提供更加可靠的依据,对加速材料研发与应用、推动相关领域的技术进步产生深远影响。


\section{研究内容与方法}




\section{论文组织结构}


