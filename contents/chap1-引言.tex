\chapter{引言}

\section{研究背景及意义}
当今时代科技发展日新月异,材料科学作为支撑众多领域发展的核心基础,其战略地位愈发凸显。从航天工程所依赖的耐极端环境、超高强韧的特种金属材料,到微电子行业不可或缺的高迁移率、微型化集成的新型半导体材料,再到生命健康领域所使用的生物兼容、功能适配的植入与诊疗材料,材料本身的性能优劣与制备质量,直接决定了下游产品的整体水平、可靠性以及可应用的场景边界。因此,系统探究材料的本征特性与服役行为,对推动技术创新突破、增强国家核心竞争力,具有至关重要的作用。

材料的微观结构与宏观性能之间关系挖掘是材料科学中的重要话题\cite{ramakrishna2019materials}。如图\figref{fig:1}研究人员利用多种表征手段从不同尺度开展材料构效关系的研究。比如,在介观尺度下揭示多层纳米结构的界面效应对裂纹传播的影响\cite{butler2018machine};在纳观尺度下表征原子催化活性变化\cite{liu2024site}等。在多种尺度研究中,原子层面的结构表征能精确揭示原子的排列、缺陷等微观信息\cite{banhart2011structural,ziatdinov2017deep,ziletti2018insightful},直接关联材料的结构与性能。这不仅能帮助理解相变、界面效应以及缺陷对材料力学、电学和化学性能的影响,还能进而优化材料设计与应用。因此,要更加深入全面地探索材料的构效关系,提升材料性能,设计符合需求的新材料,原子尺度的观测手段起到了至关重要的作用\cite{tilley2020crystals,muller2009structure}。

\begin{figure} [htb]
    \centering
    \includegraphics[width=0.6\textwidth]{images/chap1/1.png}
    \bicaption{材料微观结构与宏观性能之间关系}{Relationship between microstructure and macroscopic properties of materials}
    \label{fig:1}
\end{figure}

作为揭示材料微观结构和宏观性能之间内在联系的核心手段,现代显微影像表征技术呈现出多样化和高精度发展的趋势。电子显微镜根据成像模式、电子束与样品交互方式和样品厚度以及分辨率的差异分为扫描电子显微镜(SEM)、透射电子显微镜(TEM)和扫描透射电子显微镜(STEM)。这些技术能够识别从毫米到数十皮米的细节,并提供材料在形貌、相态、晶体学以及分子和原子结构等方面的独特信息。随着TEM成像技术及原位表征技术的快速发展,结合像差校正的原位TEM已能够在原子乃至亚埃尺度上直接观测材料的原子排列与结构演化,在揭示材料微观结构、动态行为、物理化学性质及相变机制等方面发挥着不可替代的作用。凭借其超高的空间与能量分辨率,该技术已广泛应用于材料、化学、能源及生命科学等领域,成为研究纳米至原子尺度微观行为及其演化规律的重要手段。

随着计算能力的大幅提升,机器学习和深度学习技术不断突破,AI在不同领域得到了广泛应用,并正在成为科学研究中的核心分析工具\cite{lecun2015deep}。计算机–电子显微术–材料科学的结合应运而生,许多方法也被引入到TEM数据自动化分析任务中。原子定位是晶体或分子结构解析的前置任务,在原子尺度的数据分析中,原子定位的准确性直接决定了后续材料分析的有效性,并为深入研究材料微观结构与性能之间的关系提供技术支撑。

随着图像采集设备的进步,TEM已经能够获得原子级分辨率的图像,提供原子位置、晶格参数和缺陷等信息。然而,在实际成像中,材料的原子结构及缺陷形貌往往呈现出高度复杂的空间分布特征,其边界在噪声、像差及成像模糊等因素的影响下难以清晰界定,给传统的人工分析和规则化处理方法带来了巨大挑战。在材料表征与性能分析过程中,研究人员通常需要对原子列、晶界及缺陷区域进行精确定位与标注,而这一过程高度依赖人工经验与主观判断,容易导致结果不一致,并限制分析精度与重复性。此外,由于高分辨率 TEM 图像及原位实验数据通常具有高维度和大规模特性,人工分析不仅工作量巨大,而且难以满足高通量材料研究的需求。

综上所述,开发一种兼具高精度、高效率与良好泛化能力的原子尺度结构质量提升与分割方法,已成为提升材料显微表征能力与数据驱动材料研究水平的关键方向。原子分辨率TEM图像蕴含丰富的结构信息,通过对原子排列、局域畸变及缺陷形态等特征的定量化提取,可为揭示材料相变行为、界面效应及性能演化机制提供重要依据,并为材料设计与性能优化提供数据支撑。目前,得益于卷积神经网络和Transformer架构的突破性进展,这些技术的应用为材料微结构分析带来了技术提升,但仍存在一些制约性的瓶颈问题亟待解决,具体总结如下:

(1)TEM图像普遍存在质量退化且现有方法难以兼顾精度与泛化性。受设备性能、成像条件及环境干扰等因素影响,实验TEM原子图像在采集过程中常出现噪声增强、对比度不足及结构模糊等退化现象。传统图像增强方法对复杂退化模式的适应能力有限,难以有效恢复精细原子结构。尽管深度学习方法在图像质量提升任务中展现出优势,但其通常依赖大量成对标注数据,在实验场景中面临数据获取困难及模型泛化能力不足的问题。此外,现有模型在刻画原子级细微结构时,对长距离依赖关系和全局结构特征的建模能力仍显不足。

(2)实验TEM图像中原子级目标的精准分割仍面临鲁棒性与精度不足的问题。原子分割的准确性直接影响材料微观结构的定量分析结果,但目前公开可用的高质量实验TEM标注数据较为匮乏,相关研究多依赖模拟数据进行验证,难以充分反映真实实验条件下的复杂成像特性。在低信噪比和强伪影条件下,现有分割方法对噪声较为敏感,分割稳定性和精度显著下降。同时,基于卷积神经网络的模型在捕捉细微原子结构的长距离依赖关系和全局上下文信息方面存在局限,从而制约了分割性能的进一步提升。

(3)当前TEM图像分析流程缺乏系统化的自动化表征分析框架。随着高分辨率TEM成像数据规模的持续增长,传统依赖人工或半自动方式的分析流程在处理效率、结果一致性及可扩展性方面逐渐显现不足。目前尚缺乏能够将图像质量提升、精准分割与特征提取等关键环节进行统一整合的自动化表征分析系统,限制了TEM数据的高效利用。构建基于人工智能的自动化TEM图像表征分析框架,对于提升材料微结构定量分析效率和可靠性具有重要意义。

基于以上问题,本研究针对原子尺度TEM图像质量提升与精准分割技术开展深入探索。从基于结构保持的原子质量提升和多源特征联合优化的原子定位入手,提升原子级细微结构的恢复能力和分割精度。通过构建自动化分析系统实现高通量TEM数据的智能化处理与分析,促进相关领域从传统人工分析向智能化、高效化分析模式转变,具有重要的科学价值和应用前景。


\section{研究内容与方法}
准确的原子级纳观影像分割是进行材料微观结构定量分析的关键。针对1.1节所描述的关键问题与挑战,本文主要研究内容如图\figref{fig:2}所示,主要包括:

\begin{figure} [htb]
    \centering
    \includegraphics[width=1.0\textwidth]{images/chap1/2.png}
    \bicaption{主要研究内容与方法}{Main research content and methods}
    \label{fig:2}
\end{figure}

(1)本文提出了一种基于结构保持约束的非配对TEM图像质量提升方法。通过构建空间域与频域联合特征提取网络,并结合自注意力机制实现全局结构建模与细节增强。同时,引入特征函数约束与频域一致性约束,从生成机制层面强化原子尺度结构的保真性,在提升图像清晰度与对比度的同时有效抑制结构偏移与伪影生成,为后续原子级精准分割提供更加可靠的高质量输入。

(2)本文提出了基于多源特征联合优化的原子定位方法。设计了基于生成引导的语义一致性分割方法,通过引入生成器提取的高质量特征与分割网络进行联合优化,提升原子分割的鲁棒性与精度。同时,针对原子定位任务中前景与背景重要性不均的问题,引入基于掩膜的加权循环一致性损失,强化生成过程中对原子区域结构保真的约束,进一步提升分割结果的准确性和稳定性。

(3)在此基础上,面向原子级纳观影像的定量分析中,高通量时序图像的关键信息批量提取的自动化程度低且集成性差的问题,本文设计开发了原子级视频信息的动态提取分析系统。集成了原子定位与运动追踪、多原子关系建模与行为识别、关键参量计算与运动特征提取等,实现了从原子级视频数据中高效、系统地提取动态结构信息,为材料微观结构与性能关系的深入研究提供了强有力的工具支持。


\section{论文组织结构}

本文一共分为六个部分,如图\figref{fig:3}所示,各章的主要内容安排如下。

\begin{figure} [htb]
    \centering
    \includegraphics[width=1.0\textwidth]{images/chap1/3.png}
    \bicaption{论文组织结构}{Paper Organization Structure}
    \label{fig:3}
\end{figure}

第一章:引言。简要阐述了文章的材料背景与研究意义,介绍了材料微观结构研究对于材料科学进步的重要性,对本文的研究成果做出了一定说明,并对论文的结构做了简单的介绍。

第二章:文献综述。本章分别介绍了传统方法和深度学习方法在图像质量提升与图像分割领域的研究进展,总结了现有方法的不足,并对本文涉及的相关技术进行了说明。在此基础上,归纳现有研究对本论文的启示,明确本文的创新点与研究方向,为后续方法的提出提供理论支撑与参考依据。

第三章:基于结构保持的非配对图像质量提升方法。本章首先通过空域频域联合特征提取及结构性约束设计,提升原子TEM图像质量。

第四章:基于多源特征联合优化的原子定位方法。通过生成引导的语义一致性分割方法和基于掩膜的加权循环一致性损失,提升原子分割的鲁棒性与精度。

第五章:原子级视频信息动态提取分析系统。分为自动化原子位置识别与处理模块和原子细粒度信息动态提取模块。设计开发了自动化原子位置识别与处理模块和细粒度信息动态提取模块,实现了从原子级视频数据中高效、系统地提取动态结构信息。

第六章:总结与展望。对本文的工作进行总结,对当前存在的不足进行分析,以及对未来的工作改进提出思路。
