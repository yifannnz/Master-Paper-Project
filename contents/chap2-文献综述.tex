\chapter{文献综述}

\section{原子级高分辨图像质量提升算法概述}
随着图像采集设备的进步,TEM已经能够获得原子级分辨率的图像,提供原子位置、晶格参数和缺陷等信息。然而,图像质量常受仪器失真、噪声、设备漂移和电子束损伤等影响,导致信噪比低和图像模糊,影响原子结构分析。传统的线性滤波方法能减少高斯噪声,但在保留图像边缘方面效果不佳。为应对复杂噪声和细节保留问题,深度学习算法为图像去噪和超分辨率处理提供了保真性更强的解决方案。

\subsection{基于规则化方法的质量提升}
传统的规则化图像去噪模型分为线性模型和非线性模型两类。在TEM数据去噪处理中,边缘细节的保留至关重要。然而,线性模型通常难以有效地保持图像的边缘特征,即将边缘作为图像中的不连续性时容易导致边缘模糊或扩散。相比之下,非线性模型在边缘保留方面表现较好,能够更好地保持图像的边缘细节结构。

Du等提出了一种非线性滤波算法,用于去除高分辨扫描透射电子显微镜图像中的噪声,能够有效抑制噪声同时保留细节,尤其在低剂量电子束下的噪声环境中表现良好。随后,Mevenkamp等改进了基于块匹配和3D滤波算法,将泊松噪声转化为高斯噪声并结合块匹配策略,提高了低剂量条件下的去噪效率和图像细节保留性。两者都在去噪效果和细节保留上有显著提升,但都存在计算复杂和模型依赖性的问题,限制了实时应用。HRTEM filter插件基于R. Kilaas的滤波方法,用于HRTEM和扫描透射电子显微镜(Scanning transmission electron microscopy,STEM)数据中的无定形背景去噪。该插件自动应用Wiener滤波(恢复晶体结构信息)和自适应背景减法滤波(减去频域中的平均背景),有效去除由无定形污染层带来的信号干扰。通过自动调整滤波参数,HRTEM filter能提高图像的信噪比和晶体结构细节,使原子排列和晶体缺陷的成像更清晰,从而提升后续数据分析的精确性和可靠性。
 
\subsection{基于深度学习方法的质量提升}
规则化方法在处理成像风格不同、噪声多样等条件下仍存在局限性,需要大量的参数调整与适配,且效果无法得到保证。近年来,随着人工智能和深度学习技术的发展,深度学习被应用于TEM数据处理,增强了去噪的泛化能力和鲁棒性。

AtomSegNet是专为TEM数据开发的深度学习模型,旨在实现高精度的原子分割,并提供了标准化数据集TEMImageNet。通过构建大规模的模拟数据集,涵盖了不同材料的原子分辨率ADF-STEM数据,引入多种噪声模型,以增强其对实验图像的适应性。AtomSegNet采用基于U-Net架构的编码器-解码器结构,能有效去除背景噪声并保留原子级细节。实验表明,AtomSegNet在高噪声环境下依然能准确去噪并保持原子结构的清晰度。然而,模型训练高度依赖合成数据,实验图像中的非线性失真和电子源不稳定等因素可能影响其实际应用效果。

在AtomSegNet基础上,Mohan等提出了一个仿真驱动的深度学习去噪框架(Simulation-based denoising,SBD),如图3(a)。该方法通过仿真生成大规模无噪声图像数据集并加入实验噪声,以解决实验数据缺乏的问题。SBD框架使用U-Net架构,捕捉图像中的周期性结构,显著提高了低信噪比数据的去噪效果。然而,即便是仿真驱动的方法仍然依赖大量的模拟数据,且难以避免与真实实验数据之间存在的差异。为进一步摆脱对任何形式“干净”图像的依赖,Crozier等提出了一种完全无监督的深度学习去噪框架,发表在2025年的Science期刊上。在高时间分辨率下,为了避免样品受损,需要使用低电子剂量,但低剂量会导致信噪比降低,遮蔽原子结构细节;连续帧拍摄导致噪声在时序上累积,影响对快速结构演化的观测,而普通图像叠加或滤波方法难以同时保持高分辨率和去噪能力。文章通过使用多帧时序上下文对每一帧进行还原,缓解了帧间运动模糊的问题。该框架的核心结构为盲点(Blind spot)卷积神经网络,即UDVD(Unsupervised deep video denoising)模型,如图3(b)。该模型结合了多帧信息和U-Net架构,在空间和时间维度上捕捉关键结构特征,在训练过程中排除了目标像素自身的值,仅利用其时空邻域像素信息进行估计,从而有效避免模型过拟合。该方法的提出解决了实验数据稀缺和仿真不准的问题,但该方法依赖于高质量的时间-空间一致性,且结构相对复杂,计算成本较高。因此,为了解决显微图像中间帧不一致与细节过度平滑等问题,He等提出了一个支持实时处理的、适用于时序显微图像的零样本降噪与超分辨框架MDSR-Zero,如图3(c)。该方法完全不依赖干净图像或仿真数据,通过高效的在线训练策略(Efficient online training,EOT)显著加快了模型训练速度并提高了时间一致性。EOT利用前一帧的模型参数作为当前帧的初始化,保留跨帧结构信息,使得模型学习具有时间一致性的结构表征;引入指数滑动平均机制平滑更新模型权重,抑制帧间噪声波动,从而有效解决了帧间结构无法统一建模导致的原子区域边界模糊、细节流失、图像不稳定的问题。此外,作者还设计了一种面向超分辨率的自监督损失函数,在保持低分辨率重建准确性的同时,引导模型关注高频邻域信息,增强了图像细节恢复能力。该方法不仅在合成噪声和真实显微视频数据上表现出优越的去噪与结构保真性能,而且相比现有零样本方法训练效率提高近10倍,推理延迟低,可做到边采集边处理,展示出良好的实用性与扩展性。上述方法分别从数据生成方式、网络结构设计和训练策略上提供了不同的技术路线,适用于不同类型和需求的显微图像分析任务,为真实复杂场景下的后续任务提供了坚实的基础。然而,这些方法在提升特定场景下性能的同时,也暴露出一定的泛化能力局限性,主要体现在难以全面覆盖实验图像中多样化的噪声模式和结构变化,导致在跨设备、跨样本或复杂成像条件下的适应性和鲁棒性不足。


\subsection{基于GAN的非配对图像质量提升}


\section{原子级高分辨图像原子定位算法概述}
电镜实验在获取和分析原子级结构时面临速度慢、数据少等问题,且数据往往缺乏代表性。随着新型CCD相机的出现和数据储存速度的提升,电镜实验产生的数据量急剧增加,如何从海量数据中自动化提取细粒度信息成为主要挑战。原子定位是晶体或分子结构解析的前置任务。传统的规则化方法对图像噪声较为敏感,且依赖人工调参或后期修正,难以满足大规模自动化分析需求。深度学习模型通过自动提取图像特征,具有更好的鲁棒性,能够在低对比度和高噪声的条件下输出可用结果。

\subsection{基于规则化方法的原子定位方法}
Galindo等[38]于2007年提出了峰值对法,通过布拉格滤波或维纳滤波降噪后,利用灰度水平定位峰值并计算位移场。Wen等[39]结合子集相位分析、寻峰法和最优逼近算法,能够准确确定多个界面层中的原子位置,并测量超晶格结构的应变分布。Zhang等[40]提出的多椭圆拟合方法进一步提高了原子位置的精度,通过拟合原子柱的等高线为椭圆,确定原子位置,该方法提升了对不规则原子柱的刻画能力,但较为依赖图像质量,存在一定局限性。

随之开发的自动化图像处理软件工具如Atomap[37]和CalAtom[36],也推动了自动化分析原子分辨率图像的进程。Magnus Nord研发的Atomap软件,利用基于模型的2D高斯分布自动量化原子分辨率STEM数据中原子列的位置和形状,在钙钛矿氧化物异质结构的图像上进行测试,定量分析了图像中氧原子柱的位移。CalAtom软件工具提供了三种高精度确定原子柱位置的算法:矩量法、基于模型的方法和多椭圆拟合方法。这些工具减少了手动干预的需求,使得研究者可以更高效地处理大量原子级图像数据,从而提高了分析效率和准确性。

 
\subsection{基于深度学习方法的原子定位方法}
基于深度学习的方法能够进一步提升原子定位的准确率,提升大规模图像数据处理的效率和一致性。在此基础上,能够识别复杂材料中的缺陷和局部结构变化,并将图像分析结果与材料的物理和化学特性关联起来。表1按照不同训练策略对原子定位的深度学习方法进行分类。分析发现,大多工作基于监督学习展开,在模拟数据上训练、实验数据上测试。



\section{性能评价指标概述}


\section{本章小结}