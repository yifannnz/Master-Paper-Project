\chapter{文献综述}

\section{图像质量提升算法概述}

\subsection{传统方法}

% 计算机图形学作为连接数字世界与现实世界的重要桥梁,已广泛应用于影视特效、游戏交互、工业设计与科学可视化等领域。随着用户对沉浸感与真实感期望的提升,仅依赖美术规则对复杂自然现象进行“外观拟合”难以满足需求:一方面,大尺度场景与细节交互(飞溅、卷吸、混合、界面破碎等)会显著放大伪影;另一方面,影视与交互式应用又要求在有限算力下稳定地产生可信结果。因此,基于物理的仿真方法成为提升视觉真实性与可控性的核心手段之一。

% 在物理仿真研究中,流体模拟始终是计算机图形学的热点方向。流体具有强非线性、强耦合和多尺度特征,其复杂性既来自运动方程本身,也来自材料属性与相互作用机制。按组成可分为单相流体与多相流体:单相流体可近似视为均质连续介质;多相流体则由多种组分或状态共同构成,需要同时刻画各相的运动与相间耦合、扩散/漂移与混合过程,数值稳定性与计算开销显著提升。

% 按流变学特征,流体可分为牛顿流体与非牛顿流体。牛顿流体在一定条件下黏度近似为常数;而非牛顿流体的黏度随剪切率呈非线性变化(如图\figref{fig:intro_nonnewton_curve}),并常伴随屈服、触变、剪切变稀/变稠与黏弹性等复杂效应。现实世界中大量“高黏、可塑、可混合”的材料(血液、巧克力浆、油漆、玻璃熔融物等,如图\figref{fig:intro_nonnewton_samples})本质上属于多组分混合体系,其宏观流变行为与局部浓度、结构演化密切相关。

% % \begin{figure} [htb]
% %     \centering
% %     \includegraphics[width=0.6\textwidth]{images/intro/nonnewton_curve.png}
% %     \bicaption{不同类型非牛顿流体剪切率与黏度关系曲线}{Relationship curves between shear rate and viscosity of different types of non-Newtonian fluids}
% %     \label{fig:intro_nonnewton_curve}
% % \end{figure}

% 进一步地,在图形学流体仿真中,“非牛顿”往往不仅意味着“黏度模型更复杂”,还意味着材料属性可能随时间与空间动态变化:例如混合、溶解、聚合/凝胶化等过程会改变局部结构,从而改变黏度与松弛时间等关键参数。传统单相非牛顿仿真多通过固定本构参数(如剪切变稀模型参数)来得到某一种材料的外观效果,但对于“由多相混合生成非牛顿行为”的场景,这类固定参数方式难以反映局部流变的差异,更难以稳定处理高黏度与大时间步条件下的数值刚性。

% % \begin{figure} [htb]
% %     \centering
% %     \includegraphics[width=0.6\textwidth]{images/intro/nonnewton_sample.png}
% %     \bicaption{常见的非牛顿流体}{Common non-Newtonian fluids}
% %     \label{fig:intro_nonnewton_samples}
% % \end{figure}

% 本研究的另一个重要动机来源于医疗交叉应用对“可解释、可控、可复现”的流体仿真需求。以医疗场景中骨水泥注射手术为例,骨水泥在调配与注射过程中经历显著的混合与聚合反应,其流动状态随时间快速变化,临床操作需要在合适窗口期完成注射。若骨水泥过稀可能造成渗漏并引发并发症;过稠则可能导致堵塞、填充不足等风险。由于此类过程高度依赖经验且真实试验成本高,面向教学训练与术前方案评估的物理仿真具有现实意义:它可以支持在不同骨结构、裂隙形态与操作参数下对潜在流动路径进行观察与比较,从而辅助优化方案、降低风险。

% 围绕上述需求,本文形成了一条以SPH为离散基础、以混合模型处理多相传输、以构象张量描述黏弹性本构关系的技术路线:在医疗骨水泥场景中验证了高黏混合流体的稳定可视化模拟;在更一般的多相黏弹性非牛顿场景中构建了统一的变量与求解框架;并进一步发展到可扩展的统一多相黏弹性求解器,以支持更广的材料类型与更大的黏度范围。
 
\subsection{深度学习方法}

% 目前,图形学中针对单相牛顿流体的仿真技术已较为成熟,但对于多相、高黏度、黏弹性与强非线性流变的复杂体系,仍缺乏同时兼顾稳定性、效率与统一性的通用方法。

% 在本课题中,面向多相黏弹性流体构建统一的本构表示与数值框架,有助于弥合图形学与流变/多相流理论之间的鸿沟。一方面,多相混合模型为相分数与相间耦合提供了简洁的描述;另一方面,构象张量等黏弹性工具能够以统一形式刻画弹性应力与松弛过程。将二者在同一SPH离散框架下稳定耦合,并处理“局部流变随浓度变化”的问题,可为复杂材料的可视化仿真提供可推广的基础。

% 另外,高效稳定的多相黏弹性求解器能够支撑更广泛的工业与医疗交叉场景。例如骨水泥调配与注射过程兼具多相混合、非牛顿流变和随时间变化的材料性质,属于典型的“高风险、强约束、难以反复试验”的应用问题。可视化仿真系统能够用于教学训练、术前方案评估与参数敏感性分析,降低真实试验成本并提升流程安全性。

% 基于SPH的无网格拉格朗日方法在处理大变形、自由表面与复杂边界方面具有天然优势;通过在SPH框架中引入多相混合与黏弹性本构,可以在统一管线中覆盖牛顿、非牛顿及黏弹性材料,提升效果复用性与开发效率。SPH方法相较于其他传统方法,具有以下优势:

% \begin{enumerate}
%     \item 边界处理的优越性:与欧拉方法不同,SPH方法能够自然地满足材料边界的精细划分。
%     通过追踪SPH粒子的位置变化来表示流体运动现象,由于每个粒子的计算是相对独立的,这使得
%     位于自由表面或材料交界处的粒子运动特征易于捕捉。因此,SPH方法特别适合处理多材料间的
%     耦合问题;
%     \item 简便灵活的实现方式:相比于欧拉法和某些混合方法,SPH方法拥有较为简单且直观的计算
%     框架和编程实现。此外,SPH框架的灵活性允许它轻松与其他方法结合,以优化数值误差和视觉
%     伪影等问题,从而提高模拟效果的真实性和准确性;
%     \item 天然的质量守恒特性:作为无网格的拉格朗日方法,SPH方法从本质上保证了质量守恒定律
%     的遵守,这对于保持物理模拟的准确性至关重要;
%     \item 广泛的适用性:SPH方法不仅适用于流体模拟,还可以扩展到弹塑性固体、沼泽、雪地、
%     海绵等复杂介质的建模。由于SPH粒子属性的独立计算特性,该方法也能有效模拟涉及耦合和热
%     传导的现象。
% \end{enumerate}

% 鉴于图形学领域在多相高黏与黏弹性材料模拟方面仍存在方法统一性不足与稳定性受限等问题,同时考虑到医疗材料(如骨水泥)等场景对仿真可信度与可控性的迫切需求,本文旨在构建一个基于SPH的多相黏弹性流体仿真方法与系统,为多相多类型流体的统一建模、稳定求解与可视化呈现提供支撑。

\section{研究内容}

% 本研究面向基于物理的多相黏弹性流体仿真,主要聚焦于通用与非牛顿流体模拟。现有图形学流体模拟研究中,单相牛顿流体已形成较成熟的数值与工程体系;而对于“多相耦合 + 高黏度范围 + 黏弹性本构 + 局部流变随浓度动态变化”的复杂材料场景,仍缺乏同时兼顾稳定性、效率与统一性的通用方案。针对这一问题,本文以拉格朗日粒子方法SPH为统一离散基础,通过引入混合模型(Mixture Model/Implicit Mixture Model)处理多相传输与相间耦合,并结合构象张量等黏弹性本构工具建立可扩展的统一求解框架。

% 本文的总体方法由三个逐步递进的部分构成,研究路线如图\figref{fig:intro_overview}所示:

% % \begin{figure}[htb]
% %     \centering
% %     \includegraphics[width=0.6\textwidth]{images/intro/overview.png}
% %     \bicaption{本文研究路线与总体框架示意图}{Overview of the research roadmap and overall framework}
% %     \label{fig:intro_overview}
% % \end{figure}

% \begin{enumerate}
%     \item \textbf{两相流体模拟器:}面向两相(例如溶剂/聚合物)混合与相分数演化过程,构建稳定的两相黏弹性非牛顿流体求解器。该部分以隐式混合模型为多相基础,采用构象张量计算黏弹性应力,并引入浓度驱动的相间耦合调控机制,以刻画“混合生成非牛顿”的动态过程,为后续应用提供可靠的基础求解能力。
%     \item \textbf{任意多相流体模拟器:}在两相框架的基础上,进一步扩展到任意相数与多类型材料的统一描述与稳定求解。该部分以多模式/多相构象张量表示不同相的本构参数与流变特性,配合混合模型中的相分数传输,实现对更广材料范围(牛顿/非牛顿/黏弹性)及更大黏度对比条件下的稳定模拟与可扩展计算。
%     \item \textbf{骨水泥仿真应用:}面向骨水泥调配与注射的医疗交叉场景,构建从骨结构建模、注射场景参数化、两相骨水泥流动求解到可视化呈现的端到端应用流程。该部分将通用求解器落地到真实场景,服务于教学训练与术前方案评估,并通过与骨组织边界耦合、材料外观与颜色方案等环节提升仿真结果的可解释性与可用性。
% \end{enumerate}

\section{研究创新点}

% 本文的研究工作已在作者的前期工作中展开并发表,其主要创新点归纳如下:

% \begin{enumerate}
%     \item \textbf{构建基于隐式混合模型的两相黏弹性非牛顿求解框架}\\
%     面向“混合过程生成非牛顿流变”的典型场景,本文以隐式混合模型作为组分传输与相间耦合的核心描述,在SPH框架内引入构象张量方法计算黏弹性应力,并将浓度场、流变参数与动量交换统一到一致的更新流程中。该设计避免仅用固定本构参数拟合单一材料外观的局限,使两相混合体系能够在较大黏度范围与显式时间推进下保持稳定,并更自然地呈现随浓度变化的局部流变差异。

%     \item \textbf{提出浓度驱动的相间耦合动态调控机制}\\
%     针对具有显著“成键/聚合”效应的混合体系,相间动量交换若仅由常量控制,难以表达相间作用随反应进程逐步增强的现象。为此,本文设计基于浓度场的动态耦合调控策略,并进一步以键合作用网络(Bonding Effects Network)的形式对微观成键对相间耦合的影响进行建模,使相间动量交换强度能够随局部浓度与时间演化自适应变化,从而更合理地刻画“逐渐凝聚、难以再混合”等混合动力学特征。

%     \item \textbf{在多相框架内实现局部流变自适应的统一非牛顿机制}\\
%     面向剪切变稀、剪切变稠与牛顿流体在同一场景内共存的需求,本文采用统一的变量集合描述不同流变类型,并以相分数(浓度)作为驱动信号在空间上自动计算局部流变,从而能够在非均匀溶液/混合体系中实现“局部流变自适应”。该机制将非牛顿行为从“全局固定参数”转变为“随浓度变化的局部性质”,增强了多相混合场景下对复杂流变的表达能力与参数可移植性。

%     \item \textbf{扩展到任意多相的统一黏弹性求解器}\\
%     在两相机制的基础上,本文进一步构建面向任意相数的统一多相黏弹性求解框架:以混合模型保证质量--动量的一致耦合关系,并采用多模式构象张量(multi-mode conformation tensor)对不同相的黏弹性与剪切相关特性进行统一表示。通过相级别的应力校正与统一组装策略,该框架能够在更大黏度对比与更复杂相互作用条件下保持数值稳定与可扩展计算,为多材料、多组分场景提供统一底座。

%     \item \textbf{引入黏度相关的相间影响函数以调制混合/分离行为}\\
%     针对多相体系中“混合/分离行为与黏度及黏弹性状态强相关”的现象,本文在混合模型框架内引入黏度相关的相间影响函数,用于对相间摩擦与相分数扩散强度进行调制。该设计使相界面演化与混合速率能够随材料流变状态自适应变化,从而在大黏度对比与黏弹性占优的场景中提升模拟的鲁棒性与行为可控性。

%     \item \textbf{形成面向骨水泥手术的端到端仿真应用流程}\\
%     面向骨水泥调配与注射这一高风险、强约束的医疗交叉场景,本文将两相混合与非牛顿流变求解能力落地为端到端仿真流程:覆盖骨结构/孔隙建模、注射场景参数化、骨水泥混合与动态流动求解,以及结果的可视化呈现与对比分析。该流程将“通用求解器能力”与“真实任务需求”对齐,为教学训练与术前方案评估提供可解释、可复现的仿真工具链。
% \end{enumerate}


\section{论文组织结构}

% 本文由以下几个部分组成,各部分的主要内容与逻辑关系如下:

% \textbf{第一章(引言)}阐述了研究背景与意义,阐明了多相黏弹性流体仿真的关键挑战与应用需求,并归纳了本文的主要创新点与论文组织结构。

% \textbf{第二章(文献综述)}总结了国内外在单相牛顿流体、多相牛顿流体、以及多物相耦合仿真方面的研究现状,为后续方法的提出与改进提供理论基础与对标参考。

% \textbf{第三章(基于SPH方法的流体模拟)}介绍了基于Navier-Stokes方程的流体动力学基本模型,包括压力梯度、黏弹性应力、表面张力与体积力等关键力项的表示方式,以及SPH离散化的基本原理、核函数选择与力的离散计算方法。该章为后续具体求解器的构建奠定数学与数值基础。

% \textbf{第四章(基于混合模型的两相非牛顿流体模拟)}基于SPH框架,构建了隐式混合模型下的两相非牛顿求解器。该章阐述了混合模型的基本概念、以构象张量计算黏弹性应力的方法、基于相分数的层流摩擦模型与键合作用网络机制,并通过典型算例验证了求解器在两相高黏混合与局部流变自适应方面的能力。

% \textbf{第五章(基于混合模型的多相多类型流体模拟)}将两相框架扩展到任意多相场景,采用多模式构象张量对不同相的黏弹性与剪切相关特性进行统一表示,引入多相间相漂移影响因子模型以增强数值稳定性。该章通过验证算例展示了求解器在更大黏度对比与更复杂相互作用条件下的鲁棒性与可扩展性。

% \textbf{第六章(基于多相多类型流体模拟的仿真应用)}将通用求解器落地到骨水泥注射的医疗交叉应用场景,介绍了松质骨与皮质骨的建模方法,结合两相非牛顿流体模拟来复现骨水泥调配与注射过程,为教学训练与术前方案评估提供可视化仿真支持。

% \textbf{第七章(仿真框架设计与实现)}阐述了完整仿真系统的总体架构、各模块的实现细节、计算性能优化策略,以及用户交互与可视化呈现的设计考量。

% \textbf{附录与致谢}部分包含本文工作的相关补充材料、参考文献列表、作者简历与研究成果说明。

% \begin{equation}
% \frac{D\rho}{Dt}=-\rho\,\nabla\cdot\mathbf{v},
% \label{eq:ns_continuity}
% \end{equation}
% \begin{equation}
% \rho\frac{D\mathbf{v}}{Dt}=-\nabla p+\nabla\cdot\boldsymbol{\tau}+\mathbf{f}_{\mathrm{st}}+\rho\mathbf{g}+\mathbf{f}_{\mathrm{ext}},
% \label{eq:ns_momentum}
% \end{equation}
