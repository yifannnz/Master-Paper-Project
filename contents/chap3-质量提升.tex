\chapter{基于结构保持的非配对HRTEM图像质量提升方法}



\section{非配对HRTEM图像质量提升(实验图像)}

\section{基于结构保持的改进循环式生成网络}

\subsection{空间频域联合特征提取的生成网络设计}

\subsection{基于自注意力机制的全局原子结构一致性建模}

\subsection{基于特征函数与频域一致性的结构保持约束设计}


\section{算法性能验证与实验结果分析}

\subsection{实验数据(数据集介绍)}

\subsection{实验设置与评价指标}

\subsection{实验结果与分析(定性、定量)}

 
\section{本章小结}



% 本文的总体方法由三个逐步递进的部分构成,研究路线如图\figref{fig:intro_overview}所示:

% % \begin{figure}[htb]
% %     \centering
% %     \includegraphics[width=0.6\textwidth]{images/intro/overview.png}
% %     \bicaption{本文研究路线与总体框架示意图}{Overview of the research roadmap and overall framework}
% %     \label{fig:intro_overview}
% % \end{figure}


% \begin{equation}
% \frac{D\rho}{Dt}=-\rho\,\nabla\cdot\mathbf{v},
% \label{eq:ns_continuity}
% \end{equation}
% \begin{equation}
% \rho\frac{D\mathbf{v}}{Dt}=-\nabla p+\nabla\cdot\boldsymbol{\tau}+\mathbf{f}_{\mathrm{st}}+\rho\mathbf{g}+\mathbf{f}_{\mathrm{ext}},
% \label{eq:ns_momentum}
% \end{equation}
