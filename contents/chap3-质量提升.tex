\chapter{基于结构保持的非配对HRTEM图像质量提升方法}

高质量的HRTEM图像对于在原子尺度上建立材料结构与性能之间的关系至关重要。然而,在实际实验中,为了捕获动态过程的高时间分辨率图像,往往需要在低剂量成像条件下进行采集,这不可避免地导致HRTEM图像质量严重退化,表现为原子排列模糊、结构信息不完整等问题,从而极大地限制了后续的精确结构分析能力。

传统的图像处理方法,如维纳滤波、双边滤波和BM3D等,虽然能够在一定程度上抑制随机噪声,但这些方法依赖于固定的假设和有限的局部统计特性,难以鲁棒地处理复杂背景和严重的图像退化问题。近年来,基于深度学习的方法在TEM图像去噪和增强领域展现出了显著的优势。然而,现有的大多数方法仍然依赖于配对的训练样本或合成数据进行监督学习,忽视了真实高质量实验图像中丰富而真实的纹理特征,且在严重退化的成像条件下表现不佳。

为了解决上述问题,本章提出了一种基于结构保持的非配对HRTEM图像质量提升方法HRTEM-GAN。该方法采用循环一致性生成对抗网络(CycleGAN)框架,在非配对训练条件下运行,通过图像块级别的分布建模实现高质量域和低质量域之间的双向映射。与传统的全图翻译方法不同,本方法采用图像块级建模策略,缩小了生成的有效感受野范围,使网络能够聚焦于细粒度的原子结构特征,从而缓解了全图生成中常见的原子恢复不完整问题。为了更好地建模长程原子关联,本章在生成器的瓶颈层引入了视觉Transformer(ViT)模块,使其能够捕获超越局部卷积邻域的全局依赖关系。此外,本章明确引入了频域建模和约束机制,通过专用的频域分支和频域感知损失函数来利用HRTEM图像固有的频谱特性,在增强图像质量的同时保持原子尺度的结构保真度。

本章的主要内容安排如下:首先介绍非配对HRTEM图像质量提升的问题背景和挑战;随后详细阐述基于结构保持的改进循环式生成网络的设计,包括空间频域联合特征提取的生成网络设计、基于自注意力机制的全局原子结构一致性建模,以及基于特征函数与频域一致性的结构保持约束设计;最后通过在真实实验数据集上的广泛定性和定量评估,验证所提方法的有效性,并与现有代表性方法进行对比分析,证明本方法在图像恢复质量和下游原子柱识别性能上均取得了显著提升。


\section{非配对HRTEM图像质量提升(实验图像)}

\section{基于结构保持的改进循环式生成网络}

\subsection{空间频域联合特征提取的生成网络设计}

\subsection{基于自注意力机制的全局原子结构一致性建模}

\subsection{基于特征函数与频域一致性的结构保持约束设计}


\section{算法性能验证与实验结果分析}

\subsection{实验数据(数据集介绍)}

\subsection{实验设置与评价指标}

\subsection{实验结果与分析(定性、定量)}

 
\section{本章小结}



% 本文的总体方法由三个逐步递进的部分构成,研究路线如图\figref{fig:intro_overview}所示:

% % \begin{figure}[htb]
% %     \centering
% %     \includegraphics[width=0.6\textwidth]{images/intro/overview.png}
% %     \bicaption{本文研究路线与总体框架示意图}{Overview of the research roadmap and overall framework}
% %     \label{fig:intro_overview}
% % \end{figure}


% \begin{equation}
% \frac{D\rho}{Dt}=-\rho\,\nabla\cdot\mathbf{v},
% \label{eq:ns_continuity}
% \end{equation}
% \begin{equation}
% \rho\frac{D\mathbf{v}}{Dt}=-\nabla p+\nabla\cdot\boldsymbol{\tau}+\mathbf{f}_{\mathrm{st}}+\rho\mathbf{g}+\mathbf{f}_{\mathrm{ext}},
% \label{eq:ns_momentum}
% \end{equation}
