%建议使用TeXWorks,选择XeLaTeX运行
%

\documentclass{ustbthesis}

\usepackage{tabularx}
\usepackage{amsmath}
\usepackage{subfig} % 去掉caption相关选项,避免与类冲突
\newcommand{\figref}[1]{~\ref{#1}~}
\newcommand{\tabref}[1]{~\ref{#1}~}
\newcommand{\thickhline}{\noalign{\hrule height 0.9pt}}
\renewcommand{\arraystretch}{1.25}
\newcolumntype{C}[1]{>{\centering\arraybackslash}p{#1}}
\newcolumntype{P}[1]{>{\centering\arraybackslash}p{#1}}

% 统一链接为黑色(避免绿色引用)
\hypersetup{colorlinks=true, linkcolor=black, citecolor=black, urlcolor=black}

%%%%%%%%%%%%%%%%%%%%%%%%%%%封面信息%%%%%%%%%%%%%%%%%%%%%%%%%%%%%%%%%%
\Author{申龙}
\EnglishAuthor{Shen Long}
\Title{原子级纳观影像质量提升与分割算法研究}
\EnglishTitle{Multiphase Viscoelastic Fluid Simulation and Applications Based on SPH Method}
\Subtitle{}
\EnglishSubtitle{}
\Date{2026年 1月 1日}
\EnglishDate{January, 2026}
\College{智能科学与技术学院}
\EnglishCollege{School of Intelligent Science and Technology}
\Address{海淀区学院路30号}
\EnglishAddress{30 Xueyuan Road, Haidian District}
\Affiliation{北京科技大学}
\EnglishAffiliation{University of Science and Technology Beijing}
\Postcode{100083}
\City{北京}
\EnglishCity{Beijing}
\Country{中国}
\EnglishCountry{P.R.CHINA}
\StudentCode{M202320826}
\Major{计算机技术}
\EnglishMajor{Computer Technology}
\Direction{图形学方向}
\EnglishDirection{Computer Graphics}
\Supervisor{XX}
\EnglishSupervisor{XX}
\DegreeType{专业型}
\EnglishDegreeType{Professional}
\CLCNumber{}
\UDCNumber{}
\UniversityCode{10008}
\UniversityLogoFile{images/ustb.png}
\SecurityLevel{公开}
\SubmissionDate{2026年 1月 1日}
\EnglishSubmissionDate{January, 2026}
%%%%%%%%%%%%%%%%%%%%%%%%%%%封面信息%%%%%%%%%%%%%%%%%%%%%%%%%%%%%%%%%%

\begin{document}

\TitlePage
\SecondPage
\ThirdPage

\setlength{\headheight}{14.5pt}
\pagestyle{fancy}

%% 中英文摘要
\cleardoublepage
%% 中文摘要页
\begin{ChineseAbstract}
论文中文摘要字数约为300\textasciitilde{}600字,如遇特殊需要字数可以略多,限一页。

论文摘要是论文内容不加注释和评论的简短陈述,一般以第三人称语气写成。摘要的编写应遵循下列原则:
\begin{enumerate}[label={\arabic*)},itemindent=3.5em,leftmargin=0em] 
\item 摘要应具有独立性和自含性,即不阅读论文的全文,就能获得必要的信息。摘要是学位论文的缩影,是学位论文的主要内容、见解、结论简短明了的缩写。
\item 摘要应是一篇完整的短文,可以独立使用,可以引用。
\item 摘要的内容应包含与论文等同量的主要信息,供读者确定有无必要阅读全文,也可供文摘汇编等二次文献采用。
\item 摘要一般应说明研究工作的目的意义、研究方法、研究结果、主要结论及意义、创造性成果和新见解,而重点是结论和创新点。
\item 要用文字表达,不要附图和照片,除了实在无变通办法可用以外,摘要中不用图、表、化学结构式、非公知公用的符号和术语,不要使用表格、公式、上下标以及其他特殊符号,要突出重点,阐述清楚,少用数据表。论文摘要用语力求简洁、准确。原则上300\textasciitilde{}600字。
\end{enumerate}

\ChineseKeywords{摘要;论文;要求;字数;格式(关键词个数为3\textasciitilde{}5个,正式写作请删除此括号)}
\end{ChineseAbstract}

%% 英文摘要页
\begin{EnglishAbstract}

   In environmental economics, environmental resources including environmental quality are categorized as amenity resources. Due to its importance to human welfare, the amenity resources theoretical study and valuation is an ongoing issue at the academic frontier in the environmental economics area.  
  
\EnglishKeywords{Key word 1; Key word 2; Key word 3; $\cdots$}
注:论文的英文摘要应有英文题目和关键词,内容与中文摘要相同,用另页置于中文摘要之后;外文摘要实词在300个左右。


\end{EnglishAbstract}

%% 序言
\cleardoublepage
%% 序言
\begin{Preclude}
序并非必要。论文的序,一般是作者或他人对本篇基本特征的简介,如说明研究工作缘起、背景、主旨、目的、意义、编写体例,以及资助、支持、协作经过等;也可以评述和对相关问题研究闸述。这些内容也可以在主体部分引言中说明。 
\end{Preclude}

\tableofcontents

\cleardoublepage
%% 插图和附表清单
\chapter*{插图和附表清单(如有)}\addcontentsline{toc}{chapter}{插图和附表清单(如有)}

插图或附表清单并非必要。论文中如图表较多,可以有此页。图的清单应有图号、图题和页码。表的清单应有表号、表题和页码。

根据所列内容,可将本页标题分别更改为“插图清单”、“附表清单”。

\cleardoublepage
%% 插图和附表清单
\chapter*{符号清单(如有)}\addcontentsline{toc}{chapter}{符号清单(如有)}

此页并非必要。符号、标志、缩略词、首字母缩写、计量单位、名词、术语等的注释说明汇集表,如需汇集,可集中置于此页。

根据所列内容,将本页标题分别更改为“符号清单”、“标志清单”、“缩写清单”、“计量单位清单”、“术语表”等。 


\mainmatter
\pagestyle{fancy}

\chapter{引言}

\section{研究背景及意义}
当今时代科技发展日新月异,材料科学作为支撑众多领域发展的核心基础,其战略地位愈发凸显。从航天工程所依赖的耐极端环境、超高强韧的特种金属材料,到微电子行业不可或缺的高迁移率、微型化集成的新型半导体材料,再到生命健康领域所使用的生物兼容、功能适配的植入与诊疗材料,材料本身的性能优劣与制备质量,直接决定了下游产品的整体水平、可靠性以及可应用的场景边界。因此,系统探究材料的本征特性与服役行为,对推动技术创新突破、增强国家核心竞争力,具有至关重要的作用。

材料的微观结构与宏观性能之间关系挖掘是材料科学中的重要话题\cite{ramakrishna2019materials}。研究人员利用多种表征手段从不同尺度开展材料构效关系的研究。比如,在介观尺度下揭示多层纳米结构的界面效应对裂纹传播的影响\cite{butler2018machine};在纳观尺度下表征原子催化活性变化\cite{liu2024site}等。在多种尺度研究中,原子层面的结构表征能精确揭示原子的排列、缺陷等微观信息\cite{banhart2011structural,ziatdinov2017deep,ziletti2018insightful},直接关联材料的结构与性能。这不仅能帮助理解相变、界面效应以及缺陷对材料力学、电学和化学性能的影响,还能进而优化材料设计与应用。因此,要更加深入全面地探索材料的构效关系,提升材料性能,设计符合需求的新材料,原子尺度的观测手段起到了至关重要的作用\cite{tilley2020crystals,muller2009structure}。

\begin{figure} [htb]
    \centering
    \includegraphics[width=0.6\textwidth]{images/chap1/1.png}
    \bicaption{材料微观结构与宏观性能之间关系}{Relationship between microstructure and macroscopic properties of materials}
    \label{fig:1}
\end{figure}

作为揭示材料微观结构和宏观性能之间内在联系的核心手段,现代显微影像表征技术呈现出多样化和高精度发展的趋势。电子显微镜根据成像模式、电子束与样品交互方式和样品厚度以及分辨率的差异分为扫描电子显微镜(SEM)、透射电子显微镜(TEM)和扫描透射电子显微镜(STEM)。这些技术能够识别从毫米到数十皮米的细节,并提供材料在形貌、相态、晶体学以及分子和原子结构等方面的独特信息。随着TEM成像技术及原位表征技术的快速发展,结合像差校正的原位TEM已能够在原子乃至亚埃尺度上直接观测材料的原子排列与结构演化,在揭示材料微观结构、动态行为、物理化学性质及相变机制等方面发挥着不可替代的作用。凭借其超高的空间与能量分辨率,该技术已广泛应用于材料、化学、能源及生命科学等领域,成为研究纳米至原子尺度微观行为及其演化规律的重要手段。

随着计算能力的大幅提升,机器学习和深度学习技术不断突破,AI在不同领域得到了广泛应用,并正在成为科学研究中的核心分析工具。计算机–电子显微术–材料科学的结合应运而生,许多方法也被引入到TEM数据自动化分析任务中。原子定位是晶体或分子结构解析的前置任务,在原子尺度的数据分析中,原子定位的准确性直接决定了后续材料分析的有效性,并为深入研究材料微观结构与性能之间的关系提供技术支撑。

随着图像采集设备的进步,TEM已经能够获得原子级分辨率的图像,提供原子位置、晶格参数和缺陷等信息。然而,在实际成像中,材料的原子结构及缺陷形貌往往呈现出高度复杂的空间分布特征,其边界在噪声、像差及成像模糊等因素的影响下难以清晰界定,给传统的人工分析和规则化处理方法带来了巨大挑战。在材料表征与性能分析过程中,研究人员通常需要对原子列、晶界及缺陷区域进行精确定位与标注,而这一过程高度依赖人工经验与主观判断,容易导致结果不一致,并限制分析精度与重复性。此外,由于高分辨率 TEM 图像及原位实验数据通常具有高维度和大规模特性,人工分析不仅工作量巨大,而且难以满足高通量材料研究的需求。

综上所述,开发一种准确、高效且具有良好泛化能力的原子尺度结构质量提升与分割方法,已成为提升材料显微表征与数据驱动材料研究潜力的重要研究方向。原子分辨率TEM图像中蕴含着丰富的结构信息,通过对原子排列、局域畸变及缺陷形态等特征的定量化分析,能够为理解材料的相变行为、界面效应及性能演化机制提供关键依据。这类基于显微图像的结构特征提取方法,为材料设计、性能优化及新材料发现提供了重要的数据支撑。

目前,得益于卷积神经网络和Transformer架构的突破性进展,深度学习已经成为图像分割的主流方法,但在材料围观结构分析领域,仍存在一些制约性的瓶颈问题亟待解决,具体总结如下:

(1)原子图像质量提升方面。由于设备性能限制、环境干扰及操作条件等因素,TEM原子图像在采集过程中不可避免地会出现退化现象,表现为原子排列模糊、结构信息不完整等问题。传统的视觉增强方法通常因噪声、对比度低和灰度分布不均等影响,对复杂退化图像的适应性较差,难以有效恢复高质量的原子结构信息。深度学习方法虽然在图像增强领域展现出优势,但对大规模标注数据和专业知识高度依赖,在实际应用中面临数据获取困难的挑战。有限数据训练的模型极易出现过拟合现象,在未见过的材料图像上泛化能力较差。此外,现有方法在处理细微对象多、分布广的数据时,长距离依赖关系和全局特征提取能力不足。因此,亟待开发一种精度高、数据依赖性低、泛化性强的图像质量提升方法。

(2)实验TEM图像的精准分割方面。原子定位与分割的精准性直接影响材料微观结构的定量表征和性能分析。当前缺乏公开且完备的实验数据集,现有研究多依赖基于理论模型生成的模拟数据,虽具备高可控性,但难以全面反映真实实验环境下的复杂性。现有分割方法在噪声强烈或对比度低的情况下往往表现不佳,并且通常需要依赖人工设定参数,处理效率低且存在人为误差。CNN模型在细微对象的长距离依赖关系和全局特征捕捉方面能力有限,导致分割精度受限。因此,构建涵盖多种材料体系、多样结构特征的高质量实验数据集,并开发鲁棒的精准分割算法,对于实现低信噪比、高伪影图像的高效处理具有重要意义。

(3)自动化表征分析系统缺失方面。随着高分辨显微影像数据生成速度的提升,传统的TEM图像分析方法在应对复杂样品结构、表征精度及处理效率等方面的局限性日益凸显。目前缺乏集成化的自动化表征分析系统,难以实现从图像获取、质量提升、精准分割到多维度特征提取的全流程智能化处理。开发基于人工智能和机器学习的自动化TEM图像表征分析方法,能够显著提升数据处理效率,增强对复杂样品结构和成分特征的定量分析能力,降低人工成本,提高数据处理的准确性与一致性。

综上所述,针对上述瓶颈问题,本课题旨在构建原子级纳观影像鲁棒分割算法及自动化表征分析系统,围绕图像质量提升、精准分割、自动化表征分析三方面开展深入研究,推动透射电子显微影像分析方法从传统的人工依赖向高通量、智能化、自动化方向转型。这不仅对材料科学研究具有重要的科学价值,还将为深入理解材料的本质机理与性能调控提供更加可靠的依据,对加速材料研发与应用、推动相关领域的技术进步产生深远影响。


\section{研究内容与方法}




\section{论文组织结构}



\chapter{文献综述}

\section{原子级高分辨图像质量提升算法概述}
随着图像采集设备的进步,TEM已经能够获得原子级分辨率的图像,提供原子位置、晶格参数和缺陷等信息。然而,图像质量常受仪器失真、噪声、设备漂移和电子束损伤等影响,导致信噪比低和图像模糊,影响原子结构分析。传统的线性滤波方法能减少高斯噪声,但在保留图像边缘方面效果不佳。为应对复杂噪声和细节保留问题,深度学习算法为图像去噪和超分辨率处理提供了保真性更强的解决方案。

\subsection{基于规则化方法的质量提升}
传统的图像去噪模型分为线性模型和非线性模型两类。在透射电子显微镜(TEM)图像去噪处理中,边缘细节的保留至关重要。然而,线性模型通常难以有效地保持图像的边缘特征,即将边缘作为图像中的不连续性时容易导致边缘模糊或扩散。相比之下,非线性模型在边缘保留方面表现更为优越,能够更好地维护图像的边缘细节结构。

Du\cite{du2015nonlinear}提出了一种非线性滤波算法,用于去除高分辨扫面透射电子显微镜图像中的噪声,能够有效抑制噪声的同时保留图像的细节信息,尤其在低剂量电子束下的噪声环境中表现良好。尽管该方法相比于传统的维纳滤波和高斯滤波方法表现出更高的稳定性和准确性,但由于其计算复杂度高,对噪声模型具有较强的依赖性,在复杂结构或高噪声条件下,可能会在细节保留和去噪平滑度之间产生权衡,限制其在实时和自动化应用中的潜力。

基于块匹配和3D滤波(Block-Matching and 3D Filtering,BM3D)算法\cite{dabov2007image}最初针对高斯噪声设计,通过将图像中的相似区域堆叠在三维域中,利用协作滤波去除噪声。高分辨透射电子显微镜在低剂量成像下,由于入射电子数目有限,电子计数过程的随机统计波动更加明显,从而产生与信号强度相关的泊松噪声,并在弱信号区域表现出更高的噪声占比。随后,Mevenkamp等\cite{mevenkamp2015poisson}改进了BM3D算法,将泊松噪声转化为高斯噪声并结合块匹配策略,提高了低剂量条件下的去噪效率和图像细节保留性。此外,该方法还针对周期性晶体结构提出了非局部块匹配策略,通过利用图像中的自相似性,以较低的计算成本实现有效去噪。

HRTEM filter插件是基于R. Kilaas的滤波方法\cite{kilaas1998optimal}开发的图像处理工具,专门用于HRTEM和扫描透射电子显微镜(Scanning transmission electron microscopy,STEM)数据中的无定形背景去噪。该插件自动应用Wiener滤波和自适应背景减法滤波,有效去除由无定形污染层带来的信号干扰。Wiener滤波在频域中优化信号,还原原始晶体结构信息,以最小均方误差准则实现对真实信号的最佳估计,该功能适用于从噪声和背景干扰中恢复精细的晶体特征;自适应背景减法滤波通过减去频域中平均背景来去除无定形层,适用于晶体与无定形信号强度相近的中频区域,在某些区域提供了更强的背景抑制能力。通过自动调整滤波参数,HRTEM filter能提高图像的信噪比和晶体结构细节,使原子排列和晶体缺陷的成像更清晰,从而提升后续数据分析的精确性和可靠性。
 
\subsection{基于深度学习方法的质量提升}
规则化方法在处理成像风格不同、噪声多样等条件下仍存在局限性,需要大量的参数调整与适配,且效果无法得到保证。近年来,随着人工智能和深度学习技术的发展,深度学习被应用于TEM数据处理,增强了去噪的泛化能力和鲁棒性\cite{lobato2024deep,kazimi2024self,mohan2022deep}。

AtomSegNet是专为TEM数据开发的深度学习模型,旨在实现高精度的原子分割,并提供了标准化数据集TEMImageNet\cite{lin2021temimagenet}。通过构建大规模的模拟数据集,涵盖了不同材料的原子分辨率ADF-STEM数据,引入多种噪声模型,以增强其对实验图像的适应性。AtomSegNet采用基于U-Net\cite{ronneberger2015u}的编码器-解码器结构,能有效去除背景噪声并保留原子级细节。实验表明,AtomSegNet在噪声较大的条件下,能够准确去除背景噪声,并在高噪声环境下仍然保留原子结构的清晰度。然而,模型训练高度依赖合成数据,实验图像中的非线性失真和电子源不稳定等因素可能影响其实际应用效果。

\begin{figure} [htb]
    \centering
    \includegraphics[width=0.8\textwidth]{images/chap2/1.png}
    \bicaption{基于深度学习的质量提升方法}{Deep learning-based quality enhancement methods}
    \label{fig:2-1}
\end{figure}

在AtomSegNet的基础上,Mohan等\cite{mohan2022deep}提出了一个仿真驱动的深度学习去噪框架(Simulation-based denoising,SBD),如图\figref{fig:2-1}(a)。该方法通过仿真生成大规模无噪声图像数据集并加入实验噪声,以解决实验数据缺乏的问题。SBD框架使用U-Net架构,捕捉图像中的周期性结构,显著提高了低信噪比数据的去噪效果。然而,即便是仿真驱动的方法仍然依赖大量的模拟数据,且难以避免与真实实验数据之间存在的差异。为进一步摆脱对任何形式“干净”图像的依赖,Crozier等\cite{crozier2025visualizing}提出了一种完全无监督的深度学习去噪框架,发表在2025年的Science期刊上。在高时间分辨率下,为了避免样品受损,需要使用低电子剂量,但低剂量会导致信噪比降低,遮蔽原子结构细节;连续帧拍摄导致噪声在时序上累积,影响对快速结构演化的观测,而普通图像叠加或滤波方法难以同时保持高分辨率和去噪能力。文章通过使用多帧时序上下文对每一帧进行还原,缓解了帧间运动模糊的问题。该框架的核心结构为盲点(Blind spot)卷积神经网络,即UDVD(Unsupervised deep video denoising)模型,如图\figref{fig:2-1}(b)。该模型结合了多帧信息和U-Net架构,在空间和时间维度上捕捉关键结构特征,在训练过程中排除了目标像素自身的值,仅利用其时空邻域像素信息进行估计,从而有效避免模型过拟合。该方法的提出解决了实验数据稀缺和仿真不准的问题,但该方法依赖于高质量的时间-空间一致性,且结构相对复杂,计算成本较高。因此,为了解决显微图像中间帧不一致与细节过度平滑等问题,He等\cite{he2025efficient}提出了一个支持实时处理的、适用于时序显微图像的零样本降噪与超分辨框架MDSR-Zero,如图\figref{fig:2-1}(c)。该方法完全不依赖干净图像或仿真数据,通过高效的在线训练策略(Efficient online training,EOT)显著加快了模型训练速度并提高了时间一致性。EOT利用前一帧的模型参数作为当前帧的初始化,保留跨帧结构信息,使得模型学习具有时间一致性的结构表征;引入指数滑动平均机制平滑更新模型权重,抑制帧间噪声波动,从而有效解决了帧间结构无法统一建模导致的原子区域边界模糊、细节流失、图像不稳定的问题。此外,作者还设计了一种面向超分辨率的自监督损失函数,在保持低分辨率重建准确性的同时,引导模型关注高频邻域信息,增强了图像细节恢复能力。该方法不仅在合成噪声和真实显微视频数据上表现出优越的去噪与结构保真性能,而且相比现有零样本方法训练效率提高近10倍,推理延迟低,可做到边采集边处理,展示出良好的实用性与扩展性。上述方法分别从数据生成方式、网络结构设计和训练策略上提供了不同的技术路线,适用于不同类型和需求的显微图像分析任务,为真实复杂场景下的后续任务提供了坚实的基础。然而,这些方法在提升特定场景下性能的同时,也暴露出一定的泛化能力局限性,主要体现在难以全面覆盖实验图像中多样化的噪声模式和结构变化,导致在跨设备、跨样本或复杂成像条件下的适应性和鲁棒性不足。

\subsection{基于生成对抗网络的图像质量提升}
生成对抗网络(Generative Adversarial Network,GAN)是Goodfellow\cite{goodfellow2014generative}等人在2014年提出的,通过对抗训练机制进行生成模型学习的框架,其核心目标是在无监督条件下学习真实数据分布,并从随机噪声中生成与真实数据难以区分的样本。在迭代训练过程中,生成器持续优化生成样本的质量以欺骗判别器,判别器则不断提升鉴别真伪的能力,二者形成动态博弈关系直至达到均衡。这一独特的迭代优化过程使生成器的输出分布逐渐逼近真实数据分布,同时推动判别器的判别能力同步增强。

在此基础上,大量工作进一步探索了GAN在图像质量提升领域中的潜力,并将其成功应用于超分辨率重建、图像去噪、去模糊以及低光照图像增强等典型任务中。Ledig\cite{ledig2017photo} 等人提出SRGAN,旨在解决传统超分辨率方法在大倍率放大条件下难以有效恢复高频纹理细节的问题。该研究首次将生成对抗网络引入单图像超分辨率任务,通过引入对抗损失约束,引导生成结果向自然图像流形靠拢,从而提升重建图像的感知质量。SRGAN由一个深度残差生成网络和一个判别网络构成,其中生成器负责从低分辨率图像中重建高分辨率图像,判别器则用于区分生成图像与真实高分辨率图像。与此同时,本文提出了一种感知损失函数,该损失由基于 VGG网络高层特征的内容损失与对抗损失共同组成,用以替代传统基于像素级误差的优化目标。实验结果表明,该方法在视觉质量和主观评价(MOS)指标上显著优于以MSE为优化目标的超分辨率方法,能够生成更加逼真、具有丰富纹理细节的高分辨率图像。Kupyn\cite{kupyn2018deblurgan}等人提出了DeblurGAN,一种基于条件生成对抗网络的端到端单幅图像盲运动去模糊方法,旨在解决传统去模糊算法依赖显式模糊核建模、计算复杂且难以处理复杂真实场景模糊的问题。该研究将运动去模糊任务建模为一种特殊的图像到图像映射问题,直接学习从模糊图像到清晰图像的映射关系。DeblurGAN 采用条件 GAN 结构,其中生成器以模糊图像作为输入,输出对应的清晰图像,判别器则用于判断生成结果与真实清晰图像之间的差异。为提升训练稳定性与感知质量,作者引入了基于Wasserstein GAN的对抗损失,并结合基于VGG特征的感知损失,从而在保持全局结构一致性的同时有效恢复细节纹理。与仅使用像素级损失的去模糊方法相比,DeblurGAN能够生成更加清晰且视觉上更具真实感的结果,该方法在结构相似性和主观视觉质量方面达到或超过当时的先进水平,并且在推理速度上显著优于多尺度卷积网络方法。

CycleGAN等\cite{zhu2017unpaired}无监督图像到图像翻译方法的提出,进一步拓展了GAN在图像质量提升领域的应用范围,使得在缺乏成对训练数据的情况下也能实现高质量的图像转换。该方法通过引入循环一致性损失,确保了从源域到目标域的映射以及反向映射的一致性,从而在无监督条件下实现了有效的图像风格转换和质量提升。CycleGAN由两个生成器和两个判别器组成,其中每个生成器负责一个方向的映射,判别器则用于区分生成图像与真实图像。通过同时优化对抗损失和循环一致性损失,CycleGAN能够在保持内容结构不变的前提下,实现不同风格或质量水平之间的转换。该方法在多个图像翻译任务中表现出色,如照片到艺术画作、白天到夜晚等,并且在没有成对数据的情况下实现了高质量的图像转换,为实际应用中的数据获取问题提供了有效解决方案。

\section{原子级高分辨图像原子定位算法概述}
电镜实验在获取和分析原子级结构时面临速度慢、数据少等问题,且数据往往缺乏代表性。随着新型CCD相机的出现和数据储存速度的提升,电镜实验产生的数据量急剧增加,如何从海量数据中自动化提取细粒度信息成为主要挑战。原子定位是晶体或分子结构解析的前置任务。传统的规则化方法对图像噪声较为敏感,且依赖人工调参或后期修正,难以满足大规模自动化分析需求。深度学习模型通过自动提取图像特征,具有更好的鲁棒性,能够在低对比度和高噪声的条件下输出可用结果。

\subsection{基于规则化方法的原子定位方法}
由于人工及技术水平,通过电镜实验获取和分析原子级结构时存在速度慢、数据少等问题,考虑到宏观尺度下的微观差异,获得的数据往往不具有代表性。如果加上时间尺度以及其他测量信息,由于数据储存速度和储存量的不断提升,电镜实验获得的数据量会变得非常大。因此,如何更好更快地从海量数据中提取有效信息是目前电镜实验的实验处理阶段面临的首要问题,计算机-电子显微术-材料科学的结合应运而生。传统的高分辨透射电子显微镜原子位置识别方法已被广泛应用,并取得显著进展。

Galindo等[64]人早在2007年便在寻找峰值方法(PF)和几何相位方法(GP)的基础上提出了峰值对算法(PP),并对该方法的具体流程进行了阐述。首先用布拉格滤波或维纳滤波对HRTEM图像进行降噪处理,之后通过灰度水平进行峰值定位,搜索具有相对距离和方向的两组峰值对,就可以依据此计算位移场。这个方法不仅可以应用于HRTEM图像,也可以应用于任何类型的晶格图像,如高分辨率Z衬度像。

2016年,Zhang等[63]改进了几何相位分析(GPA)。原先的几何相位分析方法是采用快速傅里叶变换的传统GPA方法,即全局GPA(G-GPA),是基于位移和相位差之间的关系。改进以后的方法是在图像中逐块进行窗口式傅里叶变换,称为子集GPA(S-GPA)。在计算小应变(小于2000με)时,S-GPA的测量精度比G-GPA高出3倍左右。对于大应变(大于15 με万),S-GPA的测量精度比G-GPA高出约50\%。此外,S-GPA方法可以显著消除相填充效应,而G-GPA方法不能。Zhang等人用该方法成功分析了InGaAs/InAlAs超晶格异质结构中的应变场分布。

在之后的研究中,Wen等[62]将子集相位分析(S-GPA)与寻峰方法(PF)和最优逼近算法(OAA)相结合,可以同时准确地确定多个界面层中的所有原子。在确定界面位置的基础上,通过优化选择多个参考区域,同时测量了超晶格结构各层的应变分布。该方法成功地应用于评价分子束外延生长的In0.6Ga0.4As/In0.44Al0.56As 超晶格结构的晶格质量。结果表明:界面晶格几乎完美,层厚均匀,没有缺陷和应力集中。

Zhang等[59]在2018年提出了多椭圆拟合方法。如图5a,该方法首先将灰度图立体化,以色谱柱的形式表现,利用原子柱周围的一系列等距强度水平的等高线,将其拟合为一系列椭圆,即通过椭圆旋转平均得到原子位置信息,在此基础上还可以计算出各原子列的强度分布。此外,原子柱的位置可以通过减去相邻原子柱的强度来细化图像,这种方法可以将原子列的位置精确到皮米范围内。Zhang等人通过测量和模拟SrTiO3的高分辨透射电镜图像,定量的得到了预期精度。

此外,自动化的图像处理软件工具如CalAtom(图5b)和Atomap(图5c)的开发,也推动了自动化分析原子分辨率图像的进程。Magnus Nord[61]研发的Atomap软件,利用基于模型的2D-高斯分布自动量化原子分辨率STEM图像中原子列的位置和形状,在钙钛矿氧化物异质结构的 HAADF和ABF STEM图像上进行了测试,定量分析ABF图像中氧原子柱的位移。Q. Zhang[60]在 MATLAB™中开发了CalAtom的软件工具,用于(扫描)透射电子显微镜图像中的原子柱定量分析。该软件提供了三种高精度确定原子柱位置的算法:矩量法、基于模型的方法和多椭圆拟合方法。基于测量位置,该软件为进一步分析提供了几个选项,如原子柱的平面尺度局部环境,局部初等组成和图像图案的实空间平均。以MoS2(1−x)Se2x单分子层的HAADF-STEM实验图像为例,首先利用MEF方法确定原子柱的位置。原子柱的伪彩色显示直观地说明了原子柱的形状和中心;通过计算每个原子列的平均强度,可以得到一个局部元素图。在映射计算中,利用原子的相对位置(原子列)来区分Mo和X2位点,特别是采用均值位移聚类法的无监督分类算法,在X2位点上实现了适当的元素分类,揭示了Se原子主要沿MoS2单分子层的晶界掺杂。这些工具减少了手动干预的需求,使得研究者可以更高效地处理大量原子级图像数据,从而大大提高了分析效率和准确性。



 
\subsection{基于深度学习方法的原子定位方法}
基于深度学习的方法能够进一步提升原子定位的准确率,提升大规模图像数据处理的效率和一致性。在此基础上,能够识别复杂材料中的缺陷和局部结构变化,并将图像分析结果与材料的物理和化学特性关联起来。表1按照不同训练策略对原子定位的深度学习方法进行分类。分析发现,大多工作基于监督学习展开,在模拟数据上训练、实验数据上测试。

\begin{figure} [H]
    \centering
    \includegraphics[width=0.8\textwidth]{images/chap2/unet.png}
    \bicaption{U-Net网络模型架构示意图}{U-Net network model architecture diagram}
    \label{fig:unet}
\end{figure}

\section{性能评价指标概述}


\section{本章小结}
\chapter{基于结构保持的非配对HRTEM图像质量提升方法}



\section{非配对HRTEM图像质量提升(实验图像)}

\section{基于结构保持的改进循环式生成网络}

\subsection{空间频域联合特征提取的生成网络设计}

\subsection{基于自注意力机制的全局原子结构一致性建模}

\subsection{基于特征函数与频域一致性的结构保持约束设计}


\section{算法性能验证与实验结果分析}

\subsection{实验数据(数据集介绍)}

\subsection{实验设置与评价指标}

\subsection{实验结果与分析(定性、定量)}

 
\section{本章小结}



% 本文的总体方法由三个逐步递进的部分构成,研究路线如图\figref{fig:intro_overview}所示:

% % \begin{figure}[htb]
% %     \centering
% %     \includegraphics[width=0.6\textwidth]{images/intro/overview.png}
% %     \bicaption{本文研究路线与总体框架示意图}{Overview of the research roadmap and overall framework}
% %     \label{fig:intro_overview}
% % \end{figure}


% \begin{equation}
% \frac{D\rho}{Dt}=-\rho\,\nabla\cdot\mathbf{v},
% \label{eq:ns_continuity}
% \end{equation}
% \begin{equation}
% \rho\frac{D\mathbf{v}}{Dt}=-\nabla p+\nabla\cdot\boldsymbol{\tau}+\mathbf{f}_{\mathrm{st}}+\rho\mathbf{g}+\mathbf{f}_{\mathrm{ext}},
% \label{eq:ns_momentum}
% \end{equation}

\chapter{文献综述}

\section{图像质量提升算法概述}

\subsection{传统方法}

\section{论文组织结构}



\chapter{高分辨TEM图像自动表征系统}

\section{需求分析与系统功能设计}

\subsection{自动化原子位置识别与处理模块}


\subsection{原子细粒度信息动态提取模块}

\section{总体架构设计}


\section{详细设计与实现}

\subsection{系统技术选型}


\subsection{自动化原子位置识别与处理模块实现}
\textbf{多原子定位:}

\textbf{多原子关系建模:}

\textbf{关键参量计算:}



\subsection{原子细粒度信息动态提取模块实现}
\textbf{多原子运动追踪:}

\textbf{多原子行为识别:}

\textbf{运动特征提取:}

\section{本章小结}

% \input{contents/chap6-BIBM}
% \input{contents/chap7-Summary}

%% 参考文献(thebibliography:手工录入)
\cleardoublepage
% contents/bib.tex —— thebibliography(手工录入参考文献)

\phantomsection
\chapter*{参考文献}
\addcontentsline{toc}{chapter}{参考文献}

\begin{thebibliography}{99}
	% 参考文献采用 GB/T 7714—2015 著录格式(手工录入)。
	% 正文使用 \cite{key} 进行引用。

	% 为便于全量维护与去重,本工程将 bibitem 列表抽取到单独文件。
	
	\bibitem{wang2024physics}
	Wang X, Xu Y, Liu S, et al. Physics-based fluid simulation in computer graphics: Survey, research trends, and challenges[J]. Computational Visual Media, 2024: 1-56.

	\bibitem{gingold1977smoothed}
	Gingold R A, Monaghan J J. Smoothed particle hydrodynamics: theory and application to non-spherical stars[J]. Monthly Notices of the Royal Astronomical Society, 1977, 181(3): 375-389.

	\bibitem{monaghan2005smoothed}
	Monaghan J J. Smoothed particle hydrodynamics[J]. Reports on Progress in Physics, 2005, 68(8): 1703.

	\bibitem{koschier2022survey}
	Koschier D, Bender J, Solenthaler B, et al. A survey on SPH methods in computer graphics[C]//Computer graphics forum: 2022: 737-760.

	\bibitem{ma2024hybrid}
	Ma Y, Xiao X, Li W, et al. Hybrid LBM-FVM solver for two-phase flow simulation[J]. Journal of Computational Physics, 2024, 506: 112920.

	\bibitem{liu2018adaptive}
	Liu C, Hu C. An adaptive multi-moment FVM approach for incompressible flows[J]. Journal of Computational Physics, 2018, 359: 239-262.

	\bibitem{becker2007weakly}
	Becker M, Teschner M. Weakly compressible SPH for free surface flows[C]//Proceedings of the 2007 ACM SIGGRAPH/Eurographics symposium on Computer animation: 2007: 209-217.

	\bibitem{solenthaler2009predictive}
	Solenthaler B, Pajarola R. Predictive-corrective incompressible SPH[C]//ACM SIGGRAPH 2009 papers: 2009: 1-6.

	\bibitem{bender2015divergence}
	Bender J, Koschier D. Divergence-free smoothed particle hydrodynamics[C]//Proceedings of the 14th ACM SIGGRAPH/Eurographics symposium on computer animation: 2015: 147-155.

	\bibitem{takahashi2023multilevel}
	Takahashi T, Batty C. A multilevel active-set preconditioner for box-constrained pressure poisson solvers[J]. Proceedings of the ACM on Computer Graphics and Interactive Techniques, 2023, 6(3): 1-22.

	\bibitem{hirasawa2021flux}
	Hirasawa N, Kanai T, Ando R. A flux-interpolated advection scheme for fluid simulation[J]. The Visual Computer, 2021, 37(9-11): 2607-2618.

	\bibitem{akinci2013versatile}
	Akinci N, Akinci G, Teschner M. Versatile surface tension and adhesion for SPH fluids[J]. ACM Transactions on Graphics (TOG), 2013, 32(6): 1-8.

	\bibitem{jeske2023implicit}
	Jeske SR, Westhofen L, Loschner F, et al. Implicit Surface Tension for SPH Fluid Simulation[J]. ACM Transactions on Graphics, 2023, 43(1): 1-14.

	\bibitem{feng2022impulse}
	Feng F, Liu J, Xiong S, et al. Impulse fluid simulation[J]. IEEE Transactions on Visualization and Computer Graphics, 2022.

	\bibitem{schechter2012ghost}
	Schechter H, Bridson R. Ghost SPH for animating water[J]. ACM Transactions on Graphics (TOG), 2012, 31(4): 1-8.

	\bibitem{brookshaw1985method}
	Brookshaw L. A method of calculating radiative heat diffusion in particle simulations[J]. Publications of the Astronomical Society of Australia, 1985, 6(2): 207-210.

	\bibitem{takahashi2015implicit}
	Takahashi T, Dobashi Y, Fujishiro I, et al. Implicit formulation for SPH-based viscous fluids[C]//Computer Graphics Forum: 2015: 493-502.

	\bibitem{peer2015implicit}
	Peer A, Ihmsen M, Cornelis J, et al. An implicit viscosity formulation for SPH fluids[J]. ACM Transactions on Graphics (TOG), 2015, 34(4): 1-10.

	\bibitem{weiler2018physically}
	Weiler M, Koschier D, Brand M, et al. A physically consistent implicit viscosity solver for SPH fluids[C]//Computer Graphics Forum: 2018: 145-155.

	\bibitem{goldade2019adaptive}
	Goldade R, Wang Y, Aanjaneya M, et al. An adaptive variational finite difference framework for efficient symmetric octree viscosity[J]. ACM Transactions on Graphics (TOG), 2019, 38(4): 1-14.

	\bibitem{shao2022fast}
	Shao H, Huang L, Michels DL. A fast unsmoothed aggregation algebraic multigrid framework for the large-scale simulation of incompressible flow[J]. ACM Transactions on Graphics (TOG), 2022, 41(4): 1-18.

	\bibitem{bender2016divergence}
	Bender J, Koschier D. Divergence-free SPH for incompressible and viscous fluids[J]. IEEE Transactions on Visualization and Computer Graphics, 2016, 23(3): 1193-1206.

	\bibitem{liu2021adapted}
	Liu S, He X, Wang W, et al. Adapted SIMPLE Algorithm for Incompressible SPH Fluids With a Broad Range Viscosity[J]. IEEE Transactions on Visualization and Computer Graphics, 2021, 28(9): 3168-3179.

	\bibitem{panuelos2023polystokes}
	Panuelos J, Goldade R, Grinspun E, et al. PolyStokes: A polynomial model reduction method for viscous fluid simulation[J]. ACM Transactions on Graphics (TOG), 2023, 42(4): 1-13.

	\bibitem{green2010particle}
	Green S. Particle simulation using cuda[J]. NVIDIA whitepaper, 2010, 6: 121-128.

	\bibitem{bridson2007fast}
	Bridson R. Fast Poisson disk sampling in arbitrary dimensions.[J]. SIGGRAPH sketches, 2007, 10(1): 1.

	\bibitem{cross65}
	Cross M M. Rheology of non-Newtonian fluids: a new flow equation for pseudoplastic systems[J]. Journal of Colloid Science, 1965.

	\bibitem{carreau1972rheological}
	Carreau P J. Rheological equations from molecular network theories[J]. Transactions of the Society of Rheology, 1972, 16(1): 99-127.

	\bibitem{andrade2014sph}
	Andrade LFDS, Sandim M, Petronetto F, et al. SPH fluids for viscous jet buckling[C]//2014 27th SIBGRAPI Conference on Graphics, Patterns and Images: 2014: 65-72.

	\bibitem{ozgen2019sph}
	Ozgen O, Kallmann M, Brown E. An SPH model to simulate the dynamic behavior of shear thickening fluids[J]. Computer Animation and Virtual Worlds, 2019, 30(5): e1870.

	\bibitem{zhu2015codimensional}
	Zhu B, Lee M, Quigley E, et al. Codimensional non-Newtonian fluids[J]. ACM Transactions on Graphics (TOG), 2015, 34(4): 1-9.

	\bibitem{su2021unified}
	Su H, Xue T, Han C, et al. A unified second-order accurate in time MPM formulation for simulating viscoelastic liquids with phase change[J]. ACM Transactions on Graphics (TOG), 2021, 40(4): 1-18.

	\bibitem{li2023unified}
	Li C, Gao Y, He J, et al. A Unified Particle-Based Solver for Non-Newtonian Behaviors Simulation[J]. IEEE Transactions on Visualization and Computer Graphics, 2023.

	\bibitem{oldroyd1950formulation}
	Oldroyd JG. On the formulation of rheological equations of state[J]. Proceedings of the Royal Society of London. Series A. Mathematical and Physical Sciences, 1950, 200(1063): 523-541.

	\bibitem{balci2011symmetric}
	Balci N, Thomases B, Renardy M, et al. Symmetric factorization of the conformation tensor in viscoelastic fluid models[J]. Journal of Non-Newtonian Fluid Mechanics, 2011, 166(11): 546-553.

	\bibitem{bejan2005constructal}
	Bejan A. The constructal law of organization in nature: tree-shaped flows and body size[J]. Journal of Experimental Biology, 2005, 208(9): 1677-1686.

	\bibitem{fattal2005time}
	Fattal R, Kupferman R. Time-dependent simulation of viscoelastic flows at high Weissenberg number using the log-conformation representation[J]. Journal of Non-Newtonian Fluid Mechanics, 2005, 126(1): 23-37.

	\bibitem{coronado2007simple}
	Coronado OM, Arora D, Behr M, et al. A simple method for simulating general viscoelastic fluid flows with an alternate log-conformation formulation[J]. Journal of Non-Newtonian Fluid Mechanics, 2007, 147(3): 189-199.

	\bibitem{afonso2009log}
	Afonso A, Oliveira PJ, Pinho F, et al. The log-conformation tensor approach in the finite-volume method framework[J]. Journal of Non-Newtonian Fluid Mechanics, 2009, 157(1-2): 55-65.

	\bibitem{moreno2019logarithmic}
	Moreno L, Codina R, Baiges J, et al. Logarithmic conformation reformulation in viscoelastic flow problems approximated by a VMS-type stabilized finite element formulation[J]. Computer Methods in Applied Mechanics and Engineering, 2019, 354: 706-731.

	\bibitem{barreiro2017conformation}
	Barreiro H, Garca-Fernandez I, Alduan I, et al. Conformation constraints for efficient viscoelastic fluid simulation[J]. ACM Transactions on Graphics (TOG), 2017, 36(6): 1-11.

	\bibitem{alduan2017dyverso}
	Alduan I, Tena A, Otaduy MA. Dyverso: A versatile multi-phase position-based fluids solution for VFX[C]//Computer Graphics Forum: 2017: 32-44.

	\bibitem{solenthaler2008density}
	Solenthaler B, Pajarola R. Density contrast SPH interfaces[J]. 2008.

	\bibitem{zhang2008material}
	Zhang DZ, Zou Q, VanderHeyden WB, et al. Material point method applied to multiphase flows[J]. Journal of Computational Physics, 2008, 227(6): 3159-3173.

	\bibitem{yan2018mpm}
	Yan X, Li CF, Chen XS, et al. MPM simulation of interacting fluids and solids[C]//Computer Graphics Forum: 2018: 183-193.

	\bibitem{da2014multimaterial}
	Da F, Batty C, Grinspun E. Multimaterial mesh-based surface tracking.[J]. ACM Trans. Graph., 2014, 33(4): 112-1.

	\bibitem{misztal2013multiphase}
	Misztal MK, Erleben K, Bargteil A, et al. Multiphase flow of immiscible fluids on unstructured moving meshes[J]. IEEE transactions on visualization and computer graphics, 2013, 20(1): 4-16.

	\bibitem{li2015multiphase}
	Li X, He X, Liu X, et al. Multiphase interface tracking with fast semi-Lagrangian contouring[J]. IEEE Transactions on Visualization and Computer Graphics, 2015, 22(8): 1973-1986.

	\bibitem{yang2018semi}
	Yang M, Ye J, Ding F, et al. A semi-explicit surface tracking mechanism for multi-phase immiscible liquids[J]. IEEE Transactions on Visualization and Computer Graphics, 2018, 25(10): 2873-2885.

	\bibitem{Mikko96}
	Manninen M, Taivassalo V, Kallio S. On the mixture model for multiphase flow[R]. VTT Publications, 1996(288): 3-67.

	\bibitem{ren2014multiple}
	Ren B, Li C, Yan X, et al. Multiple-fluid SPH simulation using a mixture model[J]. ACM Transactions on Graphics (TOG), 2014, 33(5): 1-11.

	\bibitem{yan2016multiphase}
	Yan X, Jiang YT, Li CF, et al. Multiphase SPH simulation for interactive fluids and solids[J]. ACM Transactions on Graphics (TOG), 2016, 35(4): 1-11.

	\bibitem{ren2021unified}
	Ren B, Xu B, Li C. Unified particle system for multiple-fluid flow and porous material[J]. ACM Transactions on Graphics (TOG), 2021, 40(4): 1-14.

	\bibitem{jiang2020divergence}
	Jiang Y, Li C, Deng S, et al. A Divergence-free Mixture Model for Multiphase Fluids[C]//Computer Graphics Forum: 2020: 69-77.

	\bibitem{jiang2021dynamic}
	Jiang Y, Lan Y. A Dynamic Mixture Model for Non-equilibrium Multiphase Fluids[C]//Computer Graphics Forum: 2021: 85-95.

	\bibitem{xu2023implicitly}
	Xu Y, Wang X, Wang J, et al. An Implicitly Stable Mixture Model for Dynamic Multi-fluid Simulations[C]//SIGGRAPH Asia 2023 Conference Papers: 2023: 1-11.

	\bibitem{hong2003animation}
	Hong JM, Kim CH. Animation of bubbles in liquid[C]//Computer Graphics Forum: 2003: 253-262.

	\bibitem{ihmsen2011animation}
	Ihmsen M, Bader J, Akinci G, et al. Animation of air bubbles with SPH[C]//International Conference on Computer Graphics Theory and Applications: 2011: 225-234.

	\bibitem{takahashi2003realistic}
	Takahashi T, Fujii H, Kunimatsu A, et al. Realistic animation of fluid with splash and foam[C]//Computer Graphics Forum: 2003: 391-400.

	\bibitem{muller2004point}
	Muller M, Keiser R, Nealen A, et al. Point based animation of elastic, plastic and melting objects[C]//Proceedings of the 2004 ACM SIGGRAPH/Eurographics symposium on Computer animation: 2004: 141-151.

	\bibitem{chang2009particle}
	Chang Y, Bao K, Liu Y, et al. A particle-based method for viscoelastic fluids animation[C]//Proceedings of the 16th ACM symposium on virtual reality software and technology: 2009: 111-117.

	\bibitem{fujisawa2007animation}
	Fujisawa M, Miura KT. Animation of ice melting phenomenon based on thermodynamics with thermal radiation[C]//Proceedings of the 5th international conference on Computer graphics and interactive techniques in Australia and Southeast Asia: 2007: 249-256.

	\bibitem{kim2003visual}
	Kim T, Lin MC. Visual simulation of ice crystal growth[C]//Proceedings of the 2003 ACM SIGGRAPH/Eurographics symposium on Computer animation: 2003: 86-97.

	\bibitem{kim2004hybrid}
	Kim T, Henson M, Lin MC. A hybrid algorithm for modeling ice formation[C]//Proceedings of the 2004 ACM SIGGRAPH/Eurographics symposium on Computer animation: 2004: 305-314.

	\bibitem{kim2006modeling}
	Kim T, Adalsteinsson D, Lin MC. Modeling ice dynamics as a thin-film stefan problem[C]//Symposium on Computer Animation: Proceedings of the 2006 ACM SIGGRAPH/Eurographics symposium on Computer animation: Vienna, Austria: 2006: 167-176.

	\bibitem{mihalef2009simulation}
	Mihalef V, Metaxas D, Sussman M. Simulation of two-phase flow with sub-scale droplet and bubble effects[C]//Computer Graphics Forum: 2009: 229-238.

	\bibitem{ren2015simple}
	Ren B, Jiang Y, Li C, et al. A simple approach for bubble modelling from multiphase fluid simulation[J]. Computational Visual Media, 2015, 1: 171-181.

	\bibitem{stomakhin2014augmented}
	Stomakhin A, Schroeder C, Jiang C, et al. Augmented MPM for phase-change and varied materials[J]. ACM Transactions on Graphics (TOG), 2014, 33(4): 1-11.

	\bibitem{tu2024unified}
	Tu Z, Li C, Zhao Z, et al. A Unified MPM Framework Supporting Phase-field Models and Elastic-viscoplastic Phase Transition[J]. ACM Transactions on Graphics, 2024, 43(2): 1-19.

	\bibitem{yang2015fast}
	Yang T, Chang J, Ren B, et al. Fast multiple-fluid simulation using Helmholtz free energy[J]. ACM Transactions on Graphics (TOG), 2015, 34(6): 1-11.

	\bibitem{yan2023high}
	Yan H, Ren B. High Density Ratio Multi-Fluid Simulation with Peridynamics[J]. ACM Transactions on Graphics (TOG), 2023, 42(6): 1-14.

	\bibitem{li2024kinetic}
	Li W, Wu K, Desbrun M. Kinetic simulation of turbulent multifluid flows[J]. ACM Transactions on Graphics (TOG), 2024, 43(4): 1-17.

	\bibitem{li2022efficient}
	Li W, Ma Y, Liu X, et al. Efficient kinetic simulation of two-phase flows[J]. ACM Transactions on Graphics, 2022, 41(4): 114.

	\bibitem{ishii1990two}
	Ishii M. Two-fluid model for two-phase flow[J]. Multiphase science and technology, 1990, 5(1-4).

	\bibitem{larionov2017variational}
	Larionov E, Batty C, Bridson R. Variational stokes: A unified pressure-viscosity solver for accurate viscous liquids[J]. ACM Transactions on Graphics (TOG), 2017, 36(4): 1-11.

	\bibitem{xing2022position}
	Xing J, Ruan L, Wang B, et al. Position-based surface tension flow[J]. ACM Transactions on Graphics (TOG), 2022, 41(6): 1-12.

	\bibitem{macklin2013position}
	Macklin M, Muller M. Position based fluids[J]. ACM Transactions on Graphics (TOG), 2013, 32(4): 1-12.

	\bibitem{macklin2016xpbd}
	Macklin M, Muller M, Chentanez N. XPBD: position-based simulation of compliant constrained dynamics[C]//Proceedings of the 9th International Conference on Motion in Games: 2016: 49-54.

	\bibitem{hochstetter2017evaporation}
	Hochstetter H, Kolb A. Evaporation and condensation of SPH-based fluids[C]//Proceedings of the ACM SIGGRAPH/Eurographics Symposium on Computer Animation: 2017: 1-9.

	\bibitem{fei2019multi}
	Fei Y, Batty C, Grinspun E, et al. A multi-scale model for coupling strands with shear-dependent liquid[J]. ACM Transactions on Graphics (TOG), 2019, 38(6): 1-20.

	\bibitem{li2023fluid}
	Li W, Desbrun M. Fluid-Solid Coupling in Kinetic Two-Phase Flow Simulation[J]. ACM Transactions on Graphics, 2023, 42(4): 1-14.

	\bibitem{bird1987dynamics}
	Bird RB, Armstrong RC, Hassager O. Dynamics of polymeric liquids. Vol. 1: Fluid mechanics[J]. 1987.

	\bibitem{john2024comparison}
	John T, Poole R, Kowalski A, et al. A comparison between the FENE-P and sPTT constitutive models in large-amplitude oscillatory shear[J]. Journal of Fluid Mechanics, 2024, 979: A10.

	\bibitem{renardy2021mathematician}
	Renardy M, Thomases B. A mathematician’s perspective on the Oldroyd B model: progress and future challenges[J]. Journal of Non-Newtonian Fluid Mechanics, 2021, 293: 104573.

	\bibitem{zhang2024multiphase}
	Zhang Y, Long S, Xu Y, et al. Multiphase Viscoelastic Non-Newtonian Fluid Simulation[C]//Computer Graphics Forum: 2024: e15180.

	\bibitem{wyart2014discontinuous}
	Wyart M, Cates ME. Discontinuous shear thickening without inertia in dense non-Brownian suspensions[J]. Physical review letters, 2014, 112(9): 098302.

	\bibitem{royer2016rheological}
	Royer JR, Blair DL, Hudson SD. Rheological signature of frictional interactions in shear thickening suspensions[J]. Physical review letters, 2016, 116(18): 188301.

	\bibitem{hermes2016unsteady}
	Hermes M, Guy BM, Poon WC, et al. Unsteady flow and particle migration in dense, non-Brownian suspensions[J]. Journal of Rheology, 2016, 60(5): 905-916.

	\bibitem{ShenUnifiedMCT}
	Shen L, Zhang Y, Frey S, et al. A unified viscoelastic solver for multiphase fluid simulation based on a mixture model[Z]. Unpublished manuscript.

	\bibitem{vaishya2013bone}
	Vaishya R, Chauhan M, Vaish A. Bone cement[J]. Journal of clinical orthopaedics and trauma, 2013, 4(4): 157-163.

	\bibitem{saha1984mechanical}
	Saha S, Pal S. Mechanical properties of bone cement: a review[J]. Journal of biomedical materials research, 1984, 18(4): 435-462.

	\bibitem{charnley1960anchorage}
	Charnley J. Anchorage of the femoral head prosthesis to the shaft of the femur[J]. The Journal of Bone \& Joint Surgery British Volume, 1960, 42(1): 28-30.

	\bibitem{truumees2004percutaneous}
	Truumees E, Hilibrand A, Vaccaro AR. Percutaneous vertebral augmentation[J]. The Spine Journal, 2004, 4(2): 218-229.

	\bibitem{ebeling2019efficacy}
	Ebeling PR, Akesson K, Bauer DC, et al. The efficacy and safety of vertebral augmentation: a second ASBMR task force report[J]. Journal of Bone and Mineral Research, 2019, 34(1): 3-21.

	\bibitem{breusch2005pulmonary}
	Breusch S, Malchau H, Clarius M, et al. Pulmonary embolism in cemented total hip arthroplasty[J]. The Well-Cemented Total Hip Arthroplasty: Theory and Practice, 2005, 320-331.

	\bibitem{moussazadeh2015short}
	Moussazadeh N, Rubin DG, McLaughlin L, et al. Short-segment percutaneous pedicle screw fixation with cement augmentation for tumor-induced spinal instability[J]. The Spine Journal, 2015, 15(7): 1609-1617.

	\bibitem{wilson1971glass}
	Wilson AD, Kent B. The glass-ionomer cement, a new translucent dental filling material[J]. Journal of Applied Chemistry and Biotechnology, 1971, 21(11): 313-313.

	\bibitem{nicholson1993study}
	Nicholson J, Hawkins S, Wasson E. A study of the structure of zinc polycarboxylate dental cements[J]. Journal of Materials Science: Materials in Medicine, 1993, 4: 32-35.

	\bibitem{sidhu2016review}
	Sidhu SK, Nicholson JW. A review of glass-ionomer cements for clinical dentistry[J]. Journal of functional biomaterials, 2016, 7(3): 16.

	\bibitem{ambard2006calcium}
	Ambard AJ, Mueninghoff L. Calcium phosphate cement: review of mechanical and biological properties[J]. Journal of Prosthodontics, 2006, 15(5): 321-328.

	\bibitem{newman2006survey}
	Newman TS, Yi H. A survey of the marching cubes algorithm[J]. Computers \& Graphics, 2006, 30(5): 854-879.

	\bibitem{no2014nanomaterials}
	No YJ, Roohani-Esfahani SI, Zreiqat H. Nanomaterials: the next step in injectable bone cements[J]. Nanomedicine, 2014, 9(11): 1745-1764.

	\bibitem{samad2011new}
	Samad HA, Jaafar M, Othman R, et al. New bioactive glass-ceramic: synthesis and application in PMMA bone cement composites[J]. Bio-medical materials and engineering, 2011, 21(4): 247-258.

	\bibitem{jeong2019bioactive}
	Jeong J, Kim JH, Shim JH, et al. Bioactive calcium phosphate materials and applications in bone regeneration[J]. Biomaterials research, 2019, 23(1): 4.

	\bibitem{tan2021biomaterial}
	Tan B, Tang Q, Zhong Y, et al. Biomaterial-based strategies for maxillofacial tumour therapy and bone defect regeneration[J]. International journal of oral science, 2021, 13(1): 9.

	\bibitem{li2024integrated}
	Li J, Yang Y, Sun Z, et al. Integrated evaluation of biomechanical and biological properties of the biomimetic structural bone scaffold: Biomechanics, simulation analysis, and osteogenesis[J]. Materials Today Bio, 2024, 24: 100934.

	\bibitem{shen2024visual}
	Shen L, Zhang Y, Wang X, et al. Visual simulation of bone cement blending and dynamic flow[C]//2024 IEEE International Conference on Bioinformatics and Biomedicine (BIBM): 2024: to appear.

	\bibitem{zhangBoneCementCMPB}
	Zhang Y, Shen L, Wang X, Frey S, Telea AC, Kosinka J. Bone Cement Flow Simulation: A Two-Phase Non-Newtonian Fluid Approach for Surgical Optimization[J]. Computer Methods and Programs in Biomedicine, under review.


\end{thebibliography}


%% 附录
\cleardoublepage
\appendix

\chapter{单位}

以下内容可放在附录之内:
\begin{enumerate}[label={(\arabic*)},itemindent=2em,align=left,labelsep=0em]
%\begin{enumerate}[label={(\arabic*)}, align=left, leftmargin=2.5em, labelwidth=2em, labelsep=0.5em]
%		\begin{enumerate}[label={(\arabic*)}, align=left, itemindent=3.5em, leftmargin=0em]
\item 正文内过于冗长的公式推导;
\item 方便他人阅读所需的辅助性数学工具或表格;
\item 重复性数据和图表;
\item 论文使用的主要符号的意义和单位;
\item 程序说明和程序全文;
\item 企业应用证明;
\item 项目鉴定报告;
\item 获奖成果证书;
\item 设计图纸;
\item 程序源代码;
\item 论文发表;
\item 作者简介。
\end{enumerate}

这部分内容可省略。如果省略,删掉此页。

书写格式说明:

标题“附录A 附录内容名称”样式为字体:黑体,英文用Times New Roman字体,居中,加粗,字号:小三,2.41倍行距,段前17磅,段后为16.5磅。

附录正文样式为字体宋体小四,英文用Times New Roman字体小四,两端对齐书写,段落首行左缩进2个字符。1.3倍行距(段落中有数学表达式时,可根据表达需要设置该段的行距),段前0.1行,段后0.1行,1.3倍行距。


示例


\begin{table}
\centering
\caption{表 A.1 国际单位制的基本单位}
\noindent\renewcommand{\arraystretch}{0.9}
\begin{tabular}{C{0.33\textwidth}C{0.34\textwidth}C{0.33\textwidth}}
\toprule
量的名称 & 单位名称 & 单位符号 \\
\midrule
长度 & 米 & m \\
质量 & 千克(公斤) & kg \\
时间 & 秒 & s \\
电流 & 安〔培〕 & A \\
热力学温度 & 开〔尔文〕 & K \\
发光强度 & 坎〔德拉〕 & cd \\
\bottomrule
\end{tabular}
\end{table}
\vspace{-1em}  % 向上收紧一行
\begin{table}
\centering
\caption{表 A.2 国家规定的非国际单位制单位}
\noindent\renewcommand{\arraystretch}{0.9}
\begin{tabular}{C{0.14\textwidth}C{0.2\textwidth}C{0.15\textwidth}C{0.51\textwidth}}
\toprule
量的名称 & 单位名称 & 单位符号 & 换算关系和说明 \\
\midrule
\multirow{3}{*}{时间}
 & 分 & min & 1\,min=60\,s \\
 & {\small [小]时} & h & 1\,h=60\,min=3600\,s \\
 & 天(日) & d & 1\,d=24\,h=86400\,s \\
\multirow{3}{*}{平面角}
& {\small [角]秒} & $''$ & $1'' = (\pi/648000)\,\mathrm{rad}$ \\
 & {\small [角]分} & $'$ & $1' = 1^\circ/60 = (\pi/10800)\,\mathrm{rad}$ \\
 & 度 & $^\circ$ & $1^\circ = (\pi/180)\,\mathrm{rad}$ \\
 & 度 & $^\circ$ & 1$^\circ=(\pi/180)\,rad $\\
旋转速度 & 转每分 & r/min & 1\,r/min=(1/60)\,r/s \\
长度 & 海里 & n\,mile & 1\,n\,mile=1852\,m \\ 
速度 & 节 & kn & 1\,kn=1\,n\,mile/h=(1852/3600)\,m/s\\
 &  &  & (只用于航行) \\
\bottomrule
\end{tabular}
\end{table}
%% 致谢
\cleardoublepage
%% 致谢
%\pagestyle{empty}
\chapter*{致\quad\quad 谢}\addcontentsline{toc}{chapter}{致谢}

致谢是作者对该论文的形成作出过贡献的组织或个人予以申明的文字记载,语言要客观、准确、简短。

字数一般不超过500字,最多不超过一页。论文作者可以在致谢页对下列方面致谢:

国家科学基金、资助研究工作的奖学金基金,合同单位、资助或支持的企业、组织或个人;

协助完成研究工作和提供便利条件的组织或个人;

在研究工作中提出建议和提供帮助的人;

给予转载和引用权的资料、图片、文献、研究思想和设想的所有者;

其它应感谢的组织或个人。

%% 作者简历及在学研究成果
\cleardoublepage
%% 作者简历及在学研究成果
\chapter*{作者简历及在学研究成果}\addcontentsline{toc}{chapter}{作者简历及在学研究成果}

\section*{一、主要教育经历/工作经历(从大学起,到博士入学止)}


% ...
\noindent\renewcommand{\arraystretch}{0.9}
\noindent\begin{tabularx}{\textwidth}{|>{\centering\arraybackslash}X|>{\centering\arraybackslash}X|>{\centering\arraybackslash}X|}
  \hline
  起止年月 & 学习或工作单位 & 备注 \\\hline 
  XXX年XX月至XXXX年XX月  & 在XXXX学校XXXX专业攻读学士学位  &    \\\hline
  XXX年XX月至XXXX年XX月  & 在XXXX单位从事XXXX岗位的工作  &    \\\hline
  XXX年XX月至XXXX年XX月  & 在XXXX学校XXXX专业攻读博士学位  &    \\\hline
  & & \\ \hline
\end{tabularx}

\section*{二、在学期间从事的科研工作}

(1)高性能低功耗深度神经网络处理器芯片设计(USTB06500093),主要参与人员,2017.11-2022.11。

(2)... 

\section*{三、在学期间所获的科研奖励}

(1)博士研究生国家奖学金

\section*{四、在学期间发表的论文}

\begin{enumerate}[label={[\arabic*]}, ,itemindent=*,leftmargin=*] 
\item	\uline{\textbf{Liu M K}}, LIU C X, ZHANG S T, et al. Research on Industry Development Path Planning of Resource-Rich Regions in China from the Perspective of “Resources, Assets, Capital” [J]. Sustainability, 2021, 13(7): 3988-3988.

\item	\uline{\textbf{Liu M K}}, LIU C X, PEI X D, et al.   Sustainable Risk Assessment of Resource Industry at Provincial Level in China [J]. Sustainability, 2021, 13(8): 4191-4191.

\item \uline{\textbf{刘}}\uline{\textbf{明}}\uline{\textbf{凯}}, 张寿庭, 刘昌新, 等. 基于“三资”视角的矿山企业绿色可持续发展路径研究[J]. 中国矿业, 2020, 29(07): 35–43.

\item \uline{\textbf{刘}}\uline{\textbf{明}}\uline{\textbf{凯}}, 张红艳, 王新宇. 广西有色金属产业链关联测度与发展路径规划研究[J]. 中国国土资源经济, 2020, 33(12): 65-74.

\item	虎海峰 ,\uline{\textbf{刘}}\uline{\textbf{明}}\uline{\textbf{凯}} , 樊杰. 黄河流域经济-社会-生态耦合协调效应评价及预测[J]. 中国管理科学. 待刊出.
\end{enumerate}

{\color{blue} 盲审说明(正式写作请删除此说明):}

盲审论文需要隐藏掉所有会影响盲审结果的论文作者及其导师的信息,以便论文评
阅人能够公正的进行评阅。

当学校要求提供盲审论文时,请按如下方法制作。

1) 对于论文中下列学生和导师信息。请将学生姓名、学生学号、导师姓名,依次全
部替换为[本论文作者] 、[论文作者学号]、[本论文导师]。论文中上述信息均需要替换,
包括作者研究成果等部分的有关信息。

2) 在研究成果中,论文作者发表的文章列表中应隐去所有作者的名字,只标明论文
期刊名,级别,发表年份。

3)盲审论文,请不要填写致谢,致谢页除标题、页眉、页码外请保持空白。

4)其他会影响盲审结果的信息,请采用类似方式处理。

5)提交的盲审论文应为正式论文,除了上述替换后的信息外,应为可以评阅的正式
论文。

6)论文封面,请填写专业名称、论文题目等,将学号和姓名项目保持空白不填。制
作盲审论文时,论文书脊、封二、题名页请删除。

替换前后将文档分别保存,以便盲审论文与其他论文分开管理。

{\color{blue} 盲审样例(正式写作请删除此样例):}

五、在学期间发表的论文:


[1] \textbf{第一作者}.中国矿业.中文核心期刊.2020.

[2] \textbf{第二作者(导师一作)}.中国矿业.中文核心期刊.2020.


%\pagestyle{empty}
%% 关于论文使用授权的说明
\postDeclarations
%% 学位论文数据集
\cleardoublepage
\chapter*{学位论文数据集}\addcontentsline{toc}{chapter}{学位论文数据集}
\vskip16.5pt
\noindent\zihao{5}\begin{tabu} to \textwidth{|X|X|X|X|X|}
\hline
关键词* & 密级* & 中图分类号* & UDC & 论文资助 \\\hline
 & & & & \\\hline
\multicolumn2{|l|}{学位授予单位名称*} & {学位授予单位代码*} & 学位类别* & 学位级别* \\\hline
\multicolumn{2}{|l|}{北京科技大学}&10008  & & \\\hline
\multicolumn2{|l}{论文题名*} & \multicolumn2{|l|}{并列题名} & 论文语种* \\\hline
\multicolumn2{|l}{} & \multicolumn2{|l|}{} &  \\ \hline
作者姓名*&\multicolumn{2}{l|}{} & 学号* &  \\\hline
\multicolumn2{|l|}{培养单位名称*} & 培养单位代码* & 培养单位地址 & 邮编 \\\hline
\multicolumn2{|l|}{北京科技大学} &10008 & 北京市海淀区学院路30号& 100083\\\hline
\multicolumn2{|l|}{学科专业*} & 研究方向* & 学制* & 学位授予年* \\\hline
\multicolumn2{|l|}{} & & & \\\hline
论文提交日期* & \multicolumn4{l|}{} \\\hline
导师姓名*&\multicolumn2{l|}{} & 职称* &  \\\hline
评阅人 &答辩委员会主席* & \multicolumn3{l|}{答辩委员会成员} \\\hline
& & \multicolumn3{l|}{} \\
& & \multicolumn3{l|}{} \\\hline
\multicolumn5{|l|}{电子版论文提交格式  文本(\, )  图像(\, ) 视频(\, ) 音频(\, ) 多媒体(\, ) 其他(\, )}\\
\multicolumn5{|l|}{推荐格式:application/msword; application/pdf} \\\hline
\multicolumn2{|l}{电子论文出版(发布)者} & \multicolumn2{|l|}{电子论文出版(发布)地} & 权限声明 \\\hline
\multicolumn2{|l}{} & \multicolumn2{|l|}{} & \\\hline
论文总页数* & \multicolumn4{l|}{} \\\hline
\multicolumn5{|l|}{共33项,其中带*为必填数据,为22项。}\\\hline
\end{tabu}

\end{document}
