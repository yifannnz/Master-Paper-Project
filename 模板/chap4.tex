\chapter{研究结果}

自行确定是否使用四级标题。论文中出现英文时需要使用Times New Roman字体。这个结构可以根据自己论文需要进行调整。

如图\figref{fig1},是这里插图的样式示例,以下不再重复,根据论文实际情况添加即可。

\begin{figure}[hbp]
	\centering
	\subfloat[示意图1]
	{\includegraphics[width=0.65\textwidth]{images/image1.png}\label{fig1:subfig1}}\\
	\subfloat[示意图2]
	{\includegraphics[width=0.65\textwidth]{images/image1.png}\label{fig1:subfig2}}\\
	\bicaption{非线性构形状态转移过程示意图}{Schematic diagram of nonlinear configuration state transition process.}
	\vspace{0.3em}
	{\centering (资料来源:XXXXXXXXXXXXXXXXXXX)\par}
	\label{fig1}
\end{figure}

续图示例:

\begin{figure}[hbp]
	\centering
	\subfloat[示意图1]
	{\includegraphics[width=0.6\textwidth]{images/image1.png}
		\label{fig2:subfig1}}
	\centering\bicaption{数据图}{Data Graph}
	\label{fig2}
\end{figure}

\begin{figure}[htb]
\ContinuedFloat
\addtocounter{subfigure}{-1}
\centering
\subfloat[示意图2]
{\includegraphics[width=0.6\textwidth]{images/image1.png}
	\label{fig2:subfig2}}
\bicaption{续}{Cont}
\end{figure}

如表\tabref{label:tab1},是这里表格的样式示例,以下不再重复,根据论文实际情况添加即可。

\begin{table}[hbp]
	\centering
	\bicaption{线性五杆结构各自由度随机反应数值特征}{Numerical characteristics of random response of linear five-bar structure with various degrees of freedom}
\label{label:tab1}
\noindent\renewcommand{\arraystretch}{0.9}
\begin{tabular}{P{5.25em}P{5.25em}P{4.5em}P{4.5em}P{4.5em}P{4.5em}P{4.5em}}
\toprule
\multirow{2}{*}{\centering $x_1$/m} & \multicolumn{3}{c}{$F_x$} & \multicolumn{3}{c}{$x_2$}\\
\cmidrule(lr){2-4}\cmidrule(lr){5-7}
 & 均值/N & 标准差/N & 变异系数 & 均值/m & 标准差/m & 变异系数\\
\midrule
0.000000 & 0.000000 & 0.000000 & 0.000000 & 0.000000 & 0.000000 & 0.000000\\
0.000100 & 206.006806 & 150.245905 & 0.729325 & 0.000024 & 0.000013 & 0.541667\\
0.000200 & 412.013613 & 215.100090 & 0.522070 & 0.000049 & 0.000018 & 0.367347\\
0.000300 & 618.020419 & 266.613296& 0.431399	& 0.000073 & 0.000022 & 0.301370\\
\bottomrule
\end{tabular}
\end{table}

续表示例:
\begin{table}[hbp]
    \centering\bicaption{国际单位制中具有专门名称的导出单位}{Export units of special name in International System of Units}
    \label{label:tab2}
    \noindent\renewcommand{\arraystretch}{0.9}
    \begin{tabular}{P{12em}P{6em}P{6em}P{8em}}
        \toprule
        量的名称 & 单位名称 & 单位符号 & 其他表示示例 \\\midrule
        频率  & 赫[兹] & Hz & $\mathrm{s}^{-1}$ \\
        力;重力 & 牛[顿] & N & $\mathrm{kg}\cdot\mathrm{m}/\mathrm{s}^2$ \\
        压力,压强;应力 & 帕[斯卡] & Pa & $\mathrm{N}/\mathrm{m}^2$ \\
        能量;功;热 & 焦[耳] & J & $\mathrm{N}\cdot\mathrm{m}$ \\
        		功率;辐射通量&	瓦[特]	& W	 &J/s\\
        电荷量&	库[仑]	& C	& A·s\\
        电位;电压;电动势&	伏[特]&	V	& W/A\\
        		电容&	法[拉]&	F&C/V\\
        				电阻&	欧[姆]&	Ω&	V/A\\
        \bottomrule
    \end{tabular}
\end{table}

\begin{table}[hbp]
	\ContinuedFloat
	\centering
	\bicaption{续}{Cont}
	\noindent\renewcommand{\arraystretch}{0.9}
	\begin{tabular}{P{12em}P{6em}P{6em}P{8em}}
		\toprule
		量的名称 & 单位名称 & 单位符号 & 其他表示示例 \\\midrule  
		电导&	西[门子]&	S&	A/V\\    
		磁通量	& 韦[伯]&	Wb&	V·s\\
		磁通量密度,磁感应强度	& 特[斯拉]&	T	& $\mathrm{Wb}/\mathrm{m}^2$\\
		电感	& 亨[利]&	H&	$\mathrm{Wb}/\mathrm{A}$\\
		摄氏温度&	摄氏度	& ℃	&\\
		光通量&	流明	& lm	& $\mathrm{cd}·\mathrm{sr}$\\
		光照度&	勒[克斯]&	lx&	$\mathrm{lm}/\mathrm{m}^2$\\
		放射性活度&	贝可[勒尔]&	Bq&$\mathrm{s}^{-1}$\\
		吸收剂量&	戈[瑞]&	Gy&	$\mathrm{J}/\mathrm{kg}$\\
		剂量当量&	希[沃特]&	Sv&	$\mathrm{J}/\mathrm{kg}$\\
		\bottomrule
	\end{tabular}
\end{table}

这里公式的样式示例,以下不再重复,根据论文实际情况添加即可。

将剩余的试样分成两份后放入烘箱中,分别测量放入前和烘完后的质量,并对土体的含水率进行计算,密度取两份的平均值,计算公式为:
\begin{equation}
\label{eq:04:01}
    w = \frac{m_{water}}{m_2} = \frac{m_1-m_2}{m_2}\times 100\%
\end{equation}
式中:$w$为含水率;$m_1$为放之前的质量;$m_2$为烘完之后的质量;$m_{water}$为试样的含水量。

说明1:根据实际情况占用1行或多行,编号右端对齐,表达式与编号之间可用“…”连接,如有两个以上的表达式,应用从“1”开始的阿拉伯数字进行编号,并将编号置于圆括号内;表达式较多时,可分章编号,如 \eqref{eq:04:01} 表示第四章第一个表达式。

