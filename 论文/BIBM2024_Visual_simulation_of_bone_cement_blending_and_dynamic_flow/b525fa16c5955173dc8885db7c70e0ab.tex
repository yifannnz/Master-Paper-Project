\documentclass[conference]{IEEEtran}
\IEEEoverridecommandlockouts
% The preceding line is only needed to identify funding in the first footnote. If that is unneeded, please comment it out.
\usepackage{cite}
\usepackage{amsmath,amssymb,amsfonts}
\usepackage{algorithm}
\usepackage{algpseudocode}
\usepackage{graphicx}
\usepackage{textcomp}
\usepackage{xcolor}
\def\BibTeX{{\rm B\kern-.05em{\sc i\kern-.025em b}\kern-.08em
    T\kern-.1667em\lower.7ex\hbox{E}\kern-.125emX}}
\begin{document}

\title{Visual simulation of bone cement blending and dynamic flow
\thanks{Identify applicable funding agency here. If none, delete this.}
}

\author{\IEEEauthorblockN{1\textsuperscript{st} Given Name Surname}
\IEEEauthorblockA{\textit{dept. name of organization (of Aff.)} \\
\textit{name of organization (of Aff.)}\\
City, Country \\
email address or ORCID}
\and
\IEEEauthorblockN{2\textsuperscript{nd} Given Name Surname}
\IEEEauthorblockA{\textit{dept. name of organization (of Aff.)} \\
\textit{name of organization (of Aff.)}\\
City, Country \\
email address or ORCID}

}

\maketitle

\begin{abstract}
Bone cement filling is an important method for preventing osteoporosis and treating fractures. In bone cement filling surgery, the preparation and dosage of bone cement usually depend on specific product manuals and the doctor's experience. If bone cement is not used properly, it may cause additional damage. For teaching and auxiliary medical purposes, this paper proposes a multiphase non-Newtonian fluid simulation method to simulate and visualize the flow behavior during the wet sand phase of bone cement blending and polymerization. With this method, it is possible to show intuitively the application process of bone cement under different scene settings and obtain accurate and dynamic bone cement effects. Compared with other methods, our method can accurately simulate highly viscous mixed fluids at large time steps.
\end{abstract}

\begin{IEEEkeywords}
\textcolor{red}{component, formatting, style, styling, insert}
\end{IEEEkeywords}

\section{Introduction}
With the increase in life expectancy and the aging population, approximately 200 million people worldwide suffer from osteoporosis, and diseases such as fractures can occur at any time. Bone cement filling is an early intervention measure for osteoporosis and an important treatment method for fracture and other diseases \textcolor{red}{\textbf{Paper [ref]}}.

In the process of using bone cement, doctors usually need to carefully mix it according to their own experience and specific product instructions to prevent bone cement leakage or insufficient dispersion, which causes some confusion regarding the actual use and expected effects of doctors (Fig.~\ref{fig:cases}), and is not conducive to low-cost teaching. The use of computer simulation and visualization technology can help doctors intuitively understand different proportions of the preparation and the impact of the operation on the surgical results at a low cost.

\begin{figure}[htbp]
\centerline{\includegraphics[width=0.95\columnwidth]{pics/cases.pdf}}
\caption{$a)$ is the schematic diagram of the bone cement filled inside the bone, and the green one is the bone cement. Leakage may occur if the cement is too thin: $b)$ when there is a fracture, the pressure inside the bone and a low modulation ratio may cause the cement to leak from the fracture, and $d)$ when the cement is diffused inside the bone, it may contact the nerve (the red line) and cause complications. If the cement is too thick, $c)$ it may cause obstruction, and bone cannot be fully populated.}
\label{fig:cases}
\end{figure}

Currently, there is a lack of research on the simulation and visualization of this highly viscous mixed fluid. To solve this problem, this paper proposes a new multiphase non-Newtonian fluid simulation method for bone cement preparation and coarse sand flow. The main contributions of this paper are:
\begin{itemize}
\item A viscoelastic stress method is proposed based on an implicit mixture model and conformation tensor;
\item A unified framework is constructed for describing Newtonian and non-Newtonian fluids;
\item A bonding effect network is developed for controlling implicit mixture model phase transfer.
\end{itemize}

\section{Related Work}

\subsection{Bone cement filling }

Bone cement filling is an important method for orthopedic joint replacement and the treatment of osteoporosis. In the 1960s, charnley \textcolor{red}{\textbf{Paper [ref]}} was the first to apply polymethyl methacrylate (PMMA) to the fixation of femoral prosthesis and acetabulum. In the field of dentistry, since the 1970s, more bone cement materials have been extensively investigated, such as polyzinc carboxylate \textcolor{red}{\textbf{Paper [ref]}} and glass polylinoate cement \textcolor{red}{\textbf{Paper [ref]}}.

With the advancement of minimally invasive surgical techniques, vertebral augmentation \textcolor{red}{\textbf{Paper [ref]}} has become the most commonly used surgical procedure for treating osteoporotic compression fractures. By injecting bone cement, the damaged bone can be preserved in its original form. During the process of vertebral augmentation, bone cement is injected as a filler into the damaged bone to maintain its morphology (Fig.~\ref{fig:cement_filling}). For the use of bone cement, strict operating procedures are usually required to reduce the risk of postoperative complications. Improper operation can cause bone cement leakage, and a large amount of leakage can lead to fatal consequences such as pulmonary embolism and paraplegia \textcolor{red}{\textbf{Paper [ref]}}.

\begin{figure}[htbp]
\centerline{\includegraphics[width=0.95\columnwidth]{pics/cement filling.pdf}}
\caption{Structural support. The left side shows the deformed vertebrae caused by damage or osteoporosis. Bone cement filling can effectively restore the original structure (the green area is bone cement).}
\label{fig:cement_filling}
\end{figure}

\subsection{Physics-based simulation methods}

Physics-based simulation simulates the motion of objects on a computer through precise physical laws, including fluid simulation, rigid body simulation, etc. Common simulation methods include grid-based methods, particle-based methods, and hybrid methods. The grid method has high computational accuracy, while the particle method can capture finer details. Meanwhile, different types of fluid can be divided into single-phase fluid simulation and multiphase fluid simulation. Single-phase fluid simulation can model most common substances, such as water, honey, etc. Multiphase fluid simulation can simulate a mixture of multiple miscible or non-miscible components. To model bone cement, this paper chooses a multiphase fluid framework based on the particle method.

\noindent\textbf{The Smoothed Particle Hydrodynamics (SPH):} The SPH method is a particle-based physical simulation method \textcolor{red}{\textbf{Paper [ref]}}. This method employs SPH particles to sample the fluid and calculate the physical field of the fluid. It has high numerical accuracy and can capture details such as the splash of droplets on the free surface of the fluid. Currently, there are various SPH-based fluid solvers, including Weakly compressible SPH (WCSPH) \textcolor{red}{\textbf{Paper [ref]}}, Predictive-Corrective Incompressible SPH (PCISPH) \textcolor{red}{\textbf{Paper [ref]}}, Divergence-Free SPH (DFSPH) \textcolor{red}{\textbf{Paper [ref]}}, etc. Among them, DFSPH is the most advanced fluid solver that can guarantee the incompressibility and divergence-free conditions of the fluid.

\noindent\textbf{Multiphase Fluid Simulation:} Multiphase fluid simulation is to simulate fluids in multiple phases, which can refer to different material components, such as a mixture of water and sand, or different states of the same material, such as water and bubbles. Current research based on multiphase flow simulation focuses on bubble simulation \textcolor{red}{\textbf{Paper [ref]}}, phase transition \textcolor{red}{\textbf{Paper [ref]}}, numerical accuracy \textcolor{red}{\textbf{Paper [ref]}}, etc. Commonly used multiphase fluid simulation methods include the two-fluid model \textcolor{red}{\textbf{Paper [ref]}}, suspension model \textcolor{red}{\textbf{Paper [ref]}}, and Mixture model \textcolor{red}{\textbf{Paper [ref]}}, etc. Among them, the mixture model has received much attention recently. In the mixture model, the volume fraction is utilized to represent the proportion of different phases at the same sampling position. The overall discretization method uses SPH, and the physical field of the fluid is calculated using multiphase fluid dynamics.

\section{Multiphase non-Newtonian fluid simulation for bone cement flow}
The use of bone cement involves two processes: modulation and injection. In the actual modulation process, two substances, organic solvent and solute, are mixed according to different proportions. In the injection process, different proportions of mixing will lead to varying flow performance. The mixed fluid is a non-Newtonian fluid with viscoelastic shear thinning, and polymerization reactions will change the physical properties. The fluid polymerization period can be further divided into the coarse sand stage, drawing stage, clumping stage, and hardening stage. For the above process, this paper models the fluid based on the implicit mixture model and the polymer conformation tensor method, so as to simulate the dynamic modulation process and the coarse sand flow.

\subsection{SPH-based fluid simulation}\label{AA}
The SPH method discretizes the continuous fluid medium into independent SPH particles, indicating that the continuous physical field of the fluid can be computed by numerical methods to simplify the solution of differential equations. SPH can be understood as a discretization method for spatial fields and spatial differential operations. Its essence is that the physical field information in space (e.g., fluid density, mass, velocity, pressure, etc) is defined on SPH particles. It determines the state information of the current particle at the next moment by calculating the contribution of neighbor particles and kernel function weighting (Fig. ~\ref{fig:sph_particles}). Specifically, the physical field is estimated by the following equation:
\begin{equation}
    \mathbf{A}_i = \sum _{j \in N(i)} \frac{m_j}{\rho _j}\mathbf{A}_jW_{ij}
\end{equation}
where $i$ is the index of the particle, and $j$ is the index of its neighbors, $\mathbf{A}$ denotes the physical field, $N(i)$ represents the neighborhood of particle $i$, $m$ denotes the particle mass, $\rho$ represents the particle density, and $W$ is a Gaussian-like kernel function, usually a cubic spline kernel \textcolor{red}{\textbf{Paper [ref]}}.

\begin{figure}[htbp]
\centerline{\includegraphics[width=0.7\columnwidth]{pics/SPH particles.pdf}}
\caption{SPH particle samples. $j_s$ represents the neighbors of particle $i$, and the curve denotes the Gaussian-like kernel function. When two particles become closer, $W_{ij}$ will have a larger value, and when two particles are out of range, $W_{ij}$ will have a value of 0.}
\label{fig:sph_particles}
\end{figure}

Differential computations are commonly used in numerical simulations of fluids, and with the assistance of standard SPH discretization, only the kernel function needs to be concerned in solving the differential; therefore, the divergence of the field and the Laplacian operator can be computed as follows:
\begin{equation}
    \begin{aligned} 
        \nabla \mathbf{A}_i &= \sum _{j \in N(i)} \frac{m_j}{\rho _j}\mathbf{A}_j \otimes \nabla W_{ij} \\
        \nabla \cdot \mathbf{A}_i &= \sum _{j \in N(i)} \frac{m_j}{\rho _j}\mathbf{A}_j\nabla W_{ij} \\
        \nabla ^2 \mathbf{A}_i &= \sum _{j \in N(i)} \frac{m_j}{\rho _j}\mathbf{A}_j \nabla ^2 W_{ij} \\
    \end{aligned}
\end{equation}
where $\mathbf{a} \otimes \mathbf{b} = \mathbf{a}\mathbf{b}^T$. These three equations define the computation of the gradient, divergence, and Laplacian of the field, respectively.

\subsection{Implicit mixture model with improved phase transfer}\label{BB}

\paragraph{\noindent\textbf{Implicit Mixture Model}} The mixture model uses the volume fraction scheme (Fig.~\ref{fig:vol_frac}) to represent the concentration of each phase and calculates the physical parameters at the phase level and the mixture level using multiphase fluid dynamics.

\begin{figure}[htbp]
\centerline{\includegraphics[width=0.45\columnwidth]{pics/vol frac.pdf}}
\caption{The volume fraction scheme in the mixture model with two-phase flow as an example.}
\label{fig:vol_frac}
\end{figure}

The sum of the phase volume fraction of a particle is normalized as follows:
\begin{equation}
    \sum \alpha_k = 1
\end{equation}
where $\alpha$ denotes phase volume fraction, and $k$ is the phase index.

The velocity field at the mixture level is reconstructed from the phase-level velocity field:
\begin{equation}
    \mathbf{v}_{mix} = \sum \alpha _k \mathbf{v}_k
\label{eq:mix_vel}
\end{equation}
and the density of the mixture particle is computed as:
\begin{equation}
    \rho_{mix} = \sum \alpha_k \rho_k^0
\label{eq:mix_dens}
\end{equation}
where $\rho_k^0$ is the rest density of phase $k$.

The velocity field at the mixture level is utilized to represent the actual fluid motion, which depends on the calculation of the physical field at the phase level. Here, the work of the implicit mixture model by \textcolor{red}{\textbf{Paper [ref]}} is introduced, where an implicit reconstruction method between mixture level and phase level was derived to achieve higher numerical accuracy. 

In the implicit mixture model, the effects of gravity, pressure, and viscous forces are calculated respectively. The gravity $\mathbf{g}$ belongs to the volume force and is applied equally to each phase. The calculation of pressure and viscous forces is as follows:
\begin{equation}
        \frac{D\mathbf{v}_k^p}{Dt} = \frac{\mathbf{M}_{mix}^p}{\rho_{mix}} \left(C_d+(1-C_d)\frac{\rho_{mix}}{\rho_k^0}\right)
\label{eq:p_force}
\end{equation}
\begin{equation}
    \frac{D\mathbf{v}_k^\nu}{Dt} = C_d\frac{\mathbf{M}_{mix}^\nu}{\rho_{mix}} + (1-C_d)\frac{\mathbf{M}_k^\nu}{\alpha_k \rho_k^0}
\end{equation}
where $\frac{\mathbf{v}_k}{Dt}$ denotes the acceleration associated with different forces; superscripts $p$ and $\nu$ denote pressure and viscosity, respectively; $\mathbf{M}$ represents the momentum source; $C_d \in [0, 1]$ is the model parameter derived from the implicit mixture model, and it is used to adjust the degree of influence of the mixture on the phase. A detailed explanation can be found in \textcolor{red}{\textbf{Paper [ref]}}.

\paragraph{\noindent\textbf{Improved Phase Transfer}} The phase transfer in the mixture model mainly involves two factors, namely, interphase drag force and diffusion. The calculation of drag force depends on the drift velocity of the phase, which is defined as follows:
\begin{equation}
    \mathbf{v}_k^{drift} = \mathbf{v}_k - \mathbf{v}_{mix}
\label{eq:drift_vel}
\end{equation}
and the formula for the change of phase fraction due to the two factors is:
\begin{equation}
    \begin{aligned} 
        \frac{D\alpha_{i,k}}{Dt} &= -\sum V_0(\alpha_{i,k}\mathbf{v}_{i,k}^{drift}+\alpha_{j,k}\mathbf{v}_{j,k}^{drift})\nabla \cdot W_{ij} \\
        \nabla ^2 \alpha_{i,k} &= C_f\sum (\alpha_{i,k}-\alpha_{j,k})\frac{\mathbf{x}_{ij}\cdot \nabla W_{ij}}{\|\mathbf{x}_{ij}\|^2 + \epsilon}
    \end{aligned}
\label{eq:phase_trans}
\end{equation}
where subscript $i,k$ denotes the phase $k$ of particle $i$, $C_f$ is the diffusion coefficient, $V_0$ denotess the rest volume of particle, $\mathbf{x}_{ij}=\mathbf{x}_i-\mathbf{x}_j$ is the distance vector between particles $i$ and $j$, $\mathbf{x}$ represents the position of the particle, and $\epsilon$ is a small value to avoid singularities.

In the implicit mixture model, $C_d$ is used to adjust the degree of influence of the mixture on the phase: When $C_d=0$, the phase is completely not affected by the mixture, and when $C_d=1$, the phase is completely controlled by the mixture. In the physical field, it is understood as the relationship between the velocity field of the mixture level and the phase level. When the phase velocity field completely follows the mixture, the phase does not separate, which affects the phase transport. The original implicit mixture model sets $C_d$ as a constant, indicating that the effect of the mixture on the phase is constant, but this is not true in some solutions that will react, so a mechanism is required to calculate the effect of the solute concentration on the phase transfer.

Calculating the intermolecular combination between two molecules can lead to a large overhead, so this paper proposes a bonding effect network to simplify this calculation process. Specifically, the $C_d$ of the multiphase particle is dynamically changed according to the solute concentration. For one mixture, a basic $C_d^0$ is set, and the current dynamics $C_d$ is estimated by the SPH method. This paper focuses on two-phase fluid, where $\alpha_{k_1}$ denotes the liquid phase, $\alpha_{k_2}$ denotes the polymer phase:
\begin{equation}
    C_d = C_d^0 + (1-C_d^0)\sum_{j\neq i} V_0 \alpha_{j,k_2} W_{ij}
\label{eq:update_Cd}
\end{equation}
when the solute concentration around the particle is high, $C_d \rightarrow 1$, and this will block the phase transfer to simulate the case of phase coupling Fig. ~\ref{fig:phase_trans}.

\begin{figure}[htbp]
\centerline{\includegraphics[width=0.85\columnwidth]{pics/phase trans.pdf}}
\caption{Bonding Effect Network. This mechanism is used to model the coupling of phases in an inhomogeneous solution where reactions occur. Taking the regions of three concentrations as examples, a greener particle color indicates a higher polymer phase fraction, the arrow indicates the blocking effect of the neighbor on the phase transfer of the target particle $i$, and a thicker arrow indicates a greater blocking effect.}
\label{fig:phase_trans}
\end{figure}

\subsection{Polymer conformation tensor method in the mixture model}\label{CC}

The conformation tensor is a tool for describing the material distribution in solution \textcolor{red}{\textbf{Paper [ref]}}, and it can describe Newtonian and non-Newtonian fluids. The classical configuration update formula is as follows:
\begin{equation}
    \frac{D\mathbf{U}}{Dt} = \mathbf{U}\nabla \mathbf{v}+(\nabla \mathbf{v})^T\mathbf{U}-\frac{1}{\lambda}(\mathbf{U}-\mathbf{I})
\label{eq:update_U}
\end{equation}
where $\mathbf{U}$ denotes the conformation tensor, which is a $3\times3$ matrix in 3D cases. $\lambda$ denotes the relaxation time and is used to describe the fluid viscoelasticity. However, this model can only describe Newtonian fluid, and the viscoelasticity of non-Newtonian fluids has a nonlinear relationship with the shear rate. Therefore, a variant model is used in this paper:
\begin{equation}
    \frac{D\mathbf{U}}{Dt} = \mathbf{U}\nabla \mathbf{v}+(\nabla \mathbf{v})^T\mathbf{U}-\frac{1}{\lambda}(\mathbf{U}-\mathbf{I}) - \gamma(\mathbf{U}-\mathbf{I})\mathbf{U}
\end{equation}
where $\gamma(\mathbf{U}-\mathbf{I})\mathbf{U}$ is a non-linear term to provide nonlinear capabilities, specifically shear thinning. $\gamma \in [0,1]$ is the thinning factor. A larger value of $\gamma$ contributes to a stronger thinning effect (Fig.~\ref{fig:cap_curve}).

The mixture-level stress based on the conformation tensor is defined as follows:
\begin{equation}
    \mathbf{\tau}_{mix} = c\eta_s(\mathbf{U}-\mathbf{I})
\label{eq:stress}
\end{equation}
where $c$ denotes the polymer concentration of the solution, which is equal to $\alpha _{k_2}$ in our multiphase framework. $\eta _s$ denotes the viscosity of the solution. The symmetric formulation of SPH \textcolor{red}{\textbf{Paper [ref]}} is employed to calculate the stress force:
\begin{equation}
    \frac{1}{\rho_{i,mix}}\nabla \cdot \mathbf{\tau}_{i,mix} = \sum \left( \frac{\mathbf{\tau}_{i,mix}}{\rho^2_{i,mix}} + \frac{\mathbf{\tau}_{j,mix}}{\rho^2_{j,mix}}\right)\nabla W_{ij} 
\label{eq:stress_force}
\end{equation}

The conformation tensor method does not define the stress calculation for each phase, and the multiphase framework requires reconstructing the mixture-level velocity from the phase-level velocity. Based on $\sum \mathbf{\tau}_ k = \mathbf{\tau}_{mix}$, this paper assumes that:
\begin{equation}
    \mathbf{\tau}_k = \alpha_k \mathbf{\tau}_{mix}
\end{equation}
in this way, the phase velocity can be updated according to the mixture-level stress:
\begin{equation}
    \frac{D\mathbf{v}_k^{visc}}{Dt}=\frac{\nabla \cdot \tau_k}{\alpha_k \rho_ k^0}
\label{eq:visc_force}
\end{equation}
where $\frac{D\mathbf{v}_k^{visc}}{Dt}$ represents the acceleration associated with viscoelastic force.

\begin{figure}[htbp]
\centerline{\includegraphics[width=0.9\columnwidth]{pics/cap curve.pdf}}
\caption{The shear thinning curve of our model. The viscoelasticity of shear thinning fluid decreases as the shear rate increases. In our model, the larger the value of $\gamma$, the stronger the shear thinning effect.}

\label{fig:cap_curve}
\end{figure}

Our algorithm is presented in Algorithm. ~\ref{algo:algo}. At the beginning of the algorithm, $\mathbf{U} \gets \mathbf{I}$ is set, and a unified grid is utilized to update the neighbors of each particle. Then, the simulation loop starts to update the flow field. In each iteration, some state variables in \textbf{Substep. 1} are first precomputed, and then the multiphase flow field is updated by the DFSPH algorithm and the implicit mixture model in \textbf{Substeps 2, 3, and 4}. Subsequently, the viscoelastic stress is calculated by our proposed viscoelastic non-Newtonian solver in \textbf{Substep 5}. Finally, the phase transfer is calculated from the velocity field in \textbf{Substep 6}, and the neighbors are updated.

\begin{algorithm}
\caption{Multiphase viscoelastic non-Newtonian fluid simulation algorithm.}
\begin{algorithmic}[l]

\State \textbf{PREPARE:}
    \State \hspace{5mm} \textbf{For} All Particles \textbf{do}:
        \State \hspace{10mm} $\mathbf{U} \gets \mathbf{I}$
        \State \hspace{10mm} Init neighbors.

\State \textbf{LOOP:}
    \State \hspace{5mm} \textbf{1. Precmpute }
    \State \hspace{5mm} \textbf{For} All Particles \textbf{do}:
        \State \hspace{10mm} compute mixture vel $\mathbf{v}_{mix}$ \Comment{Eq.~\ref{eq:mix_vel}}
        \State \hspace{10mm} compute phase drift vel $\mathbf{v}^{drift}_k$ \Comment{Eq.~\ref{eq:drift_vel}}
        \State \hspace{10mm} update mixture density  $\rho _{mix}$ \Comment{Eq.~\ref{eq:mix_dens}}
        \State \hspace{10mm} update $C_d$ \Comment{Eq.~\ref{eq:update_Cd}}

    \State \hspace{5mm} \textbf{2. Div-Free Solver }
    \State \hspace{5mm} \textbf{For} All Particles \textbf{do}:
        \State \hspace{10mm} compute div-free force $\mathbf{M}_{mix}^{p,div}$ using \textbf{DFSPH} 
        \State \hspace{10mm} update phase vel $\mathbf{v}_k$ \Comment{Eq.~\ref{eq:p_force}}
        \State \hspace{10mm} update mixture vel $\mathbf{v}_{mix}$ \Comment{Eq.~\ref{eq:mix_vel}}

    \State \hspace{5mm} \textbf{3. Advect }
    \State \hspace{5mm} \textbf{For} All Particles \textbf{do}:
        \State \hspace{10mm} update phase vel using $\mathbf{v}_k \gets \mathbf{v}_k + \mathbf{g}\Delta t$
        \State \hspace{10mm} update mixture vel $\mathbf{v}_{mix}$ \Comment{Eq.~\ref{eq:mix_vel}}

    \State \hspace{5mm} \textbf{4. Incompressible Solver }
    \State \hspace{5mm} \textbf{For} All Particles \textbf{do}:
        \State \hspace{10mm} compute incomp force $\mathbf{M}^{p}_{mix}$ using \textbf{DFSPH} 
        \State \hspace{10mm} update phase vel $\mathbf{v}_k$ \Comment{Eq.~\ref{eq:p_force}}
        \State \hspace{10mm} update mixture vel $\mathbf{v}_{mix}$ \Comment{Eq.~\ref{eq:mix_vel}}

    \State \hspace{5mm} \textbf{5. Viscoelastic Solver }
    \State \hspace{5mm} \textbf{For} All Particles \textbf{do}:
        \State \hspace{10mm} update comformation tensor $\mathbf{U}$ \Comment{Eq.~\ref{eq:update_U}}
        \State \hspace{10mm} compute viscoelastic force \Comment{Eq.~\ref{eq:stress},~\ref{eq:stress_force}}
        \State \hspace{10mm} update phase vel $\mathbf{v}_k$ \Comment{Eq.~\ref{eq:visc_force}}
        \State \hspace{10mm} update mixture vel $\mathbf{v}_{mix}$ \Comment{Eq.~\ref{eq:mix_vel}}

    \State \hspace{5mm} \textbf{6. Final Step }
    \State \hspace{5mm} \textbf{For} All Particles \textbf{do}:
        \State \hspace{10mm} update phase volume fraction $\alpha_k$ \Comment{Eq.~\ref{eq:phase_trans}}
        \State \hspace{10mm} update phase drift vel $\mathbf{v}^{drift}_k$ \Comment{Eq.~\ref{eq:drift_vel}}
        \State \hspace{10mm} update particle position using $\mathbf{x} \gets \mathbf{x}+\mathbf{v}_{mix}\Delta t$
        \State \hspace{10mm} update neighbors.

\end{algorithmic}
\label{algo:algo}
\end{algorithm}

\section{Experiments}


\section*{Acknowledgment}

The preferred spelling of the word ``acknowledgment'' in America is without 
an ``e'' after the ``g''. Avoid the stilted expression ``one of us (R. B. 
G.) thanks $\ldots$''. Instead, try ``R. B. G. thanks$\ldots$''. Put sponsor 
acknowledgments in the unnumbered footnote on the first page.





\begin{thebibliography}{00}
\bibitem{b1} G. Eason, B. Noble, and I. N. Sneddon, ``On certain integrals of Lipschitz-Hankel type involving products of Bessel functions,'' Phil. Trans. Roy. Soc. London, vol. A247, pp. 529--551, April 1955.
\bibitem{b2} J. Clerk Maxwell, A Treatise on Electricity and Magnetism, 3rd ed., vol. 2. Oxford: Clarendon, 1892, pp.68--73.
\bibitem{b3} I. S. Jacobs and C. P. Bean, ``Fine particles, thin films and exchange anisotropy,'' in Magnetism, vol. III, G. T. Rado and H. Suhl, Eds. New York: Academic, 1963, pp. 271--350.
\bibitem{b4} K. Elissa, ``Title of paper if known,'' unpublished.
\bibitem{b5} R. Nicole, ``Title of paper with only first word capitalized,'' J. Name Stand. Abbrev., in press.
\bibitem{b6} Y. Yorozu, M. Hirano, K. Oka, and Y. Tagawa, ``Electron spectroscopy studies on magneto-optical media and plastic substrate interface,'' IEEE Transl. J. Magn. Japan, vol. 2, pp. 740--741, August 1987 [Digests 9th Annual Conf. Magnetics Japan, p. 301, 1982].
\bibitem{b7} M. Young, The Technical Writer's Handbook. Mill Valley, CA: University Science, 1989.
\end{thebibliography}


\end{document}
