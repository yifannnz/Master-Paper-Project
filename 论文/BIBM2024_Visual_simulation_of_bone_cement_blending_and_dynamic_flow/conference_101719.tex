\documentclass[conference]{IEEEtran}
\IEEEoverridecommandlockouts
% The preceding line is only needed to identify funding in the first footnote. If that is unneeded, please comment it out.
\usepackage{cite}
\usepackage{amsmath,amssymb,amsfonts}
\usepackage{algorithm}
\usepackage{algpseudocode}
\usepackage{graphicx}
\usepackage{textcomp}
\usepackage{xcolor}
\newcommand\alex[1]{{\color{red}ALEX: #1}}
\newcommand\steffen[1]{{\color{purple}STEFFEN: #1}}

\def\BibTeX{{\rm B\kern-.05em{\sc i\kern-.025em b}\kern-.08em
    T\kern-.1667em\lower.7ex\hbox{E}\kern-.125emX}}
\begin{document}

\title{Visual simulation of bone cement blending\\ and dynamic flow

}

\author{\IEEEauthorblockN{1\textsuperscript{st} Long Shen}
\IEEEauthorblockA{
\textit{School of Intelligence}\\
\textit{Science and Technology,} \\
\textit{University of Science}\\
\textit{and Technology Beijing}\\
Beijing, China \\
sl$\_$111211@163.com}
\and
\IEEEauthorblockN{2\textsuperscript{nd}
Yalan Zhang\textsuperscript{*}}
\IEEEauthorblockA{
\textit{School of Intelligence}\\
\textit{Science and Technology,} \\
\textit{University of Science}\\
\textit{and Technology Beijing}\\
Beijing, China \\
zhangyl@ustb.edu.cn}
\and
\IEEEauthorblockN{3\textsuperscript{rd} Steffen Frey}
\IEEEauthorblockA{\textit{Bernoulli Institute} \\
\textit{University of Groningen}\\
Groningen, the Netherlands \\
s.d.frey@rug.nl}
\and
\IEEEauthorblockN{4\textsuperscript{th} Alex Telea}
\IEEEauthorblockA{\textit{Department of Information} \\
\textit{and Computing Sciences}\\
\textit{Utrecht University}\\
Utrecht, the Netherlands\\
a.c.telea@uu.nl}
\and
\IEEEauthorblockN{5\textsuperscript{th} Ji\v{r}\'i Kosinka}
\IEEEauthorblockA{\textit{Bernoulli Institute} \\
\textit{University of Groningen}\\
Groningen, the Netherlands \\
j.kosinka@rug.nl}
\and
\IEEEauthorblockN{6\textsuperscript{th} 
Xiaokun Wang}
\IEEEauthorblockA{
\textit{School of Intelligence}\\
\textit{Science and Technology,} \\
\textit{University of Science}\\
\textit{and Technology Beijing}\\
Beijing, China \\
wangxiaokun@ustb.edu.cn}
\and
\IEEEauthorblockN{7\textsuperscript{th} Xiaojuan Ban\textsuperscript{*}}
\IEEEauthorblockA{
\textit{Beijing Key Laboratory of Knowledge}\\
\textit{Engineeringfor Materials Science,}\\
\textit{University of Science}\\
\textit{and Technology Beijing}\\
Beijing, China \\
banxj@ustb.edu.cn}
}

\maketitle

\begin{abstract}
Bone cement filling is an important method for preventing osteoporosis and treating fractures. In bone cement filling surgery, the preparation and dosage of the cement usually depend on specific product manuals and the doctor's experience. If bone cement is not used properly, it may cause additional damage. For teaching and auxiliary medical purposes, for example, assisting doctors to observe the possible flow of bone cement, this paper proposes a multiphase non-Newtonian fluid simulation method to simulate and visualize the flow behavior during the wet sand phase of bone cement blending and polymerization. Our method enables  showing intuitively the application process of bone cement under different scene settings to obtain dynamic bone cement effects with high stability and performance. Compared with other methods, our method can simulate highly viscous mixed fluids efficiently and robustly, which supports our method's usage in the aforementioned training and experimentation scenarios.
\end{abstract}

\begin{IEEEkeywords}
Medical visualization, Bone filling simulation, Multiphase non-Newtonian fluid modeling,  Bone cement effects
\end{IEEEkeywords}

\section{Introduction}
\label{sec:intro}
With the rising life expectancy and an aging population, about 200 million people worldwide suffer from osteoporosis, fractures and other diseases that can occur at any time. Bone cement filling is an early intervention in osteoporosis and also an important treatment option for bone fractures~\cite{vaishya2013bone,saha1984mechanical}.

In the process of using bone cement, medical doctors usually need to carefully mix it according to their own experience and specific product instructions to prevent bone cement leakage or insufficient dispersion (Fig.~\ref{fig:cases}). This trial-and-error process can be time- and effort-intensive and is not conducive to low-cost teaching. Computer simulation and visualization techniques can help, at a low cost, doctors intuitively understand different preparation scenarios and the impact of the operation on the surgical results.

\begin{figure}[t]
\centerline{\includegraphics[width=0.95\columnwidth]{pics/cases.pdf}}
\caption{$a)$ Schematic diagram of bone cement (green) filled inside a bone (gray). $b)$ When there is a fracture(as shown by purple lines), and the cement is too thin, the pressure inside the bone and a low modulation ratio may cause the cement to leak from the fracture. $c)$ If the cement is too thick, it may cause obstruction, and the bone cannot be fully filled. $d)$ When the cement is diffused inside the bone, it may reach the nerve (red) and cause complications.}
\label{fig:cases}
\end{figure}

Current research on the simulation and visualization of highly viscous mixed fluids, such as cement, is limited. To address this problem, we propose a new multiphase non-Newtonian fluid simulation method for bone cement preparation and coarse sand flow, which is the early stage of the bone cement flow state. Our contributions are:
\begin{itemize}
\item a viscoelastic stress method based on an implicit mixture model and conformation tensor;
\item a unified framework for describing Newtonian and non-Newtonian fluids;
\item a bonding effect network for controlling the implicit mixture model's phase transfer.
\end{itemize}

Our model allows physicians to quickly set up Newtonian or non-Newtonian fluids with different properties and perform simulation of mixing, injection, etc. This in turn allows specialists to quickly investigate various scenarios and parameter settings (for the bone cement mix) and also helps with training.

We start by reviewing relevant related work (Section~\ref{sec:background}), then present our simulation method (Section~\ref{sec:method}), describe our experiments and results (Section~\ref{sec:experiments}), and finally conclude the paper (Section~\ref{sec:Lconclusion}).

\section{Related Work}
\label{sec:background}

\subsection{Bone cement filling }

Bone cement filling is an important method for orthopedic joint replacement and the treatment of osteoporosis. In the 1960s, Charnley~\cite{charnley1960anchorage} was the first to apply polymethyl methacrylate (PMMA) to the fixation of femoral prosthesis and acetabulum. More bone cement materials have been extensively investigated in dentistry since the 1970s such as polyzinc carboxylate~\cite{nicholson1993study} and glass polylinoate cement~\cite{wilson1971glass,sidhu2016review}.

With the advancement of minimally invasive surgical techniques, vertebral augmentation~\cite{truumees2004percutaneous,ebeling2019efficacy} has become the most commonly used surgical procedure for treating osteoporotic compression fractures. By injecting bone cement, the damaged bone can be preserved in its original form. During the process of vertebral augmentation, bone cement is injected as a filler into the damaged bone to maintain its morphology (Fig.~\ref{fig:cement_filling}). However, when using bone cement, strict operating procedures are required to reduce the risk of postoperative complications. Improper operation can cause bone cement leakage; a large amount of leakage can lead to fatal consequences such as pulmonary embolism and paraplegia ~\cite{breusch2005pulmonary,moussazadeh2015short}.

\begin{figure}[t]
\centerline{\includegraphics[width=0.95\columnwidth]{pics/cement filling.pdf}}
\caption{Structural support in vertebral augmentation. Left: Deformed vertebrae caused by damage or osteoporosis. Right: Bone cement filling (green) can effectively restore the original structure.}
\label{fig:cement_filling}
\end{figure}

\subsection{Physics-based simulation methods}
Given the expenses incurred with running multiple physical experiments to assess how bone cement will behave in a concrete given context (that is, bone morphology), \emph{physics-based simulations} are an attractive alternative both for professionals and, potentially even more importantly, for training practitioners. We outline below related work in fluid simulations which is relevant to our context.

\noindent\textbf{Smoothed Particle Hydrodynamics (SPH):} The SPH method~\cite{wang2024physics} uses particles to sample the fluid and calculate its physical properties during a simulation. SPH has high numerical accuracy and can capture details such as the splash of droplets on the fluid's free surface. Several SPH-based fluid solvers exist including Weakly compressible SPH (WCSPH)~\cite{becker2007weakly}, Predictive-Corrective Incompressible SPH (PCISPH)\,\cite{solenthaler2009predictive}, and Divergence-Free SPH (DFSPH)\,\cite{bender2015divergence}. Among them, DFSPH is arguably the most advanced fluid solver that can guarantee the incompressibility and divergence-free conditions of the fluid. 

\noindent\textbf{Multiphase Fluid Simulation:} 
Multiphase fluid simulation can simulate a mixture of several miscible or non-miscible components, such as is the case of bone cement. In more detail, multiphase simulations can handle different material components, like a mixture of water and sand (for cement) or different states of the same material, such as water and bubbles. Current multiphase flow simulation research covers bubble simulation\,\cite{mihalef2009simulation,ren2015simple}, phase transition\,\cite{tu2024unified}, and improving numerical accuracy\,\cite{ren2014multiple,jiang2021dynamic,xu2023implicitly}. Commonly used multiphase fluid simulation methods include two-fluid models\,\cite{ishii1990two}, suspension models\,\cite{wyart2014discontinuous}, and mixture models\,\cite{ren2014multiple}. Among them, the mixture model has received much attention recently. In the mixture model, the volume fraction is used to represent the proportion of different phases at the same sampling position. The overall discretization method uses SPH; the physical field of the fluid is calculated using multiphase fluid dynamics. For more details on SPH and multiphase simulation methods, we refer the interested reader to a recent survey\,\cite{wang2024physics}.

\section{Multiphase non-Newtonian fluid simulation for bone cement flow}
\label{sec:method}
%

% \steffen{in this section it should be highlighted what exactly we do differently than prior work (in other words, more explicitly link the discussion to the technical contributions promised at the end of the intro)}

The use of bone cement involves two processes: modulation and injection. During \emph{modulation}, two substances -- organic solvent and solute -- are mixed in different proportions, which  leads to varying flow performance during the \emph{injection} phase. The mixed fluid is a non-Newtonian fluid with viscoelastic shear thinning, and polymerization reactions will change its physical properties. The fluid polymerization period can be further split into four stages: coarse sand, drawing, clumping, and hardening. In this paper, we model the fluid based on an implicit mixture model and the polymer conformation tensor method, so as to simulate the dynamic modulation process and the coarse sand flow. The other stages remain as future work.

Previous work~\cite{xu2023implicitly} assumes that the
effect of the mixture on the phases is constant. However, this
is not true in some solutions that react. We extend this to the dynamic setting using a bonding effect network (Section~\ref{BB}) in our algorithm (Section\ref{CC}). Before doing that, we continue by describing the relevant basics of SPH methods (Section~\ref{AA}).

\subsection{SPH-based fluid simulation}
\label{AA}
%
The SPH method discretizes the continuous fluid medium into independent particles. SPH can be understood as a discretization method for spatial fields and spatial differential operations. The physical field information in space (e.g., fluid density, mass, velocity, pressure) is defined on SPH particles. SPH determines the state information of each particle at the next time step based on the contribution of neighbor particles, weighted by a kernel function (Fig.~\ref{fig:sph_particles}). Specifically, a physical field $\mathbf{A}_i$ (sampled at particle $i$) is estimated as
\begin{equation}
    \mathbf{A}_i = \sum _{j \in N(i)} \frac{m_j}{\rho _j}\mathbf{A}_jW_{ij},
\end{equation}
where $N(i)$ is the neighborhood of $i$ that affects that particle, $m$ denotes the particle mass, $\rho$ is the particle density, and $W$ is a Gaussian-like (\emph{e.g.}, cubic spline) kernel function.

\begin{figure}[t]
\centerline{\includegraphics[width=0.7\columnwidth]{pics/SPH particles.pdf}}
\caption{SPH particle samples. Here, $j_s$ represents the neighbors of particle $i$, and the curve denotes the Gaussian-like kernel function. When two particles become close, $W_{ij}$ will have a larger value, and when two particles are out of range, $W_{ij}$ will vanish.}
\label{fig:sph_particles}
\end{figure}

Estimating the gradient, divergence, and Laplacian of field \textbf{A} using the SPH standard discretization can be done as follows
\begin{equation}
    \begin{aligned} 
        \nabla \mathbf{A}_i &= \sum _{j \in N(i)} \frac{m_j}{\rho _j}\mathbf{A}_j \otimes \nabla W_{ij}, \\
        \nabla \cdot \mathbf{A}_i &= \sum _{j \in N(i)} \frac{m_j}{\rho _j}\mathbf{A}_j\nabla W_{ij},\\
        \nabla ^2 \mathbf{A}_i &= \sum _{j \in N(i)} \frac{m_j}{\rho _j}\mathbf{A}_j \nabla ^2 W_{ij}, \\
    \end{aligned}
\end{equation}
where $\mathbf{a} \otimes \mathbf{b} = \mathbf{a}\mathbf{b}^T$.

\subsection{Implicit mixture model with improved phase transfer}
\label{BB}
%
\noindent\textbf{Implicit Mixture Model:} The mixture model uses the volume fraction scheme (Fig.~\ref{fig:vol_frac}) to represent the concentration of each phase and calculates the physical parameters at the phase level and the mixture level using multiphase fluid dynamics.
%
\begin{figure}[htbp]
\centerline{\includegraphics[width=0.45\columnwidth]{pics/vol frac.pdf}}
\caption{Example of volume fraction scheme in a mixture model with a two-phase flow (phases are indicated by colors).}
\label{fig:vol_frac}
\end{figure}

The sum of the phase volume fractions $\alpha_k$, for all $k$ existing phases of a particle $i$, is normalized as
\begin{equation}
    \sum_k \alpha_{i,k} = 1.
\end{equation}
where subscript $i,k$ denotes phase $k$ of particle $i$.

The velocity field at the mixture level is reconstructed from the phase-level velocity fields $\mathbf{v}_k$ via
\begin{equation}
    \mathbf{v}_{i,\mathrm{mix}} = \sum_k \alpha _{i,k} \mathbf{v}_{i,k}.
\label{eq:mix_vel}
\end{equation}

The density of the mixture particle is computed as
\begin{equation}
    \rho_{i,\mathrm{mix}} = \sum_k \alpha_{i,k} \rho_k^0,
\label{eq:mix_dens}
\end{equation}
where $\rho_k^0$ is the rest density of phase $k$.

The velocity field at the mixture level $\mathbf{v}_\mathrm{mix}$ is used to represent the actual fluid motion, which depends on the calculation of the physical field at the phase level. Here, we use the implicit mixture model~\cite{xu2023implicitly}, where an implicit reconstruction method between mixture level and phase level was derived to achieve higher numerical accuracy. 
This model considers the effects of gravity, pressure, and viscous forces. The gravity $\mathbf{g}$ belongs to the volume force and is applied equally to each phase. Pressure and viscous forces are computed as
%
\begin{equation}
        \frac{D\mathbf{v}_{i,k}^p}{Dt} = \frac{\mathbf{M}_{i,\mathrm{mix}}^p}{\rho_{i,\mathrm{mix}}} \left(C_d+(1-C_d)\frac{\rho_{i,\mathrm{mix}}}{\rho_k^0}\right),
\label{eq:p_force}
\end{equation}
\begin{equation}
    \frac{D\mathbf{v}_{i,k}^\nu}{Dt} = C_d\frac{\mathbf{M}_{i,\mathrm{mix}}^\nu}{\rho_{i,\mathrm{mix}}} + (1-C_d)\frac{\mathbf{M}_{i,k}^\nu}{\alpha_{i,k} \rho_k^0},
\end{equation}
%
where $\frac{D\mathbf{v}_k}{Dt}$ denotes the acceleration associated with different forces; superscripts $p$ and $\nu$ denote pressure and viscosity, respectively; $\mathbf{M}$ represents the momentum source; and $C_d \in [0, 1]$ is the model parameter derived from the implicit mixture model, used to adjust the degree of influence of the mixture on each phase. 

\noindent\textbf{Improved Phase Transfer:} The phase transfer in the mixture model mainly involves two factors: interphase drag force and diffusion. The calculation of drag force depends on the drift velocity of the phase, which is defined as 
\begin{equation}
    \mathbf{v}_{i,k}^\mathrm{drift} = \mathbf{v}_{i,k} - \mathbf{v}_{i,\mathrm{mix}}.
\label{eq:drift_vel}
\end{equation}
%
The change of phase fraction due to the two factors is given by
\begin{equation}
    \begin{aligned} 
        \frac{D\alpha_{i,k}}{Dt} &= -\sum_{j\in N(i)} V_0(\alpha_{i,k}\mathbf{v}_{i,k}^\mathrm{drift}+\alpha_{j,k}\mathbf{v}_{j,k}^\mathrm{drift})\nabla \cdot W_{ij} \\
        \nabla ^2 \alpha_{i,k} &= C_f\sum_{j\in N(i)} (\alpha_{i,k}-\alpha_{j,k})\frac{\mathbf{x}_{ij}\cdot \nabla W_{ij}}{\|\mathbf{x}_{ij}\|^2 + \epsilon},
    \end{aligned}
\label{eq:phase_trans}
\end{equation}
%
where $C_f$ is the diffusion coefficient; $V_0$ denotes the rest volume of a particle; $\mathbf{x}_{ij}=\mathbf{x}_i-\mathbf{x}_j$; $\mathbf{x}$ is the position of the particle; and $\epsilon$ is a small regularization constant.

In the implicit mixture model, $C_d$ is used to adjust the degree of influence of the mixture on the phase: when $C_d=0$, the phase is completely unaffected by the mixture; when $C_d=1$, the phase is completely controlled by the mixture. In the physical field, the influence of the mixture on the phase is understood as the relationship between the velocity field of the mixture level and the phase level. When the phase velocity field completely follows the mixture, the phase does not separate, which affects phase transport. The original implicit mixture model~\cite{xu2023implicitly} sets $C_d$ as a constant, indicating that the effect of the mixture on the phase is constant. However, this is not true in some solutions that react. Hence, a mechanism is needed to compute the effect of the solute concentration on the phase transfer.

Calculating the exact intermolecular combination between two molecules can lead to a large overhead. To simplify this computation, we propose to use a \emph{bonding effect network} (see Fig.~\ref{fig:phase_trans}). Specifically, we change the $C_d$ value of the multiphase particle dynamically according to the solute concentration. For one mixture, a basic $C_d^0$ value is set; we next estimate  the current dynamic $C_d$ by the SPH method. We focus on two-phase fluids, where $\alpha_{k_1}$ denotes the liquid phase and  $\alpha_{k_2}$ denotes the polymer phase. As such, we have
%
\begin{equation}
    C_d = C_d^0 + (1-C_d^0)\sum_{j\neq i} V_0 \alpha_{j,k_2} W_{ij}.
\label{eq:update_Cd}
\end{equation}
%
When the solute concentration around a particle is high, $C_d \rightarrow 1$. This will block the phase transfer to simulate the case of phase coupling.

\begin{figure}[t]
\centerline{\includegraphics[width=0.85\columnwidth]{pics/phase trans.pdf}}
\caption{Our bonding effect network used to model the coupling of phases in an inhomogeneous solution where reactions occur. Using the regions of three concentrations as an example, a greener particle color indicates a higher polymer phase fraction. The arrows indicate the blocking effect of the neighbor on the phase transfer of the target particle $i$. Thicker arrows indicate a stronger blocking effect.}
\label{fig:phase_trans}
\end{figure}

\begin{figure}[t]
\centerline{\includegraphics[width=0.9\columnwidth]{pics/cap curve.pdf}}
\caption{The shear thinning curve of our model. The viscoelasticity of shear thinning fluid decreases as the shear rate increases. In our model, the larger the value of $\gamma$, the stronger the shear thinning effect.}
\label{fig:cap_curve}
\end{figure}

\subsection{Polymer conformation tensor method in the mixture model}
\label{CC}
%
The conformation tensor is a tool for describing the material distribution in solutions~\cite{bejan2005constructal}, both for Newtonian and non-Newtonian fluids. The classical configuration update formula for the conformation tensor $\mathbf{U}$ (a $3\times3$ matrix in the 3D case) is
%
\begin{equation}
    \frac{D\mathbf{U}}{Dt} = \mathbf{U}\nabla \mathbf{v}+(\nabla \mathbf{v})^T\mathbf{U}-\frac{1}{\lambda}(\mathbf{U}-\mathbf{I}),
\label{eq:update_U}
\end{equation}
%
where $\lambda$ denotes the relaxation time used to describe the fluid viscoelasticity. However, this model can only describe Newtonian fluids; the viscoelasticity of non-Newtonian fluids has a nonlinear relationship with the shear rate. 
To model this, we use an alternative model given by
\begin{equation}
    \frac{D\mathbf{U}}{Dt} = \mathbf{U}\nabla \mathbf{v}+(\nabla \mathbf{v})^T\mathbf{U}-\frac{1}{\lambda}(\mathbf{U}-\mathbf{I}) - \gamma(\mathbf{U}-\mathbf{I})\mathbf{U},
\end{equation}
%
where $\gamma(\mathbf{U}-\mathbf{I})\mathbf{U}$ is a non-linear term that can model shear thinning; and $\gamma \in [0,1]$ is the thinning factor. A larger value of $\gamma$ contributes to a stronger thinning effect (Fig.~\ref{fig:cap_curve}).

The mixture-level stress based on the conformation tensor is defined as
%
\begin{equation}
    \mathbf{\tau}_{i,\mathrm{mix}} = c\eta_s(\mathbf{U}_i-\mathbf{I}),
\label{eq:stress}
\end{equation}
%
where $c$ denotes the polymer concentration of the solution, which is equal to $\alpha _{k_2}$ in our multiphase framework; and $\eta _s$ denotes the viscosity of the solution. We use the  symmetric formulation of SPH~\cite{monaghan2005smoothed} to calculate the stress force as
%
\begin{equation}
    \frac{1}{\rho_{i,\mathrm{mix}}}\nabla \cdot \mathbf{\tau}_{i,\mathrm{mix}} = \sum _{j\in N(i)} \left( \frac{\mathbf{\tau}_{i,\mathrm{mix}}}{\rho^2_{i,\mathrm{mix}}} + \frac{\mathbf{\tau}_{j,\mathrm{mix}}}{\rho^2_{j,\mathrm{mix}}}\right)\nabla W_{ij}. 
\label{eq:stress_force}
\end{equation}

The conformation tensor method does not define the stress calculation for each phase, and the multiphase framework requires reconstructing the mixture-level velocity from the phase-level velocity. Based on the normalization condition $\mathbf{\tau}_\mathrm{mix} = \sum_k \mathbf{\tau}_ k$, we obtain 
%
\begin{equation}
    \mathbf{\tau}_{i,k} = \alpha_{i,k} \mathbf{\tau}_{i,\mathrm{mix}}.
\end{equation}
%
In this way, the phase velocity can be updated according to the mixture-level stress as
\begin{equation}
    \frac{D\mathbf{v}_{i,k}^\mathrm{visc}}{Dt}=\frac{\nabla \cdot \tau_{i,k}}{\alpha_{i,k} \rho_ k^0},
\label{eq:visc_force}
\end{equation}
%
where $\frac{D\mathbf{v}_k^\mathrm{visc}}{Dt}$ represents the acceleration associated with the viscoelastic force.

Algorithm~\ref{algo:algo} outlines our end-to-end simulation and visualization method called IMM-CT (implicit mixture model with conformation tensor). At the start, some preparatory work is required, including bone modeling, setting the scene and model parameters (step a). When initializing the model parameters, $\mathbf{U}$ is set to $\mathbf{I}$, and  a uniform grid is used to update the neighbors $N(i)$ of each particle $i$. 
Then, step b solves the dynamic parameters of the multiphase fluid according to the initial input, including pressure, gravity, viscoelastic forces and implements phase transport. Finally the fluid surface is reconstructed using 3D utilities and rendered.

\begin{algorithm}[t]
\small
\caption{Bone cement simulation algorithm.}
\begin{algorithmic}[l]

\State \textbf{a) Prepare:}
    \State \hspace{2mm} 1. Modeling bone with different poriness
    
    \State \hspace{2mm} 2. Set injection direction, speed and position

    \State \hspace{2mm} 3. Import the particle models and initialize the solver parameters: before the solver
    loop step, set $\mathbf{U}_i \gets \mathbf{I}$ for all particle $i$ and
    initialize the neighbors $N(i)$

\State \textbf{b) Bone cement modeling and motion solution:}
    \State \hspace{2mm} \textbf{1. Pressure computation}
        \State \hspace{4mm} compute div-free force $\mathbf{M}_{i,\mathrm{mix}}^{p}$ using \textbf{VFSPH} 
        \State \hspace{4mm} update phase velocity $\mathbf{v}_{i,k}$ \Comment{Eq.~\ref{eq:p_force}}
        \State \hspace{4mm} update mixture velocity $\mathbf{v}_{i,\mathrm{mix}}$ \Comment{Eq.~\ref{eq:mix_vel}}

    \State \hspace{2mm} \textbf{2. Advect}
        \State \hspace{4mm} update phase velocity using $\mathbf{v}_{i,k} \gets \mathbf{v}_{i,k} + \mathbf{g}\Delta t$
        \State \hspace{4mm} update mixture velocity $\mathbf{v}_{i,\mathrm{mix}}$ \Comment{Eq.~\ref{eq:mix_vel}}

    \State \hspace{2mm} \textbf{3. Viscoelasticity computation}
        \State \hspace{4mm} update ${C_d}_i$ \Comment{Eq.~\ref{eq:update_Cd}}
        \State \hspace{4mm} update conformation tensor $\mathbf{U}_i$ \Comment{Eq.~\ref{eq:update_U}}
        \State \hspace{4mm} compute viscoelastic force \Comment{Eq.~\ref{eq:stress},~\ref{eq:stress_force}}
        \State \hspace{4mm} update phase velocity $\mathbf{v}_{i,k}$ \Comment{Eq.~\ref{eq:visc_force}}
        \State \hspace{4mm} update mixture velocity $\mathbf{v}_{i,\mathrm{mix}}$ \Comment{Eq.~\ref{eq:mix_vel}}

    \State \hspace{2mm} \textbf{4. Final step }
        \State \hspace{4mm} update phase volume fraction $\alpha_{i,k}$ \Comment{Eq.~\ref{eq:phase_trans}}
        \State \hspace{4mm} update phase drift velocity $\mathbf{v}^\mathrm{drift}_{i,k}$ \Comment{Eq.~\ref{eq:drift_vel}}
        \State \hspace{4mm} update particle position using $\mathbf{x}_i \gets \mathbf{x}_i +\mathbf{v}_{i,\mathrm{mix}}\Delta t$
        \State \hspace{4mm} update neighbors $N(i)$
        \State \hspace{4mm} store the point cloud $\mathbf{x}_i$ according to the output frame rate
\State --End Sim Loop--   
\State \textbf{c) Render:} From the simulated point clouds $\{\mathbf{x}_i\}$, we finally reconstruct and render the fluid surface using e.g. Houdini.

\end{algorithmic}
\label{algo:algo}
\end{algorithm}

\section{Experiments}
\label{sec:experiments}
%
We tested our method by simulating various modulation and mixing/injection processes of bone cement. In the following experiments, we refer to the parameters of our method given in Tab.~\ref{tab:params}. We designed two sets of experiments to verify the advantages of our model, as described next.

\begin{figure}[t]
\centering
 \includegraphics[width=0.6\columnwidth]{pics/model_1.pdf}
\caption{Modulation mixing scene. The two cylinders in the container are the two phases to be mixed by the high-speed rotating fan below them.}
\label{fig:model_1}
\end{figure}

\begin{table}[b]
\centering
\caption{Model parameters}
\begin{tabular}{lll}
\hline
Parameter & Meaning & Range\\
\hline
$C_f$ & diffusion coefficient & [0, 1]\\
$C^0_d$ & rest drag coefficient &  [0, 1]\\
$\gamma$ & phase volume fraction threshold & [0, 1]\\
$\eta_s$ & rest viscosity of solution & (0, 15)\\
$\lambda$ & relaxation time & (0, 1)\\
\hline
\end{tabular}
\label{tab:params}
\end{table}

\noindent\textbf{Modulation mixing:} We set up three sets of mixing scenarios with different viscosity ratios. The base viscosity (i.e., solvent viscosity) was set to $0.01$ $Pa\cdot s$. The experiment details the stability and performance of the mixing process of different schemes under different viscosity ratios of substances. Figure~\ref{fig:model_1} shows the setup of the experiment. Figure~\ref{fig:mix} visually compares results obtained by our method (IMM-CT) with two other schemes (DFSPH and IMM). Figure~\ref{fig:perf_curve} shows the maximum acceptable time step versus viscosity ratio for the tested schemes. It reveals that our IMM-CT scheme has advantages in both stability and performance.

\begin{figure*}[t]
\centerline{\includegraphics[width=\textwidth]{pics/compare.pdf}}
\caption{Three schemes (DFSPH, IMM, and IMM-CT) are used to mix fluids with three different viscosity ratios to compare stability and performance. In the 1:10 scenario, DFSPH shows instability when the fluid is stirred at high speed. In the 1:100 scenario, all the three methods can run stably, but DFSPH  cannot obtain a uniform mixed state. In the 1:1000 scenario, DFSPH still can't evenly mix and IMM fails to run stably with an acceptable time step, the simulation collapsed at high speed of stirring. Our method (IMM-CT) runs stably in all three scenarios.}
\label{fig:mix}
\end{figure*}

\begin{figure}[t]
\centerline{\includegraphics[width=0.95\columnwidth]{pics/perf curve.pdf}}
\caption{Comparison of the maximum acceptable time step for  three schemes at different viscosity ratios. We define the viscosity ratio as the solute viscosity over the solvent viscosity, and the unit of viscosity is $Pa\cdot s$. Our method (IMM-CT) maintains the advantage in all cases. For viscosity ratios below $0.001$, both IMM and DFSPH  fail to stabilize the simulation.}
\label{fig:perf_curve}
\end{figure}

\begin{figure}[t]
\centerline{\includegraphics[width=0.7\columnwidth]{pics/model_2.pdf}}
\caption{Bone cement injection scene. The long silver tube is an injecting syringe. The bone interior is modeled as a porous structure with different regions having different sparsities. 
Darker orange regions in the figure are the sparser regions, which are used to simulate osteoporosis.}
\label{fig:model_2}
\end{figure}


\noindent\textbf{Injection:} We chose vertebral fillings, which are common in surgery, as a demonstration. We model the bone interior as a porous material with different local porosities (Fig.~\ref{fig:model_2}). We use Blender to model the porous structure, first endow the Mesh object with volume and adjust the local density, and then convert the volume into mesh. We next injected different concentrations and dosages of bone cement into the bone to observe the filling process. This experiment aims to reflect the surgical results caused by different proportions and dosages as shown next in Fig.~\ref{fig:inject}. Different modulation ratios lead to different fluidity of the cement. In the scenario of $0.8:0.2$ (first two rows in Fig.~\ref{fig:inject}), the purpose of filling can be quickly achieved, but it is also easy to overflow. The ratio of $0.45:0.55$ (second two rows in Fig.~\ref{fig:inject}) shows good performance, and the ratio of $0.15:0.85$ leads (third two rows in Fig.~\ref{fig:inject}) to too viscous fluid, so that the fluidity is insufficient, and the filling cannot be completed, so there is no target image.

\begin{figure*}[t]
\centerline{\includegraphics[width=\textwidth]{pics/inject.pdf}}
\caption{Three scenarios of bone cement injection with different proportions. The volume ratio of solvent to solute is shown on the left side of the figure. Each row shows a few frames from the injection process; each column represents the same timestamp. Gray boxes around the images indicate insufficient dispersion; green boxes indicate appropriate dispersion; red boxes indicate that overflow occurs. The right column shows the top view at the appropriate injection volume.}
\label{fig:inject}
\end{figure*}

It is worth noting that in our work the polymer phase has a viscosity of 8 $Pa\cdot s$. In practice, viscosity can be set according to the properties of the products, so different mixing proportion will follow. In addition to the effect of the modulation ratio on the dispersion of the fluid, the filling situation is also affected by the injection speed and injection direction, among other factors. Although we have not given the corresponding demonstration, these experiments are completely feasible. Due to the reason of rendering, the filling situation inside the bone cement may not be fully presented, resulting in visual artifacts of the area vacancy on the image (Fig.~\ref{fig:inject}). We provided renders from two perspectives for each set of experiments to compensate for the missing filling area caused by rendering.

% \steffen{in the abstract, for example, we claim that we are able to "obtain and dynamic bone cement effects with high accuracy"; it did not become clear to me how we evaluate accuracy here}

\section{Conclusion}
\label{sec:Lconclusion}
%
We have presented a multiphase non-Newtonian fluid simulation method based on the conception tensor method, capable of covering simulations of material mixing with a wide range of viscosity ratios at acceptable time steps, and combined with porous media modeling. Our method targets simulating the full flow of bone cement injection efficiently and effectively, thereby helping practitioners to experiment with different simulation parameters in a cost-effective way. Compared with other existing methods, our model has advantages in performance and stability.

Future work includes extending the model to support solidification and other stages in the flow phase of bone cement. An equally interesting direction is to incorporate the treatment of tension and capillary forces which plain an important role in real materials.
\section{Acknowledgements}
\label{sec:Acknowledgements}
This research was supported by the National Natural Science Foundation of China (No. 62306032), the National Key Research and Development Program of China (No. 2022ZD0118001), the Guangdong Basic and Applied Basic Research Foundation (No. 2022A1515110350), and the Interdisciplinary Research Project for Young Teachers of USTB (No. FRF-IDRY-22-025). The computing work is partly supported by the MAGICOM Platform of Beijing Advanced Innovation Center for Materials Genome Engineering.

\bibliographystyle{IEEEtran} 
\bibliography{main}


\end{document}
