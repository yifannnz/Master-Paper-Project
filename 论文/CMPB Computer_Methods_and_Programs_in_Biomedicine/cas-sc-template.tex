%% 
%% Copyright 2019-2024 Elsevier Ltd
%% 
%% This file is part of the 'CAS Bundle'.
%% --------------------------------------
%% 
%% It may be distributed under the conditions of the LaTeX Project Public
%% License, either version 1.3c of this license or (at your option) any
%% later version.  The latest version of this license is in
%%    http://www.latex-project.org/lppl.txt
%% and version 1.3c or later is part of all distributions of LaTeX
%% version 1999/12/01 or later.
%% 
%% The list of all files belonging to the 'CAS Bundle' is
%% given in the file `manifest.txt'.
%% 
%% Template article for cas-sc documentclass for 
%% double column output.

\documentclass[a4paper,fleqn]{cas-sc}

% If the frontmatter runs over more than one page
% use the longmktitle option.

%\documentclass[a4paper,fleqn,longmktitle]{cas-sc}

%\usepackage[numbers]{natbib}
%\usepackage[authoryear]{natbib}
\usepackage[authoryear,longnamesfirst]{natbib}
\usepackage{ulem}
\usepackage{subfigure}
\usepackage{multirow}

%%%Author macros
\def\tsc#1{\csdef{#1}{\textsc{\lowercase{#1}}\xspace}}
\tsc{WGM}
\tsc{QE}
%%%

% Uncomment and use as if needed
%\newtheorem{theorem}{Theorem}
%\newtheorem{lemma}[theorem]{Lemma}
%\newdefinition{rmk}{Remark}
%\newproof{pf}{Proof}
%\newproof{pot}{Proof of Theorem \ref{thm}}

\begin{document}
\let\WriteBookmarks\relax
\def\floatpagepagefraction{1}
\def\textpagefraction{.001}

% Short title
\shorttitle{}    

% Short author
\shortauthors{}  

% Main title of the paper
\title [mode = title]{Bone Cement Flow Simulation: A Two-Phase Non-Newtonian Fluid Approach for Surgical Optimization}  

% Title footnote mark
% eg: \tnotemark[1]
%\tnotemark[1] 

% Title footnote 1.
% eg: \tnotetext[1]{Title footnote text}
%\tnotetext[1]{} 

% First author
%
% Options: Use if required
% eg: \author[1,3]{Author Name}[type=editor,
%       style=chinese,
%       auid=000,
%       bioid=1,
%       prefix=Sir,
%       orcid=0000-0000-0000-0000,
%       facebook=<facebook id>,
%       twitter=<twitter id>,
%       linkedin=<linkedin id>,
%       gplus=<gplus id>]

\author[1,2]{Yalan Zhang}%[<options>]

% Corresponding author indication
%\cormark[1]

% Footnote of the first author
%\fnmark[1]

% Email id of the first author
\ead{zhangyl@ustb.edu.cn}

% URL of the first author
%\ead[url]{}

% Credit authorship
% eg: \credit{Conceptualization of this study, Methodology, Software}
\credit{}

% Address/affiliation
\affiliation[1]{organization={School of Intelligence Science and Technology, University of Science and Technology Beijing},
%             addressline={}, 
             city={Beijing},
%          citysep={}, % Uncomment if no comma needed between city and postcode
             postcode={100083}, 
%             state={},
            country={China}}
\affiliation[2]{organization={Shunde Innovation School, University of Science and Technology Beijing},
%            addressline={}, 
            city={Foshan},
%          citysep={}, % Uncomment if no comma needed between city and postcode
            postcode={528300}, 
%            state={},
            country={China}}
            
\author[1]{Long Shen}%[]

% Footnote of the second author
%\fnmark[2]

% Email id of the second author
\ead{sl_111211@163.com}

% URL of the second author
%\ead[url]{}

% Credit authorship
\credit{}

\author[1]{Xiaokun Wang}%[]

% Footnote of the second author
%\fnmark[2]

% Email id of the second author
\ead{wangxiaokun@ustb.edu.cn}

% URL of the second author
%\ead[url]{}

% Credit authorship
\credit{}

\author[3]{Steffen Frey}%[]

% Footnote of the second author
%\fnmark[2]

% Email id of the second author
\ead{s.d.frey@rug.nl}

% URL of the second author
%\ead[url]{}

% Credit authorship
\credit{}

\affiliation[3]{organization={Bernoulli Institute, University of Groningen},
%            addressline={}, 
            city={Groningen},
%          citysep={}, % Uncomment if no comma needed between city and postcode
            postcode={9747AG}, 
%            state={},
            country={the Netherlands}}
            
\author[4]{Alexandru C Telea}%[]

% Footnote of the second author
%\fnmark[2]

% Email id of the second author
\ead{a.c.telea@uu.nl}

% URL of the second author
%\ead[url]{}

% Credit authorship
\credit{}
\affiliation[4]{organization={Department of Information and Computing Sciences, Utrecht University},
%            addressline={}, 
            city={Utrecht},
%          citysep={}, % Uncomment if no comma needed between city and postcode
            postcode={3584CC}, 
%            state={},
            country={the Netherlands}}
            
\author[3]{Ji\v{r}\'i Kosinka}%[]

% Footnote of the second author
%\fnmark[2]

% Email id of the second author
\ead{j.kosinka@rug.nl}

% URL of the second author
%\ead[url]{}

% Credit authorship
\credit{}

\author[1]{Xiaojuan Ban}%[]

% Footnote of the second author
%\fnmark[2]

% Corresponding author indication
\cormark[1]

% Email id of the second author
\ead{banxj@ustb.edu.cn}

% URL of the second author
%\ead[url]{}

% Credit authorship
\credit{}



% Address/affiliation


% Corresponding author text
\cortext[1]{Corresponding author}

% Footnote text
%\fntext[1]{}

% For a title note without a number/mark
%\nonumnote{}

% Here goes the abstract
\begin{abstract}
Bone cement injection is a key technique for the prevention of osteoporosis and the treatment of fractures. In such procedures,  preparing and dosing the cement typically follow specific product guidelines and the clinician's experience; however, improper use of bone cement can result in various complications. Computer simulation technologies offer an effective solution to this problem by enabling physicians to design surgical plans at low cost, including determining the cement mixing ratios, injection dosage, and injection angles. Different simulation scenarios allow clinicians to observe potential cement flow patterns, thereby selecting the optimal surgical approach to mitigate procedural risks. This paper supports the above by proposing a multiphase non-Newtonian fluid simulation method to model and visualize the flow behavior during the wet sand phase of bone cement preparation and mixing. Our method provides an intuitive representation of the bone cement application process under various procedural conditions and simulates stably and with high computational performance the coupling of bone boundaries to cement with varying viscoelastic properties. To the best of our knowledge, no comparable cross-disciplinary technical study exists. Compared to existing multiphase fluid simulation techniques, our method simulates highly viscous multiphase fluids more efficiently and stably, making it well-suited for aforementioned training and experimental scenarios. 
\end{abstract}

% Use if graphical abstract is present
\begin{graphicalabstract}
\centerline{
\includegraphics[width=0.95\textwidth]{pics/ga.png}}
\end{graphicalabstract}


% Research highlights
\begin{highlights}
    \item A visualization framework for simulating the entire process of bone cement surgery
    \item A new physical simulation approach for two-phase non-Newtonian fluids
    \item An advanced bone modeling technique to enhance the authenticity of the simulation
\end{highlights}


% Keywords
% Each keyword is seperated by \sep
\begin{keywords}
 \sep Medical visualization \sep Bone filling simulation \sep Multiphase non-Newtonian fluid modeling \sep Bone cement effects
\end{keywords}

\maketitle

% Main text
\section{Introduction}
\label{intro}
%
With the extension of human lifespan and the increasing prevalence of an aging population, approximately 200 million individuals worldwide are affected by osteoporosis, fractures, and other orthopedic conditions that may arise unexpectedly. The management of osteoporosis has emerged as a significant medical challenge. Currently, bone cement filling serves as an early intervention for osteoporosis and represents a key therapeutic strategy for addressing bone fractures~\citep{vaishya2013bone,saha1984mechanical}. With advancements in minimally invasive surgical techniques, \textit{vertebral augmentation} has become the predominant procedure for treating osteoporotic compression fractures -- bone cement is injected into the compromised vertebrae as a filler to restore structural integrity (see Fig.~\ref{fig:cement_filling}). However, adherence to stringent operational guidelines during the application of bone cement is essential to mitigate the risk of postoperative complications. Improper handling may result in cement leakage which can lead to severe consequences such as pulmonary embolism and paraplegia~\citep{breusch2005pulmonary,moussazadeh2015short}.

\begin{figure}[t]
\centerline{\includegraphics[width=0.75\columnwidth]{pics/cement filling.png}}
\caption{Structural Support in Vertebral Augmentation.
a)~The normal height of an adult vertebra is approximately 2.2 cm, with a dense internal bone structure. b)~In cases of osteoporosis, the bone tissue within the vertebra becomes sparse, resulting in compression and deformation of the vertebra under vertical load, which can lead to nerve compression. c)~Vertebral augmentation with bone cement (green) involves the injection of cement into the vertebra's porous regions to restore its structural integrity.}
\label{fig:cement_filling}
\end{figure}

In current clinical practice, surgeons typically rely on their experience and the instructions provided by bone cement manufacturers to carefully mix the cement, aiming to prevent errors in preparation that could lead to cement leakage or insufficient dispersion (see Fig.~\ref{fig:cases}).  For new surgeons, training in bone cement \textit{preparation} is both time-consuming and costly. Separately, during \textit{injection}, the insertion angle and direction of the syringe can result in varying filling outcomes. To assist surgeons in accurately preparing and injecting bone cement and selecting an optimal surgical approach, we propose the use of computer simulation and visualization techniques. This approach enables efficient, cost-effective, and repeatable evaluation of the feasibility of different surgical plans.

\begin{figure}[htbp]
\centering
\includegraphics[width=0.7\columnwidth]{pics/cases.png}
\caption{Several Cases of Bone Cement Filling. a) A schematic illustration showing bone cement (green) filling within the bone (gray). b) In the presence of fractures (indicated by black lines), insufficient cement thickness may lead to leakage through the fractures due to increased intraosseous pressure and a low modulation ratio. c) Excessive cement thickness can cause obstruction, preventing the complete filling of the bone. d) The diffusion of bone cement within the bone poses a risk of extending toward adjacent nerves (purple), potentially resulting in complications.}
\label{fig:cases}
\end{figure}

Our approach consists of three main components: bone modeling, bone cement fluid modeling, and rendering. In our recent preliminary discussion on \cite{shen2024visual}, we did an initial investigation into multiphase non-Newtonian fluid modeling and the general application flow. However, the techniques applied in bone modeling is not accurate and scene setup is also incorrect, for example, the position that the screw is inserted into the bone will cause incorrect flow of bone cement. We have made further improvements to address the aforementioned issues, the novel contributions and main modifications on top of \cite{shen2024visual} are:
%
\begin{itemize}
\item A more detailed and updated bone modeling approach;
\item A more standardized application process;
\item Modeling and rendering color schemes based on real medical materials;
\item Scene settings that better align with actual surgical environments.
\end{itemize}

The structure of this paper is as follows. Section~\ref{sec:background} reviews related work on bone cement covering the historical development of bone cement materials and associated fluid simulation algorithms within the realm of computer graphics. Section~\ref{sec:method} details the theoretical framework underpinning our simulation approach and its individual components. Section~\ref{sec:experiments} presents comparative experiments and their results that highlight our method. Finally, Section~\ref{sec:Lconclusion} concludes the paper.

\section{Related Work}
\label{sec:background}
%
\subsection{Bone cement materials}
\label{sec:bone_cement}
%
Bone cement filling is an important method for orthopedic joint replacement and the treatment of osteoporosis. In the 1960s, \cite{charnley1960anchorage} was the first to apply polymethyl methacrylate (PMMA) to the fixation of femoral prostheses and acetabulum. Polymethyl methacrylate (PMMA) cement is currently the most commonly used bone cement material. PMMA is an acrylic polymer composed of a powder (primarily polymethyl methacrylate particles) and a liquid (methyl methacrylate monomer) which undergoes a polymerization reaction upon mixing. PMMA has good biocompatibility, which can reduce irritation to human tissues, and it also allows for chemical modification through copolymerization or blending with other materials, enabling adjustments to its physical and chemical properties.

Calcium phosphate cement (CPC, \cite{ambard2006calcium}) is also a widely utilized biomaterial in the medical domain. CPC is synthesized through the reaction of calcium phosphate powder with a liquid phase, resulting in a hardened structure predominantly composed of hydrated calcium phosphate, with hydroxyapatite (HAP) as its principal component. Owing to its compositional resemblance to the mineral phase of bone tissue, CPC has exceptional biocompatibility and bioactivity. In vivo studies indicate that CPC can be gradually resorbed and ultimately substituted by newly formed bone tissue, thereby facilitating osteogenesis and enabling direct integration with native bone structures. Recently, CPC has found extensive application in various bone regeneration contexts, \textit{e.g.} maxillofacial defect repair~\citep{tan2021biomaterial} and the development of innovative composite materials~\citep{jeong2019bioactive}.

Other composite materials have also been proposed more recently, such as bioactive ceramic composite bone cement~\citep{samad2011new} and nanomaterial composite bone cement~\citep{no2014nanomaterials}. These materials typically use conventional bone cement as a base and incorporate specific substances to achieve desired antibacterial properties or enhanced mechanical characteristics.

\subsection{Computer simulation methods}
\label{sec:simulation_methods}
%
Given the expenses incurred with running multiple physical experiments to assess how bone cement will behave in a concrete given setting (injection angle, rate of injection, exact properties of the cement mix, \textit{etc.}), \textit{physics-based simulations} are an attractive alternative both for professionals and, potentially even more importantly, for training practitioners. We outline below related work in fluid simulation which is relevant to our context.

Fluid simulation has been a main research focus in computer graphics. To date, several simulation algorithms targeted at different fluid types have been proposed. Fluid simulation research mainly focuses on aspects such as pressure solvers, viscosity solvers, boundary handling, and detail enhancement. Various efficient solvers have been developed for different simulation methods. Lagrangian simulation is a particle-based approach, with common discretization methods including Smoothed Particle Hydrodynamics (SPH, \cite{wang2024physics}), and Position-Based Dynamics (PBD, \cite{macklin2016xpbd}). Within such frameworks, Divergence-free SPH (DFSPH, \cite{bender2015divergence}) and Position-Based Fluids (PBF, \citep{macklin2013position}) provide stable and efficient pressure solutions. Eulerian simulation, on the other hand, is a grid-based method, with typical fluid simulation frameworks including the Lattice Boltzmann Method (LBM, \cite{ma2024hybrid}) and Finite Volume Method (FVM,  \cite{liu2018adaptive}). In terms of numerical accuracy, grid-based methods generally outperform particle-based methods. However, particle-based methods offer better performance in terms of fluid surface detail representation and parallelization of the computations.

Current research on fluid simulation primarily focuses on the simulation of \textit{single-phase} Newtonian fluids, with state-of-the-art solvers achieving high numerical accuracy and visually appealing animation results ~\citep{panuelos2023polystokes,liu2021adapted,su2021unified}. In contrast, studies on more complex \textit{multiphase} fluids are relatively limited and tend to concentrate on multiphase fluid frameworks~\citep{ren2014multiple, yang2015fast}, bubble simulation~\citep{li2022efficient}, and phase transition phenomena ~\citep{yan2016multiphase}. Simulation techniques for multiphase non-Newtonian fluids remain significantly underdeveloped. 

Non-Newtonian fluids are a class of fluids whose viscosity exhibits a nonlinear relationship with shear rate, with common types including shear-thinning and shear-thickening fluids. In the context of bone cement, it is a shear-thinning non-Newtonian fluid formed by the mixture of various substances. Therefore, simulation techniques for multiphase non-Newtonian fluids are crucial for this scenario.

% Numbered list
% Use the style of numbering in square brackets.
% If nothing is used, default style will be taken.
%\begin{enumerate}[a)]
%\item 
%\item 
%\item 
%\end{enumerate}  

% Unnumbered list
%\begin{itemize}
%\item 
%\item 
%\item 
%\end{itemize}  

% Description list
%\begin{description}
%\item[]
%\item[] 
%\item[] 
%\end{description}  

\section{Simulation Method}
\label{sec:method}
%
The application of bone cement encompasses two primary processes: modulation and injection. During the \textit{modulation} stage, an organic solvent and solute are combined in varying proportions, resulting in distinct flow characteristics during the subsequent \textit{injection} stage. The resultant mixture behaves as a non-Newtonian fluid exhibiting viscoelastic shear-thinning properties, with polymerization reactions inducing alterations in its physical attributes. The period of fluid polymerization can be delineated into four distinct stages: coarse sand formation, drawing, clumping, and hardening. In this study, we employ an implicit mixture model alongside the polymer conformation tensor method to simulate both the dynamic modulation process and coarse sand flow; further exploration of the remaining stages is reserved for future research.

Previous research posits that the influence of the mixture on the phases remains constant~\citep{xu2023implicitly}. However, this assumption does not hold for certain reactive solutions. We extend this concept to a dynamic framework by incorporating a bonding effect network (Sec.~\ref{BB}) within our algorithm (Sec.~\ref{CC}). Prior to this implementation, we outline the fundamental principles of bone modeling (Sec.~\ref{00}) and Smoothed Particle Hydrodynamics (SPH) methods (Sec.~\ref{AA}).

\subsection{Bone modeling }
\label{00}
%
Considering that the structure of vertebrae can be divided into an outer cortical bone and an inner cancellous bone, we separately model these two components and eventually merge them into a unified model. The cortical bone has a relatively uniform volumetric density, whereas the cancellous bone is essentially a porous material. Therefore, to properly simulate cement injection, 
we need to model this porosity. Separately, since our fluid solver is based on a particle method, we need to convert the bone model into a particle-based representation for fluid-solid coupling interactions.

% \sout{Initially, we assign volume to the mesh model and adjust the voxel resolution to an appropriate level. To model the porous nature of the cancellous bone, we introduce noise to alter the local density within the volume. For the cortical bone, we assign a uniform volumetric density. Subsequently, the voxel model, with introduced perturbations, is converted back into a mesh model, followed by surface particle sampling to obtain a particle-based representation.}

In our process, we consider three types of 3D models: voxel models, mesh grids, and particle-based models. The voxel model is a 3D geometric representation based on volumetric elements, capable of representing the volumetric structure of an object. The mesh grid model is a 3D surface model formed by stitching together polygonal faces, used to describe the external surface of an object. The particle-based model is a 3D model composed of point clouds, obtained through volume sampling and surface sampling of the voxel and mesh grid models.

First, Voronoi textures were employed to generate cancellous bone voxel models with varying bone densities, as the Voronoi method has been demonstrated to provide excellent biomechanical properties and fluid transport capabilities \citep{li2024integrated}. Subsequently, the marching cubes algorithm was applied to reconstruct a smooth surface mesh from the voxel data \citep{newman2006survey}. Next, Poisson Disk Sampling was utilized to achieve uniform particle distribution on the surface of the mesh \citep{bridson2007fast}. During this process, a minimum distance constraint between particles was imposed to ensure uniformity. To maintain consistency with the fluid solver, the minimum distance was set equal to the diameter of the fluid particles. Finally, the model was imported into the fluid solver for further analysis. A graphical representation of this workflow is shown in Fig. \ref{fig:modeling}.

\begin{figure}[htbp]
\centering
\includegraphics[width=\columnwidth]{pics/bone modeling.png}
\caption{Bone Modeling. %\sout{We discretized the mesh model into voxels and modified the local volume density by introducing noise to create a porous material analogous to cancellous bone. Subsequently, we conducted surface particle sampling of the generated mesh model and imported it into our fluid solver.} 
a) illustrates the workflow for creating a trabecular bone particle model from the basic voxel model. First, Voronoi textures are used to generate an irregular porous structure in the voxel model. By adjusting parameters such as the distribution of Voronoi seeds, bone structures resembling both healthy and osteoporotic bone can be obtained. The voxel model is then converted into a mesh model using the marching cubes technique, and finally, Poisson Disk Sampling is applied to uniformly sample particles on the surface. b) demonstrates a simplified process of trabecular bone shaping and the cortical bone particle model.}
\label{fig:modeling}
\end{figure}

\subsection{SPH-based fluid simulation}
\label{AA}
%
The SPH method discretizes the continuous fluid medium into independent particles. SPH can be understood as a discretization method for spatial fields and spatial differential operations. The physical field information in space (e.g., fluid density, mass, velocity, pressure) is defined on SPH particles. SPH determines the state information of each particle at the next time step based on the contribution of neighbor particles (at the current time step) weighted by a kernel function (Fig.~\ref{fig:sph_particles}). Specifically, a physical field $\mathbf{A}_i$ (sampled at particle $i$) is estimated as
%
\begin{equation}
    \mathbf{A}_i = \sum _{j \in N(i)} \frac{m_j}{\rho _j}\mathbf{A}_jW_{ij},
\end{equation}
%
where $N(i)$ is the neighborhood of particle $i$ that affects that particle, $m$ denotes particle mass, $\rho$ is the particle density, and $W$ is a Gaussian-like (\textit{e.g.}, cubic spline) kernel function centered at particle $i$ and evaluated at particle $j$.

\begin{figure}[htbp]
\centering
\includegraphics[width=0.4\columnwidth]{pics/SPH particles.png}
\caption{Smoothed Particle Hydrodynamics (SPH) Particle Sampling. The target particle $i$ is shown in orange. SPH utilizes a Gaussian-like kernel function $W_{ij}$ to compute the influence of surrounding particles $j$ on the target particle. When two particles are in close proximity, $W_{ij}$ takes on a larger value, if they are outside the interaction range, $W_{ij}$ diminishes to zero. Only neighboring particles within the SPH support radius ($j_1$ to $j_{12}$ in the figure) are involved in the calculation.}
\label{fig:sph_particles}
\end{figure}

Estimating the gradient, divergence, and Laplacian of field \textbf{A} using the standard SPH discretization can be done as follows
\begin{equation}
    \begin{aligned} 
        \nabla \mathbf{A}_i &= \sum _{j \in N(i)} \frac{m_j}{\rho _j}\mathbf{A}_j \otimes \nabla W_{ij}, \\
        \nabla \cdot \mathbf{A}_i &= \sum _{j \in N(i)} \frac{m_j}{\rho _j}\mathbf{A}_j\nabla W_{ij},\\
        \nabla ^2 \mathbf{A}_i &= \sum _{j \in N(i)} \frac{m_j}{\rho _j}\mathbf{A}_j \nabla ^2 W_{ij}, \\
    \end{aligned}
\end{equation}
where $\mathbf{a} \otimes \mathbf{b} = \mathbf{a}\mathbf{b}^T$.

\subsection{Implicit mixture model with improved phase transfer}
\label{BB}
%

In this section, we introduce the implicit mixture model (Sec.~\ref{BB.1}) and our improved phase transfer scheme (Sec.~\ref{BB.2}).

\subsubsection{Implicit Mixture Model (IMM)}
\label{BB.1}
The mixture model uses the volume fraction scheme (Fig.~\ref{fig:vol_frac}) to represent the concentration of each phase and calculates the physical parameters at the phase level and the mixture level using multiphase fluid dynamics.
%
\begin{figure}[htbp]
\centering
\includegraphics[width=0.4\columnwidth]{pics/vol frac.png}
\caption{Multi-phase Particle. This figure illustrates the volume fraction scheme in a mixture model for a two-phase flow, where the different phases are represented by distinct colors.}
\label{fig:vol_frac}
\end{figure}

The sum of the phase volume fractions $\alpha_k$, for all $k$ existing phases of a particle $i$, is normalized as
\begin{equation}
    \sum_k \alpha_{i,k} = 1.
\end{equation}
Here and next, we use the notation $Q_{i,k}$ to denote the quantity $Q$ related to phase $k$ measured for particle $i$.

The velocity field at the mixture level is reconstructed from the phase-level velocity fields $\mathbf{v}_k$ via
\begin{equation}
    \mathbf{v}_{i,\mathrm{mix}} = \sum_k \alpha _{i,k} \mathbf{v}_{i,k}.
\label{eq:mix_vel}
\end{equation}

The density of the mixture particle is computed as
\begin{equation}
    \rho_{i,\mathrm{mix}} = \sum_k \alpha_{i,k} \rho_k^0,
\label{eq:mix_dens}
\end{equation}
where $\rho_k^0$ %\sout{is the rest density of phase $k$.} 
represents the density of the substance of phase k.

The velocity field at the mixture level $\mathbf{v}_\mathrm{mix}$ is used to represent the actual fluid motion, which depends on the calculation of the physical field at the phase level. Here, we use the implicit mixture model in~\cite{xu2023implicitly}, where an implicit reconstruction method between mixture level and phase level was derived to achieve high numerical accuracy. 
This model considers the effects of gravity, pressure, and viscous forces. %\sout{The gravity $\mathbf{g}$ belongs to the volume force and is applied equally to each phase.} 
Pressure, viscous and external forces are computed for each particle $i$ and phase $k$ as
%
\begin{equation}
        \frac{D\mathbf{v}_{i,k}^p}{Dt} = \frac{\mathbf{M}_{i,\mathrm{mix}}^p}{\rho_{i,\mathrm{mix}}} \left(C_d+(1-C_d)\frac{\rho_{i,\mathrm{mix}}}{\rho_k^0}\right),
\label{eq:p_force}
\end{equation}
\begin{equation}
    \frac{D\mathbf{v}_{i,k}^\nu}{Dt} = C_d\frac{\mathbf{M}_{i,\mathrm{mix}}^\nu}{\rho_{i,\mathrm{mix}}} + (1-C_d)\frac{\mathbf{M}_{i,k}^\nu}{\alpha_{i,k} \rho_k^0},
\end{equation}
\begin{equation}
    \frac{D\mathbf{v}_{i,k}^\mathrm{ext}}{Dt} = g,
\end{equation}
%
where $\frac{D\mathbf{v}_k}{Dt}$ denotes the acceleration associated with different forces; superscripts $p, \nu, \mathrm{ext}$ denote pressure, viscosity and external force, respectively; $\mathbf{M}$ represents the momentum source; and $C_d \in [0, 1]$ is the model parameter derived from the implicit mixture model, used to adjust the degree of influence on the mixture of each phase. 

\smallskip

\subsubsection{Improved Phase Transfer}
\label{BB.2}
The phase transfer in the mixture model mainly involves two factors: interphase drag force and diffusion. The calculation of drag force depends on the drift velocity of the phase, which is defined as 
\begin{equation}
    \mathbf{v}_{i,k}^\mathrm{drift} = \mathbf{v}_{i,k} - \mathbf{v}_{i,\mathrm{mix}}.
\label{eq:drift_vel}
\end{equation}
%
The change of phase fraction due to the two factors is given by
\begin{equation}
    \begin{aligned} 
        \frac{D\alpha_{i,k}}{Dt} &= -\sum_{j\in N(i)} V_0(\alpha_{i,k}\mathbf{v}_{i,k}^\mathrm{drift}+\alpha_{j,k}\mathbf{v}_{j,k}^\mathrm{drift})\nabla \cdot W_{ij}, \\
        \nabla ^2 \alpha_{i,k} &= C_f\sum_{j\in N(i)} (\alpha_{i,k}-\alpha_{j,k})\frac{\mathbf{x}_{ij}\cdot \nabla W_{ij}}{\|\mathbf{x}_{ij}\|^2 + \epsilon},
    \end{aligned}
\label{eq:phase_trans}
\end{equation}
%
where $C_f$ is the diffusion coefficient; $V_0$ denotes the rest volume of a particle; $\mathbf{x}$ is the position of the particle; $\mathbf{x}_{ij}=\mathbf{x}_i-\mathbf{x}_j$; and $\epsilon$ is a small regularization constant (e.g. $\epsilon \simeq 0.001$).

In the original implicit mixture model, $C_d$ is used to adjust the degree of influence of the mixture on phases: when $C_d=0$, all phases are completely unaffected by the mixture; when $C_d=1$, all phases are completely controlled by the mixture. In the physical field, the influence of the mixture on phases is understood as the relationship between the velocity field of mixture-level and phase-level. When phase velocity fields completely follow the mixture, phases do not separate, which affects phase transport. The original implicit mixture model~\citep{xu2023implicitly} sets $C_d$ as a constant, indicating that the effect of the mixture on the phase is constant. However, this is not true in some solutions that react, such as bone cement and cornstarch water. Hence, a mechanism is needed to compute the effect of the solute concentration on the phase transfer.

Calculating the exact intermolecular combination between two molecules can lead to a large overhead. To simplify this computation, we propose to use a \textit{bonding effect network} (see Fig.~\ref{fig:phase_trans}). Specifically, we change the $C_d$ value of the multiphase particle dynamically according to the solute concentration. For one mixture, a basic $C_d^0$ value is set; we next estimate the current dynamic $C_d$ by the SPH method. Given our application to bone cement, we focus on two-phase fluids, where $\alpha_{k_1}$ denotes the liquid phase and $\alpha_{k_2}$ denotes the polymer phase. As such, we have
%
\begin{equation}
    C_d = C_d^0 + (1-C_d^0)\sum_{j\neq i} V_0 \alpha_{j,k_2} W_{ij}.
\label{eq:update_Cd}
\end{equation}
%
When the solute concentration around a particle is high, $C_d \rightarrow 1$. This will block the phase transfer to simulate the case of phase coupling.

\begin{figure}[t]
\centering
\includegraphics[width=0.5\columnwidth]{pics/phase trans.png}
\caption{Bonding Effect Network. The network models the coupling of phases in an inhomogeneous solution where chemical reactions take place. Using the example of three concentration regions, a greener particle color represents a higher fraction of the polymer phase. Arrows denote the blocking effect exerted by neighboring particles on the phase transfer of the target particle $i$, with thicker arrows indicating a stronger blocking influence.}
\label{fig:phase_trans}
\end{figure}

\begin{figure}[t]
\centering
\includegraphics[width=0.45\columnwidth]{pics/cap curve.png}
\caption{Depiction of the shear thinning behavior predicted by our model. As the shear rate increases, the viscoelastic properties of the shear thinning fluid exhibit a decreasing trend. In the figure, $\gamma$ represents the shear thinning coefficient, which quantifies the extent of shear thinning behavior in the fluid. In our model, a larger value of $\gamma$ corresponds to a more pronounced shear thinning effect.}
\label{fig:cap_curve}
\end{figure}

\subsection{Polymer conformation tensor method in the mixture model (IMM-CT)}
\label{CC}
%
The conformation tensor is a tool used for describing the material distribution in solutions~\citep{bejan2005constructal}, both for Newtonian and non-Newtonian fluids. The classical configuration update formula for the conformation tensor $\mathbf{U}$ (a $3\times3$ matrix in the 3D case) is
%
\begin{equation}
    \frac{D\mathbf{U}}{Dt} = \mathbf{U}\nabla \mathbf{v}+(\nabla \mathbf{v})^T\mathbf{U}-\frac{1}{\lambda}(\mathbf{U}-\mathbf{I}),
\label{eq:update_U}
\end{equation}
%
where $\lambda$ denotes the relaxation time used to describe the fluid viscoelasticity and $\mathbf{I}$ is the identity matrix of the conformation tensor, representing the undeformed state of the object. However, this model can only describe Newtonian fluids; the viscoelasticity of non-Newtonian fluids has a nonlinear relationship with the shear rate. 
To model this, we use an alternative model given by
\begin{equation}
    \frac{D\mathbf{U}}{Dt} = \mathbf{U}\nabla \mathbf{v}+(\nabla \mathbf{v})^T\mathbf{U}-\frac{1}{\lambda}(\mathbf{U}-\mathbf{I}) - \gamma(\mathbf{U}-\mathbf{I})\mathbf{U},
\end{equation}
%
where $\gamma(\mathbf{U}-\mathbf{I})\mathbf{U}$ is a non-linear term that can model shear thinning, which is the key component to improve the model; and $\gamma \in [0,1]$ is the thinning factor. A larger value of $\gamma$ contributes to a stronger thinning effect (Fig.~\ref{fig:cap_curve}).

The mixture-level stress based on the conformation tensor is defined as
%
\begin{equation}
    \mathbf{\tau}_{i,\mathrm{mix}} = c\eta_s(\mathbf{U}_i-\mathbf{I}),
\label{eq:stress}
\end{equation}
%
where $c$ denotes the polymer concentration of the solution, which is equal to $\alpha _{k_2}$ in our multiphase framework; and $\eta _s$ denotes the viscosity of the solution. We use the  symmetric formulation of SPH~\citep{monaghan2005smoothed} to calculate the stress force as
%
\begin{equation}
    \nabla \cdot \mathbf{\tau}_{i,\mathrm{mix}} = \rho_{i,\mathrm{mix}} \sum _{j\in N(i)} \left( \frac{\mathbf{\tau}_{i,\mathrm{mix}}}{\rho^2_{i,\mathrm{mix}}} + \frac{\mathbf{\tau}_{j,\mathrm{mix}}}{\rho^2_{j,\mathrm{mix}}}\right)\nabla W_{ij}. 
\label{eq:stress_force}
\end{equation}

The conformation tensor method does not define the stress calculation for each phase, and the multiphase framework requires reconstructing the mixture-level velocity from the phase-level velocity. Based on the normalization condition $\mathbf{\tau}_\mathrm{mix} = \sum_k \mathbf{\tau}_ k$, we obtain 
%
\begin{equation}
    \mathbf{\tau}_{i,k} = \alpha_{i,k} \mathbf{\tau}_{i,\mathrm{mix}}.
\end{equation}
%
In this way, the phase velocity can be updated according to the mixture-level stress as
\begin{equation}
    \frac{D\mathbf{v}_{i,k}^\mathrm{visc}}{Dt}=\frac{\nabla \cdot \tau_{i,k}}{\alpha_{i,k} \rho_ k^0},
\label{eq:visc_force}
\end{equation}
%
where $\frac{D\mathbf{v}_k^\mathrm{visc}}{Dt}$ represents the acceleration associated with the viscoelastic force.

Algorithm~\ref{algo:algo} outlines our end-to-end simulation and visualization method called IMM-CT (Implicit Mixture Model with Conformation Tensor). At the start (step a)), we perform some preparatory work, including bone modeling (Sec.\ref{00}) and setting the scene and model parameters. When initializing the model parameters, $\mathbf{U}$ is set to $\mathbf{I}$, and a uniform grid is used to update the neighbors $N(i)$ of each particle $i$. The grid size corresponds to the entire simulation space, with the cell size equal to the SPH support radius. 
Then, step b) solves the dynamic parameters of the multiphase fluid according to the initial input, including pressure, gravity, viscoelastic forces, and implements phase transport. Finally, in step c) the fluid surface is reconstructed and rendered with 3D-tools (e.g.\ Blender).

\begin{algorithm}[htbp]
\small
\caption{Simulation and visualization algorithm of bone cement injection.}
\begin{algorithmic}[l]

\State \textbf{a) Preparation:}
    \State \hspace{2mm} 1. Modeling bone with different bone density
    
    \State \hspace{2mm} 2. Set injection direction, speed, and position

    \State \hspace{2mm} 3. Import the particle models and initialize the fluid solver

\State \textbf{b) Bone cement modeling and motion solution:}
\State --Start Sim Loop--   
    \State \hspace{2mm} \textbf{1. Pressure computation}
        \State \hspace{4mm} compute div-free force $\mathbf{M}_{i,\mathrm{mix}}^{p}$ %\sout{using \textbf{VFSPH}} 
        \State \hspace{4mm} update phase velocity $\mathbf{v}_{i,k}$ \Comment{Eq.~\ref{eq:p_force}}
        \State \hspace{4mm} update mixture velocity $\mathbf{v}_{i,\mathrm{mix}}$ \Comment{Eq.~\ref{eq:mix_vel}}

    \State \hspace{2mm} \textbf{2. Advection}
        \State \hspace{4mm} update phase velocity using $\mathbf{v}_{i,k} \gets \mathbf{v}_{i,k} + \mathbf{g}\Delta t$
        \State \hspace{4mm} update mixture velocity $\mathbf{v}_{i,\mathrm{mix}}$ \Comment{Eq.~\ref{eq:mix_vel}}

    \State \hspace{2mm} \textbf{3. Viscoelasticity computation}
        \State \hspace{4mm} update ${C_d}_i$ \Comment{Eq.~\ref{eq:update_Cd}}
        \State \hspace{4mm} update conformation tensor $\mathbf{U}_i$ \Comment{Eq.~\ref{eq:update_U}}
        \State \hspace{4mm} compute viscoelastic force \Comment{Eq.~\ref{eq:stress},~\ref{eq:stress_force}}
        \State \hspace{4mm} update phase velocity $\mathbf{v}_{i,k}$ \Comment{Eq.~\ref{eq:visc_force}}
        \State \hspace{4mm} update mixture velocity $\mathbf{v}_{i,\mathrm{mix}}$ \Comment{Eq.~\ref{eq:mix_vel}}

    \State \hspace{2mm} \textbf{4. Final step }
        \State \hspace{4mm} update phase volume fraction $\alpha_{i,k}$ \Comment{Eq.~\ref{eq:phase_trans}}
        \State \hspace{4mm} update phase drift velocity $\mathbf{v}^\mathrm{drift}_{i,k}$ \Comment{Eq.~\ref{eq:drift_vel}}
        \State \hspace{4mm} update particle position using $\mathbf{x}_i \gets \mathbf{x}_i +\mathbf{v}_{i,\mathrm{mix}}\Delta t$
        \State \hspace{4mm} update neighbors $N(i)$
        \State \hspace{4mm} store the point cloud $\mathbf{x}_i$ according to the output frame rate
\State --End Sim Loop--   
\State \textbf{c) Render:} From the simulated point clouds $\{\mathbf{x}_i\}$, we finally reconstruct and render the fluid surface.

\end{algorithmic}
\label{algo:algo}
\end{algorithm}

\section{Experiments}
\label{sec:experiments}
%
We tested our method by simulating various modulation and mixing/injection processes of bone cement. In the following experiments, we set the parameters of our method to the values given in Tab.~\ref{tab:params}. We designed two sets of experiments to verify the advantages of our model, as described next. All algorithms were implemented using C++17 and CUDA 11.6, and executed on an RTX 4070 GPU.

\smallskip

\begin{figure}[htbp]
\centering
 \includegraphics[width=0.4\columnwidth]{pics/mix scene.png}
\caption{Modulation mixing scenario modeling. We set up two emitters for the mixing scene, one for the solvent and one for the polymer solute, with red arrows indicating the direction of emission.}
\label{fig:model_1}
\end{figure}

\begin{figure}[htbp]
\centering
\includegraphics[width=\columnwidth]{pics/injection scene.png}
\caption{Bone cement injection scenario. The left part shows a model of a medical screw used for injection, with six injection ports at the bottom of the screw. The right part represents the unified scenario setup after bone modeling, which includes three components: cortical bone, cancellous bone, and an invasive injection screw. All models are eventually sampled into particle models.}
\label{fig:model_2}
\end{figure}

\begin{table}[htbp]
\centering
\caption{Model parameters}
\begin{tabular}{lll}
\hline
Parameter & Meaning & Range\\
\hline
$C_f$ & diffusion coefficient & [0, 1]\\
$C^0_d$ & rest drag coefficient &  [0, 1]\\
$\gamma$ & phase volume fraction threshold & [0, 1]\\
$\eta_s$ & rest viscosity of solution & (0, 15)\\
$\lambda$ & relaxation time & (0, 1)\\
\hline
\end{tabular}
\label{tab:params}
\end{table}

\noindent\textbf{Modulation mixing:} We conducted fluid mixing experiments with different viscosity settings and presented two representative groups among them. Considering the characteristics of the two substances used in preparing bone cement, we configured the fluid into two parts: solution and solute. The modeling of this scenario is shown in Fig.~\ref{fig:model_1}. The actual visualization results are presented in Fig.~\ref{fig:mix}. In all scenarios, the viscosity of the solution was set to $0.01 Pa\cdot s$, while the viscosity of the solute was set to $5 Pa\cdot s$ in Fig.~\ref{fig:mix}. a) and to $10 Pa\cdot s$ in Fig.~\ref{fig:mix}. b). The experiments demonstrate that our method shows higher stability compared to the state-of-the-art single-phase and multiphase fluid solvers. Additionally, we provide performance comparison data and actual timing cost for each solver in Fig.~\ref{fig:perf_curve} and Tab.~\ref{tab:sim timing}. The curves indicate that our solver exhibits superior performance across a broader range of fluid viscosities.

\begin{figure*}[htbp]
\centering
\includegraphics[width=\columnwidth]{pics/pdf/compare.pdf}
\caption{Stability Comparison of Different Methods. The stability of the simulation method is reflected in the ability to obtain good simulation results under various settings. The figure presents the visualization results of fluid mixing scenarios under two different viscosity settings using three different methods. The color of the fluid particles transitions from the solvent (green) to the solute (blue). For DFSPH, the particle color is defined by the initial settings, while for IMM and IMM-CT, dynamic mixing colors can be displayed. In each experiment, the same maximum possible simulation time step was used for all methods, to reduce computation costs. a) When the viscosity ratio of the solvent to the solute is 500, the fluid surface using the DFSPH method can no longer remain stable, resulting in a large number of unexpectedly splashing particles, while IMM and IMM-CT can mix the fluids stably; b) When the viscosity ratio is 1000, the DFSPH method remains generally stable, but the fluid surface is still too active and not representative of real conditions. The IMM method can no longer simulate stably, while the IMM-CT method continues to maintain stability. For lower viscosity ratios, all three methods can simulate stably, but they do not accurately represent the bone cement mixing scenario. For higher viscosity ratios, the IMM-CT method performs well, while the DFSPH and IMM methods lead to simulation crashes.}
\label{fig:mix}
\end{figure*}

\begin{figure}[htbp]
\centering
\includegraphics[width=0.8\columnwidth]{pics/perf curve.png}
\caption{Comparison of simulation overhead among different methods. In the figure, a) the IMM-CT curve illustrates our proposed method, while the IMM curve depicts the original implicit mixture model, which is the best 'mixture model'-based approach, and the DFSPH curve indicates the divergence-free SPH solver, which is the best single-phase fluid solver. The y-axis denotes the maximum permissible time step during simulation; given that total simulation time remains constant, a larger time step signifies enhanced solver efficiency. The x-axis represents the viscosity ratio between the two phases. Our method can sustain a larger simulation time step across a wide range of high viscosity ratios, thereby achieving superior solver performance and significantly reducing overall simulation time (showed in Tab. \ref{tab:sim timing}).}
\label{fig:perf_curve}
\end{figure}

\begin{table}[htbp]
\centering
\caption{Simulation timing costs}
\begin{tabular}{lll}
\hline
Scenario & Method & \makecell{Timing cost for \\ 1s animation(s)}\\
\hline
\multirow{3}{*}{Fig.10.a)} & DFSPH & 176.135 \\
& IMM & 122.368 \\
& IMM-CT & 59.582 \\
\multirow{3}{*}{Fig.10.b)} & DFSPH & 241.275 \\
& IMM & 183.466 \\
& IMM-CT & 93.057 \\
\hline
\end{tabular}
\label{tab:sim timing}
\end{table}

\smallskip 

\noindent\textbf{Injection:} We used vertebral filling as a demonstration scenario. Our bone modeling approach is illustrated in Fig.~\ref{fig:modeling}. After obtaining the porous cancellous bone and the dense cortical bone, We used actual screw models provided by the hospital and applied the same particle sampling process to obtain the corresponding particle models, with the scene setup shown in Fig.~\ref{fig:model_2}. We also developed relevant particle emitters to simulate the six injection ports on the screws, these emitters can be configured to set the emission direction, emission speed, and outlet size, among other parameters. For the injection scheme, we demonstrated the visualization results of bone cement injections with two different viscosities, as shown in Fig.~\ref{fig:inject}. 

Our method allows for a clear observation of the bone cement filling process and the fluid boundaries, providing valuable insights for doctors in selecting the optimal injection strategy. Additionally, our approach offers great flexibility in setting up the injection scenario, allowing for customizable screw positioning, injection outlet direction, injection speed, fluid viscosity, and other parameters.

\begin{figure*}[htbp]
\centering
\includegraphics[width=\columnwidth]{pics/inject.png}
\caption{Visualization of Bone Cement Injection Scenarios. a) We present the injection visualization results for two different viscosities of bone cement. The upper set displays the injection predictions for high-viscosity bone cement, while the lower set represents the scenario with low-viscosity bone cement. Each set of experiments includes RGB-rendered images from two different viewpoints. The bone cement fluid is rendered in green, and the bone tissue is visualized with a semi-transparent material. The image sequences are arranged chronologically, with the horizontal axis representing simulation timestamps, where each column corresponds to the same time point. b) We provide a more detailed representation of the filling state at time $t_5$. In the top view and side view of the figure, the areas outlined in red indicate the fluid boundaries, the Zoom-in view shows the specific shape of the fluid. These various pieces of information are crucial for orthopedic surgeons in evaluating the feasibility of the surgical plan.}
\label{fig:inject}
\end{figure*}

% \steffen{in the abstract, for example, we claim that we are able to "obtain and dynamic bone cement effects with high accuracy"; it did not become clear to me how we evaluate accuracy here}

\section{Conclusion}
\label{sec:Lconclusion}

We proposed a comprehensive visualization method for bone cement injection surgery. The method mainly includes bone modeling, development of multiphase non-Newtonian fluid algorithm, and rendering. Our bone modeling method can easily model trabecular bone with different densities and merge them into a unified model. Our multiphase non-Newtonian fluid simulator can efficiently and stably simulate the flow of bone cement with a wide viscosity range. In the rendering phase, the color scheme we choose also matches the real material. Our method covers the entire process of injection surgery, including bone cement mixing, bone modeling, scene setup, and visualization, which can help doctors implement better surgical plans and conduct low-cost simulation training tasks.

Future work includes extending the model to support solidification and other stages in the flow phase of bone cement. An equally interesting direction is to incorporate the treatment of tension and capillary forces which play an important role in real materials. Additionally, there needs to be increased collaboration with doctors to apply it in clinical practice.

\section{Acknowledgements}
\label{sec:Acknowledgements}
This research was supported by the National Natural Science Foundation of China (No.\ 62306032), the National Key Research and Development Program of China (No.\ 2022ZD0118001), the Guangdong Basic and Applied Basic Research Foundation (No.\ 2022A1515110350), and the Interdisciplinary Research Project for Young Teachers of USTB (No.\ FRF-IDRY-22-025). The computing work is partly supported by the MAGICOM Platform of Beijing Advanced Innovation Center for Materials Genome Engineering.

% To print the credit authorship contribution details
\printcredits

\clearpage

%% Loading bibliography style file
%\bibliographystyle{model1-num-names}
\bibliographystyle{cas-model2-names.bst}

% Loading bibliography database
\bibliography{cas-refs.bib}

% Biography
%\bio{}
% Here goes the biography details.
%\endbio

%\bio{pic1}
% Here goes the biography details.
%\endbio

\end{document}

