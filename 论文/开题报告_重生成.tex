
\documentclass[12pt]{article}
\usepackage[UTF8]{ctex}
\usepackage{amsmath,amssymb}
\usepackage{geometry}
\usepackage{setspace}
\usepackage{cite}
\geometry{a4paper,margin=2.5cm}
\setstretch{1.5}

\title{基于光滑粒子流体动力学的多相非牛顿流体的仿真与应用}
\author{申龙}
\date{}

\begin{document}
\maketitle

\section{研究背景及意义}

\subsection{研究背景}
计算机图形学作为连接数字世界与现实世界的桥梁,在影视制作、工业设计、医学仿真等领域具有重要作用。
随着真实性需求提升,基于物理的流体模拟逐渐成为研究重点。

流体可分为牛顿流体与非牛顿流体。牛顿流体粘度恒定,而非牛顿流体粘度随剪切率变化,
在医学和工业中更为常见,因此其仿真具有重要意义。

\subsection{研究意义}
当前图形学研究多集中于单相牛顿流体,多相非牛顿流体仍缺乏高效稳定的模拟方法。
SPH 方法具有无网格、质量守恒等优势,适合用于复杂流体建模。
本研究旨在构建高效、稳定的多相非牛顿流体仿真框架,并应用于骨水泥注射手术模拟。

\section{国内外研究现状}

\subsection{粘度计算方案}
Navier--Stokes 方程描述不可压缩流体运动:
\begin{equation}
\rho \left( \frac{\partial \mathbf{v}}{\partial t} + \mathbf{v}\cdot\nabla \mathbf{v} \right)
= -\nabla p + \mu \nabla^2 \mathbf{v} + \mathbf{f}
\end{equation}

SPH 中常用的粘度求解方法包括人工粘度、拉普拉斯算子离散和应变率张量方法。
其中隐式粘度方法可显著提升数值稳定性。

\subsection{数值优化方案}
近年来研究集中于隐式求解、多重网格加速以及统一构型模型,
以提升高粘性和非牛顿流体的计算效率和稳定性。

\subsection{多相流体模拟}
多相流体模型通过体积分数描述不同相的混合状态,
能够模拟材料混合、沉淀等复杂现象。
然而目前图形学领域尚缺乏针对多相非牛顿流体的系统研究。

\section{研究内容与预期目标}

\subsection{研究内容}
\begin{enumerate}
  \item 基于 SPH 的混合模型离散化计算
  \item 基于构象张量的多相非牛顿粘度模型
  \item 面向骨水泥注射手术的仿真与可视化算法
\end{enumerate}

\subsection{预期目标}
提出统一的多相非牛顿流体模拟算法,
并构建完整的骨水泥注射仿真系统,为医学应用提供辅助工具。

\section{关键问题与创新点}

\subsection{关键问题}
\begin{itemize}
  \item 混合模型与构象张量方法的稳定耦合
  \item 多模式构象张量的数值鲁棒性
  \item 多相边界条件下的稳定仿真
\end{itemize}

\subsection{创新点}
\begin{itemize}
  \item 提出新的两相与多相非牛顿流体仿真算法
  \item 引入构象相互影响机制增强物理真实性
  \item 构建面向医学场景的可视化仿真流程
\end{itemize}

\section{时间安排}
\begin{itemize}
  \item 2023.09--2024.01:文献调研与问题分析
  \item 2024.02--2025.01:算法研究与实验验证
  \item 2025.02--2026.06:系统实现与论文撰写
\end{itemize}

\bibliographystyle{plain}
\bibliography{references}

\end{document}
